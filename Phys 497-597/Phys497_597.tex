\documentclass[12pt, a4paper, oneside, openright, titlepage]{book}
\usepackage[utf8]{inputenc}
\raggedbottom
\usepackage{import}


%%%%%%%%%%%%%%%%% Book Formatting Comments:

%%%%%%%%%%%%%%%%%%%%%%%%%%%%%%%%%%%%% for Part

%%%%%%%%%%%%%%%%%%%%%% for chapter

%%%%%%%%%%%%%%%%%%%% for section




%%%%%% PACKAGES %%%%%%%
\usepackage{hyperref}
\hypersetup{
    colorlinks,
    citecolor=black,
    filecolor=black,
    linkcolor=black,
    urlcolor=black
}
\usepackage{amsmath} % Math display options
\usepackage{amssymb} % Math symbols
%\usepackage{amsfonts} % Math fonts
%\usepackage{amsthm}
\usepackage{mathtools} % General math tools
\usepackage{array} % Allows you to write arrays
\usepackage{empheq} % For boxing equations
% \usepackage{mathabx}
% \usepackage{mathrsfs}
\usepackage{nameref}
\usepackage{wrapfig}

\usepackage{soul}
\usepackage[normalem]{ulem}

\usepackage{txfonts}
\usepackage{cancel}
\usepackage[toc, page]{appendix}
\usepackage{titletoc,tocloft}
\setlength{\cftchapindent}{1em}
\setlength{\cftsecindent}{2em}
\setlength{\cftsubsecindent}{3em}
%\setlength{\cftsubsubsecindent}{4em}
\usepackage{titlesec}

%\titleformat{\section}
%  {\normalfont\fontsize{25}{15}\bfseries}{\thesection}%{1em}{}
%\titleformat{\section}
%  {\normalfont\fontsize{20}{15}\bfseries}%{\thesubsection}{1em}{}
%\setcounter{secnumdepth}{1}  
  
  

%\newcommand\numberthis{\refstepcounter{equation}\tag{\theequation}} % For equation labelling
\usepackage[framemethod=tikz]{mdframed}

\usepackage{tikz} % For drawing commutative diagrams
\usetikzlibrary{cd}
\usetikzlibrary{calc}
\tikzset{every picture/.style={line width=0.75pt}} %set default line width to 0.75p

\usepackage{datetime}
\usepackage[margin=2cm]{geometry}
\setlength{\parskip}{1em}
\usepackage{makeidx}         % allows index generation
\usepackage{graphicx}       % standard LaTeX graphics tool
\usepackage{multicol}        % used for the two-column index
\usepackage[bottom]{footmisc}% places footnotes at page bottom

\usepackage{newtxtext}       % 
\usepackage{newtxmath}       % selects Times Roman as basic font
\usepackage{float}
\usepackage{fancyhdr}
\setlength{\headheight}{15pt} 
\pagestyle{fancy}
\lhead[\leftmark]{}
\rhead[]{\leftmark}

%\usepackage{enumitem}

\usepackage{url}
\allowdisplaybreaks

%%%%%% ENVIRONMENTS %%%
\definecolor{purp}{rgb}{0.29, 0, 0.51}
\definecolor{bloo}{rgb}{0, 0.13, 0.80}



%%\newtheoremstyle{note}% hnamei
%{3pt}% hSpace above
%{3pt}% hSpace belowi
%{}% hBody fonti
%{}% hIndent amounti
%{\itshape}% hTheorem head fonti
%{:}% hPunctuation after theorem headi
%{.5em}% hSpace after theorem headi
%{}% hTheorem head spec (can be left empty, meaning ‘normal’)i





% %%%%%%%%%%%%% THEOREM DEFINITIONS

\spnewtheorem{axiom}{Axiom}[chapter]{\bfseries}{\itshape}

\spnewtheorem{construction}{Construction}[chapter]{\bfseries}{\itshape}

\spnewtheorem{props}{Properties}[chapter]{\bfseries}{\itshape}


\renewcommand{\qedsymbol}{$\blacksquare$}


\numberwithin{equation}{section}

\newenvironment{qest}{
    \begin{center}
        \em
    }
    {
    \end{center}
    }

%%%%%% MACROS %%%%%%%%%
%% New Commands
\newcommand{\ip}[1]{\left\langle#1\right\rangle} %%% Inner product
\newcommand{\abs}[1]{\lvert#1\rvert} %%% Modulus
\newcommand\diag{\operatorname{diag}} %%% diag matrix
\newcommand\tr{\mbox{tr}\.} %%% trace
\newcommand\C{\mathbb C} %%% Complex numbers
\newcommand\R{\mathbb R} %%% Real numbers
\newcommand\Z{\mathbb Z} %%% Integers
\newcommand\Q{\mathbb Q} %%% Rationals
\newcommand\N{\mathbb N} %%% Naturals
\newcommand\F{\mathbb F} %%% An arbitrary field
\newcommand\ste{\operatorname{St}} %%% Steinberg Representation
\newcommand\GL{\mathbf{GL}} %%% General Linear group
\newcommand\SL{\mathbf{SL}} %%% Special linear group
\newcommand\gl{\mathfrak{gl}} %%% General linear algebra
\newcommand\G{\mathbf{G}} %%% connected reductive group
\newcommand\g{\mathfrak{g}} %%% Lie algebra of G
\newcommand\Hbf{\mathbf{H}} %%% Theta fixed points of G
\newcommand\X{\mathbf{X}} %%% Symmetric space X
\newcommand{\catname}[1]{\normalfont\textbf{#1}}
\newcommand{\Set}{\catname{Set}} %%% Category set
\newcommand{\Grp}{\catname{Grp}} %%% Category group
\newcommand{\Rmod}{\catname{R-Mod}} %%% Category r-modules
\newcommand{\Mon}{\catname{Mon}} %%% Category monoid
\newcommand{\Ring}{\catname{Ring}} %%% Category ring
\newcommand{\Topp}{\catname{Top}} %%% Category Topological spaces
\newcommand{\Vect}{\catname{Vect}_{k}} %%% category vector spaces'
\newcommand\Hom{\mathbf{Hom}} %%% Arrows

\newcommand{\map}[2]{\begin{array}{c} #1 \\ #2 \end{array}}
\newcommand{\inner}[2]{\left\langle #1, #2\right\rangle}
\newcommand{\norm}[1]{\left|\left| #1\right|\right|}

\newcommand{\Emph}[1]{\textbf{\ul{\emph{#1}}}}




%% Math operators
\DeclareMathOperator{\ran}{Im} %%% image
\DeclareMathOperator{\aut}{Aut} %%% Automorphisms
\DeclareMathOperator{\spn}{span} %%% span
\DeclareMathOperator{\ann}{Ann} %%% annihilator
\DeclareMathOperator{\rank}{rank} %%% Rank
\DeclareMathOperator{\ch}{char} %%% characteristic
\DeclareMathOperator{\ev}{\bf{ev}} %%% evaluation
\DeclareMathOperator{\sgn}{sign} %%% sign
\DeclareMathOperator{\id}{Id} %%% identity
\DeclareMathOperator{\supp}{supp} %%% support
\DeclareMathOperator{\inn}{Inn} %%% Inner aut
\DeclareMathOperator{\en}{End} %%% Endomorphisms
\DeclareMathOperator{\sym}{Sym} %%% Group of symmetries


%% Diagram Environments
\iffalse
\begin{center}
    \begin{tikzpicture}[baseline= (a).base]
        \node[scale=1] (a) at (0,0){
          \begin{tikzcd}
           
          \end{tikzcd}
        };
    \end{tikzpicture}
\end{center}
\fi

\usepackage[most]{tcolorbox}
%%%%%%%%%%%%%% ENVIRONMENTS
%%% tcolorbox environments

%%% \tcolorboxenvironment: An existing environment is redefined to be allowed to be boxed inside another tcolorbox during use
\tcolorboxenvironment{proof}{% `proof' from `amsthm'  
blanker,breakable=true,left=0.3em,
borderline west={3pt}{-3pt}{black!35}}


%%% titled environments
\newtcolorbox[auto counter, number within=section]{nthm}[2][]{
colback=red!5!white,
colframe=red!5!white,
attach title to upper={},
title={\textbf{Theorem \thetcbcounter\  (#2)\ }},
coltitle={black},
sharp corners,
breakable=true,
enhanced jigsaw,
borderline west = {3pt}{-3pt}{red!75!black},
#1
}

\newtcolorbox[use counter from=nthm, number within=section]{ndefn}[2][]{
colback=blue!5!white,
colframe=blue!5!white,
attach title to upper={},
title={\textbf{Definition \thetcbcounter\  (#2)\ }},
coltitle={black},
sharp corners,
breakable=true,
enhanced jigsaw,
borderline west = {3pt}{-3pt}{blue!75!black},
#1
}

\newtcolorbox[auto counter, number within=section]{eg}[1][]{
colback=white,
attach title to upper={},
title={\underline{Example:}\smallbreak},
coltitle={green!50!black},
sharp corners,
breakable=true,
enhanced jigsaw,
frame hidden, 
borderline west = {3pt}{-3pt}{green!50!black},
#1
}

\newtcolorbox[auto counter, number within=section]{qst}[1][]{
colback=white,
attach title to upper={},
title={\underline{Question?}\smallbreak},
coltitle={blue!50!black},
sharp corners,
breakable=true,
enhanced jigsaw,
frame hidden, 
borderline west = {3pt}{-3pt}{blue!50!black},
#1
}

\newtcolorbox[auto counter, number within=section]{rmk}[1][]{
colback=white,
attach title to upper={},
title={\underline{Remark:}\smallbreak},
coltitle={red!50!black},
sharp corners,
breakable=true,
enhanced jigsaw,
frame hidden, 
borderline west = {3pt}{-3pt}{red!50!black},
#1
}


\newtcolorbox[auto counter, number within=section]{nte}[1][]{
colback=white,
attach title to upper={},
title={\underline{Note:}\smallbreak},
coltitle={red!25!white},
sharp corners,
breakable=true,
enhanced jigsaw,
frame hidden, 
borderline west = {3pt}{-3pt}{red!25!white},
#1
}


%%% untitled environments
\newtcolorbox[use counter from=nthm, number within=section]{thm}[1][]{
colback=red!5!white,
colframe=red!5!white,
attach title to upper={},
title={\textbf{Theorem \thetcbcounter\ }},
coltitle={black},
sharp corners,
breakable=true,
enhanced jigsaw,
borderline west = {3pt}{-3pt}{red!75!black},
#1
} 

\newtcolorbox[use counter from=nthm, number within=section]{defn}[1][]{
colback=blue!5!white,
colframe=blue!5!white,
attach title to upper={},
title={\textbf{Definition \thetcbcounter\ }},
coltitle={black},
sharp corners,
breakable=true,
enhanced jigsaw,
borderline west = {3pt}{-3pt}{blue!75!black},
#1
}

\newtcolorbox[use counter from=nthm, number within=section]{cor}[1][]{
colback=red!5!white,
colframe=red!5!white,
attach title to upper={},
title={\textbf{Corollary \thetcbcounter\ }},
coltitle={black},
sharp corners,
breakable=true,
enhanced jigsaw,
borderline west = {3pt}{-3pt}{red!50!black},
#1
}

\newtcolorbox[use counter from=nthm, number within=section]{prop}[1][]{
colback=orange!5!white,
colframe=orange!5!white,
attach title to upper={},
title={\textbf{Proposition \thetcbcounter} \ },
coltitle={black},
sharp corners,
breakable=true,
enhanced jigsaw,
borderline west = {3pt}{-3pt}{orange!75!black},
#1
}

\newtcolorbox[use counter from=nthm, number within=section]{lem}[1][]{
colback=orange!5!white,
colframe=orange!5!white,
attach title to upper={},
title={\textbf{Lemma \thetcbcounter} \ },
coltitle={black},
sharp corners,
breakable=true,
enhanced jigsaw,
frame hidden, 
borderline west = {3pt}{-3pt}{orange!75!black},
#1
}

\newtcolorbox[auto counter, number within=section]{home}[1][]{
colback=white,
attach title to upper={},
title={\textbf{Exercise:}\ \ },
coltitle={black},
sharp corners,
breakable=true,
enhanced jigsaw,
frame hidden, 
borderline west = {3pt}{-3pt}{black!10!lime},
#1
}


\newdateformat{monthdayyeardate}{%
    \monthname[\THEMONTH]~\THEDAY, \THEYEAR}
%%%%%%%%%%%%%%%%%%%%%%%


%%% Specific Macros %%%


%%%%%% BEGIN %%%%%%%%%%


\begin{document}

%%%%%% TITLE PAGE %%%%%

\begin{titlepage}
    \centering
    \scshape
    \vspace*{\baselineskip}
    \rule{\textwidth}{1.6pt}\vspace*{-\baselineskip}\vspace*{2pt}
    \rule{\textwidth}{0.4pt}
    
    \vspace{0.75\baselineskip}
    
    {\LARGE Physics Lab Techniques}
    
    \vspace{0.75\baselineskip}
    
    \rule{\textwidth}{0.4pt}\vspace*{-\baselineskip}\vspace{3.2pt}
    \rule{\textwidth}{1.6pt}
    
    \vspace{2\baselineskip}
    Phys 497-597 \\
    \vspace*{3\baselineskip}
    \monthdayyeardate\today \\
    \vspace*{5.0\baselineskip}
    
    {\scshape\Large E/Ea Thompson(they/them), \\ Physics and Math Honors\\}
    
    \vspace{1.0\baselineskip}
    \textit{Solo Pursuit of Learning}
    \vfill
    \enlargethispage{1in}
    \begin{figure}[b!]
    \makebox[\textwidth]{\includegraphics[width=\paperwidth, height =10cm]{../Crab.jpg}}
    \end{figure}
\end{titlepage}

%%%%%%%%%%%%%%%%%%%%%%%
\tableofcontents

%%%%%%%%%%%%%%%%%%%%%%%%%%%%%%%%%%%%% Part 0
\part{Uncertainty and Measurement}

\chapter{Naive Undergraduate Uncertainty}



%%%%%%%%%%%%%%%%%%%% Section A.1
\section{Uncertainty and Measurement}

Measurement is a process that results in a value that can be attributed to a quantitative property of a phenomenon, object or substance. The object of the measurement (\Emph{measurand}) has the property that it has a numerical magnitude and a reference that gives its meaning (its \Emph{unit}). The process involves a comparison with an accepted standard, unless one does not yet exist. Instruments for measurement need to be calibrated so that they properly relate to the relevant standard.

\begin{qst}
    How do we know if our measurements are adequate? 
\end{qst}

In a scientific experiment, measurement uncertainty is critical for determining whether an experiment agrees with a theoretical prediction or not. The BIPM (Bureau International des Poids et Mesures) ``Guide to the expression of uncertainty in measurement" (GUM) is a widely accepted standard.

\begin{defn}
    \Emph{Uncertainty} in a measurement is a parameter that characterizes the dispersion of values that could reasonably be attributed to the measured quantity.

    The term ``\Emph{standard uncertainty}" should be used to refer to uncertanties calculated according to the GUM guidelines.
\end{defn}

Uncertainty is a quantification of the doubt about the measurement result. 

\begin{rmk}
    \textbf{Note:} Uncertainty is distinct from \Emph{error}, which is the (generally unknown) difference between the measured value and the `true value' of the measured quantity. \Emph{Random errors} cause values to be randomly distributed above and below the true value. \Emph{Systematic errors} causes all measurements to be consistently over or underestimate.
\end{rmk}

Many factors may contribute to the uncertainty, including those arising from imperfect knowledge of systematic effects. Uncertainty creeps in many direction (see GUM 3.3.2). First we have limits to our \Emph{measurement devices}:
\begin{itemize}
    \item Your measuring device has a limited \Emph{resolution}.
    \item Your measuring device has a \Emph{calibration uncertainty}. How close is a millimeter on your ruler to an exact millimeter? Are there other factors factors (``influence quantities") such as temperature that might affect that uncertainty?
\end{itemize}
\noindent Even if your measuring device is arbitrarily precise, and well-calibrated, there are still \Emph{inherent errors} that cannot be avoided:
\begin{itemize}
    \item Errors often arise due to \Emph{incomplete definition} of the quantity being measured, or due to \Emph{imperfect realization} of that definition. For example, your height varies considerably based on your posture. Similarly, exactly which points on the feet and ehad are being considered will also affect the measurement of your height. If those factors aren't part of the definition of your ``height," or if they aren't perfectly controlled during the measurement, then they become a source of uncertainty.
    \item There may be \Emph{systematic effects} on the measurement that can only be corrected to a limited perfection. Systematic effects should always be corrected to the best of the experimenter's ability during analysis. Nonetheless, there will always be a limit to how well you know the systematic effects, and on how well you can correct the results. This contributes to an uncertainty due to incomplete knowledge of the required value of the correction. \Emph{Note:} the measurement uncertainty reflects your uncertainty in applying that correction, not the magnitude of the correction. Explicitly apply the correction during the analysis, not when writing down the raw data, and \Emph{always explain the correction in your report}.
\end{itemize}




\subsection{Type A Uncertainty}

Repeated measurements under apparently identical conditions often give variable results. There are limits on both \Emph{repeatability} (under exactly the same conditions in a short time) and \Emph{reproducibility} (under slightly different conditions or over a long time). The variance in repeated measurements gives a rigorous way to quantify many types of uncertainty.

\begin{defn}
    \Emph{Type A} uncertainty is derived through repeated measurements and statistical methods.

    \Emph{Type B} uncertainties include all methods of estimating uncertainty that are not included in Type A.
\end{defn}


The following are common statistical methods used for estimating Type A uncertainties:

\begin{defn}
    The arithmetic sample mean (average) of $n$ independent samples $X_1,...,X_n$ is $$\overline{X} =  \frac{\sum_{k=1}^nX_n}{n}$$
\end{defn}

\begin{defn}
    The best unbiased estimator for the \Emph{standard deviation} of a sample of $n$ independent random variables $X_1,...,X_n$ of sample mean $\overline{X}$ is $$s(X_k) = \sigma = \sqrt{\frac{\sum_{k=1}^n(X_k-\overline{X})^2}{n-1}}$$
    In excel, we can use \Emph{STDEV.S()}
\end{defn}

\begin{defn}
    The best estimate for the standard deviation, or standard uncertainty, of the mean $\overline{X}$ of a sample of $n$ independent random variables $X_1,...,X_n$ is $$u(\overline{X}) = \frac{s(X_k)}{\sqrt{n}}$$
\end{defn}

From these formulations we observe that Type A uncertainties decrease as sample sizes tend to infinity, which is a statement of the central limit theorem.

If repeated measurements are impractical, but some sources of uncertainty simply cannot be estimated through repeated measurements, Type B uncertainties are used. For example, calibration uncertainty is a Type B uncertainty that cannot be revealed through repeated measurements without access to a reference standard. Similarly, imperfect corrections for systematic effects typically result in Type B uncertainties.

\subsection{Quantifying Standard Uncertainties}

The Central Limit Theorem is the statement that in the limit as $n$ goes to infinite for a sample of $n$ independent identically distributed random variables $X_1,...,X_n$, the distribution of the sample means $\overline{X}$ follows a Gaussian or normal distribution with mean equal to the true mean of the $X_i$, and uncertainty equal to $\sigma/\sqrt{n}$, for $\sigma = s(X_k)$. 

In relation to the standard deviation for Type A uncertainties, we have the following rules of thumb:

\begin{prop}
    One can expect the measurement error, the unknown difference between the measurement error and the true value, to be less than the standard deviation about $68\%$ of the time. For two standard deviations we have a $95.4\%$ certainty, and for three standard deviations we have about a $99.7\%$ certainty.
\end{prop}

Note that uncertainties are not safety factors, nor confidence intervals. Often we want large than a $68\%$ confidence, so we multiply the standard deviation by a \Emph{coverage factor} $k$ to obtain a ``expanded uncertainty" with a higher level of confidence. For example, the discovery of the Higgs Boson used a coverage factor of $k = 5$ (corresponding to a $99.99994\%$ confidence)

\subsection{Type B Uncertainties}

Most uncertainties are Type B, simply because Type A uncertainties are often impractical. For example, the manufacturer's quoted uncertainty for a measurement device is a Type B uncertainty. Type B uncertainties should be honestly estimated to conform to the same definition of ``uncertainty" as for Type A, with the same amount of confidence (for example $68\%$). Rigorously this requires estimating the probability distribution and then assessing its standard deviation.

When evaluating a Type B uncertainty you should think ``\Emph{what range am I about $68\%$ sure that the measurement fallse in?}" Evidently, this is not something you can typically estimate to high accuracy, but an honest experiment requires an honest effort to evaluate the Type B uncertainties. \Emph{There is no rigorous recipe for how to estimate Type B uncertainties.} There is no one right way at estimation, but here are some examples:

\begin{eg}[Digital Quantization]
    With a digital instrument, there is always an uncertainty in the last digit. Assuming that every value in the range $[x-\Delta,x+\Delta]$ is equally likely, for $\Delta$ half the size of the unit of the last digit (i.e. half the size of the length between last digit values), we have a uniform distribution with uncertainty $u = (x+\Delta - (x-\Delta))^2/12 = \Delta/\sqrt{3}$. More approximately, $68\%$ of the probability is contained in the range $x \pm 0.68\Delta$, giving $u \approx 0.68\Delta$.
\end{eg}

\begin{eg}[Rulers and Analog Meters]
    This is a case where there is no ``right way." It will depend on how small the divisions on the ruler are, how good your eyesight is, and how closely you can align the object with the ruler.
\end{eg}

\begin{eg}[Half-Division Example]
    It is common to see uncertainty reported as half the smallest division on an analog instrument or half the last digit of a digital instrument. While frequent, this is not actually part of the GUM standard. Uncertainty of half a division is about typical for many analog measurements, and slightly too large for most digital measurements. It is usually a reasonable starting point, but should be avoided as a hard fast rule.
\end{eg}

\begin{eg}[Calibration Reports]
    Manufacturers may issue calibration reports for their instruments stating the uncertainty in its calibration. It may be stated as a standard uncertainty, but more often will be stated as a $95\%$ confidence interval $x \pm \Delta$, or sometimes even a $99\%$ confidence interval. In these cases we can use the standard rule of thumb to convert these to the standard uncertainty by taking either half or one third.
\end{eg}

\begin{eg}[Systematic Effects]
    You should always correct measurements for known systematic effects. The uncertainty reflects the uncertainty in that correction, not its magnitude.
\end{eg}

\begin{eg}[Event Counting]
    Counting the number of random events that occur in a given time is subject to statistical fluctuation. Some intervls will naturally have more events than others. This leads to an uncertainty in the measured event rate. The appropriate statistical distribution in this case is the Poisson distribution, which has a standard deviation equal to the square root of the mean. Therefore, if the observed count rate is $n$ then its uncertainty is $\sqrt{n}$.
\end{eg}


\subsection{Combining Uncertainties}

Often we will have multiple sources of uncertainty, $u_i$, that contribute to an overall total uncertainty in a measurement. Assuming these uncertainties are uncorrelated, they can be combined into a \Emph{combined standard uncertainty} $u_c$, defined by $$u_c = \sqrt{\sum_{i=1}^Nu_i^2}$$
In practice, one uncertainty source often dominates over the others, in which case the combined uncertainty is essentially just the dominant uncertainty.

It is important not to ``double-count" your uncertainties. The final uncertainty should be an honest assessment of your doubt about the measurement result, neither unrealistically large nor unrealistically small.

Sometimes the quantity of interest must be calculated from one or more measurements. In these cases we use a process of \Emph{propagation} of uncertainty.

\begin{defn}
    If the final quantity $r$ is a function of $N$ independent uncorrelated variables, $x_1,...,x_N$, with uncertainties $u(x_1),...,u(x_N)$, then the propagated standard uncertainty is $$u_c(r) = \sqrt{\sum_{i=1}^N\left(\frac{\partial r}{\partial x_i}\right)^2u(x_i)^2}$$
\end{defn}

\subsection{Reporting Uncertainties}

Reporting a measurement with a precision that greatly exceeds its measuremetn uncertainty gives a false sense of accuracy, and should be avoided. Unless there are good reasons to do otherwise, uncertainties should be reported to one significant digit. Measurements and uncertainties should be reported in the same units, and with the same number of decimal places. Using scientific notation if appropriate is advised.

\subsection{Comparing a Measurement with an Expected Value}

Experiments comparing a measured value $x$ to an expected value $\mu$ are often called \Emph{hypothesis tests} in statistics, where we test $x = \mu$ by asking ``what is the likelihood that $x$ and $\mu$ differ by at least the amount observed?" That is, we ask what is the likelihood that $x = \mu$, and they differ by the amount observed. This is often known as the Null Hypothesis, and if the likelihood is very low, say less than $5\%$ as standard, then the hypothesis is rejected and the values are said to disagree. In measured likelihood is called the $p$-value.

If the experiment is dominated by well-characterized Type A errors, then there is a mathematically rigorous way to test the hypothesis known as a ``Student's t-test". However, experiments are often not dominated by Type A errors. Instead we often use some reasonable approximations. First calculate the normalized difference $$t = \frac{x-\mu}{u_c(x)}$$ where $u_c(x)$ is the uncertainty in $x$. A large value of $t$ means that $x$ and $\mu$ are dissimilar, while a small value of $t$ means that $x$ is very similar to $\mu$. 

We then take the assumption that $x$ follows a normal distribution. Although not rigorously justified, we have constructed $u_c(x)$ specifically to make this assumption as defensible as possible. Under the normal assumption, there is an $\approx 68\%$ chance of observing $|t| < 1$, an $\approx 95.4\%$ chance of observing $|t| < 2$, and an $\approx 99.7\%$ chance of observing $|t| < 3$. If $|t|$ is extremely small, there is a high chance something went wrong in the experiment, and you have overestimated your uncertanties.

If we want to compare two measurements with uncertanties, we can use the modified $t$ value $$t = \frac{x_1 - x_2}{\sqrt{u_c(x_1)^2+u_c(x_2)^2}}$$ which is again an approximation so as to be reasonable for not just Type A uncertainties but Type B uncertanties as well.


\subsection{Line Fitting}

Often we want to quantify a linear relationship between two quantities. \Emph{Line fitting} is the problem of finding a line that describes the data well. Intuitively, the problem is equivalent to minimizing the distances between the points and the line. The most commonly-used method is ``least-squares" fitting, where the minimized quantity is the sum of the squares of the vertical distances between the data points and the fit line: $$\chi^2 = \sum_{i=1}^n[y_i - (mx_i + b)]^2$$ for data points $(x_i,y_i)$ and fit $y = mx+b$.

The closed form of the minimization for these parameters of the least squares fit is $$m = \frac{\sum_{i=1}^n(x_i-\overline{x})(y_i - \overline{y})}{\sum_{i=1}^n(x_i-\overline{x})^2}$$ and $$b = \overline{y} - m\overline{x}$$ These solutions turn out to be the most probable line if the errors in $y$ are normally-distributed and the errors in $x$ are negligible. In Excel this is implemented using the LINEST function. The LINEST function takes in arguments $$LINEST(known-ys,known-xs,const = TRUE,stats = TRUE)$$ If dealing with a general polynomial, we raise $known-xs$ to $\{...\}$ with the powers of $x$ inside, or $known-xs^COLUMN(\$a:\$n)$ for $n$ the highest power. We can also apply $ln$ to $known-xs$ for logarithmic fits. This command will return an array of values, the first row being coefficients, going from $a_n$ to $a_0$ (left to right), and then uncertainties left to right in the second row, and the $R^2$ value in the third row.

When dealing with a nonlinear function, we often attempt to convert it to a linear form by changing the dependent/independent variables being graphed on either axis, and possibly taking logarithms if dealing with exponential relationships.

There is evidently an uncertainty associated with the values of the fitted coefficients $m$ and $b$. We can define a Type A uncertainty in $m$ based on the scatter of repeated measurements about a line. Intuitively, this makes sense: we should become less certain about $m$ as the scatter of the points about the line increases. This Type $A$ uncertainty is given by $$u(m) = \frac{\sum_{i=1}^n[y_i-(mx_i+b)]^2}{(n-2)\sum_{i=1}^n(x_i-\overline{x})^2}\;\;\;u(b) = u(m)\sqrt{\frac{1}{n}\sum_{i=1}^nx_i^2}$$ This is also implemented as part of the LINEST function in Excel.

\subsection{Line Fitting with Measurement Uncertainty}

The uncertainty estimate in the previous section has a few important restrictions:
\begin{itemize}
    \item It is strictly Type-A, and cannot account for fundamentally Type-B uncertainties such as uncertainties in calibration or uncertainties in correcting for a systematic effect.
    \item It weights all points equally, assuming they all have the same uncertainty
    \item It assumes the $x_i$ values are exact, attributing all uncertainty to $y_i$
\end{itemize}
These restrictions can affect both the uncertainties and the values of the $m$ and $b$. We can lift the first two restrictions in the list by propagating uncertainty values through the linear lin-fitting equations, which allows for inclusion of Type-B uncertanties and unequal uncertanties for individual points. 

The last restriction in the list comes from using the vertical distances instead of the uncertainty-weighted shortest distances in the definition of $\chi^2$. It becomes a valid assumption only if $$u(y) \gg\frac{dy}{dx}u(x)$$ which is only sometimes true. If the converse is true, consider simply reversing $x$ and $y$. If the uncertainties are comparable then you may combine the uncertainty in $u(x)$ with the uncertainty in $u(y)$ $$\sigma_i = \sqrt{u(y_i)^2+\left[\frac{dy}{dx}u(x_i)\right]^2}$$ and then use $\sigma$ in place of $u(y)$.

If the uncertainty in $x$, $u(x)$, is significant, then the combined uncertainty $\sigma_i$ should be calculated using the last equation. Otherwise, let $\sigma_i = u(y_i)$, and then compute $$\Delta = \sum_{i=1}^n\frac{1}{\sigma_i^2}\sum_{i=1}^n\frac{x_i^2}{\sigma_i^2} - \left(\sum_{i=1}^n \frac{x_i}{\sigma_i^2}\right)^2$$ Next, compute $m$ and $b$ as \begin{align*}
    m &= \frac{1}{\Delta}\left(\sum_{i=1}^n\frac{1}{\sigma_i^2}\sum_{i=1}^n\frac{x_iy_i}{\sigma_i^2} - \sum_{i=1}^n \frac{x_i}{\sigma_i^2}\sum_{i=1}^n\frac{y_i}{\sigma_i^2}\right)^2 \\
    b &= \frac{1}{\Delta}\left(\sum_{i=1}^n\frac{x_i^2}{\sigma_i^2}\sum_{i=1}^n\frac{y_i}{\sigma_i^2} - \sum_{i=1}^n \frac{x_i}{\sigma_i^2}\sum_{i=1}^n\frac{x_iy_i}{\sigma_i^2}\right)^2
\end{align*}
and the uncertainties are \begin{align*}
    u(m) &= \sqrt{\frac{1}{\Delta}\sum_{i=1}^n\frac{1}{\sigma_i^2}} \\
    u(b) &= \sqrt{\frac{1}{\Delta}\sum_{i=1}^n\frac{x_i^2}{\sigma_i^2}}
\end{align*}

For lines fitted in this manner, data can be expected to lie within the range $$y = mx+b \pm k\sqrt{[u(b)]^2+[(x-\overline{x}_i)u(m)]^2}$$









%%%%%%%%%%%%%%%%%%%%%%%%%%%%%%%%%%%%%%%%%%
\chapter{Advanced Uncertainty Analysis Using GUM}


\section{Introduction}

The branch of science concerned with maintaining and increasing the accuracy of measurement is known as \textbf{metrology}. 


\subsection{The laser}

The laser (`light amplification by stimulated emission of radiation'), invented and developed in the early 1960s, offers very high accuracy in length measurement in research and industry.


\section{Measurement Fundamentals}

When experimental values with varying numbers of significant figures are brought together there are several rules that allow us to quote answers to a defensible number of significant figures.

\begin{itemize}
    \item[1.] In the absence of any explicit statement about the uncertainty of a quoted value, the approximate uncertainty in a value can be estimated as half the possible range of values with an extra decimal place that are all consistent, after rounding, with the quoted value.
    \item[2.] When values are multiplied or divided, quote the answer to the number of significant digits that implies a proportional uncertainty closest to the greater of the component proportional uncertainties
    \item[3.] When numbers are added or subtracted, quote the answer to the number of significant digits that implies a proportional uncertainty closest to the greater of the component proportional uncertainties.
    \item[4.] When a quantity with proportional uncertainty $p$ is raised to the power $n$, the resultant proportional uncertainty is $|np|$ and the quoted number of significant figures should reflect this.
\end{itemize}



\subsection{Proportional uncertainty}

To obtain a proportional uncertainty for a measurement we can divide the uncertainty by the measured value. This can be expressed in terms of parts per million or even parts per billion.


\section{Terms used in measurement}

\subsection{Measurement and related terms}

\textbf{Measurement} is a process by which a value of a particular quantity such as the temperature of a water bath or the pH of a solution is obtained. A particular quantity determined through measurement is called a \textbf{measurand}. 

\subsection{Calibration}

\textbf{Calibration} is essential for establishing the traceability of the instrument or artefact to a primary standard. During calibration, a value measured by an instrument or provided by an artefact is compared with that obtained from a standard instrument or artefact. If there is a discrepancy between the value as indicated by the instrument or artefact and the corresponding standard, then the difference between the two is quoted as a correction to the instrument or artefact. The correction \textbf{always} has a stated associated uncertainty.

The result of a measurement is said to be \textbf{traceable} if, through an unbroken chain of comparisons often involving working and secondary standards, the result can be compared with a primary standard. A requirement of traceability is that the chain of comparisons be documented.


\subsection{Value}

The process of measurement yields a \textbf{value} of a particular quantity. In many cases we assume that there is an `actual' or \textbf{true} value of a quantity, such as the value of the charge of an electron. It is the true value that we would like to establish through measurement.

Though we are unable to find the true value of a quantity through measurement, we are able to obtain an \textbf{estimate} of the true value. When only random sources act to influence the values obtained, the \textbf{best estimate} of the true value is taken to be the mean of $n$ values:
%%
\begin{equation}
    \overline{x} = \frac{1}{n}\sum_{i=1}^nx_i
\end{equation}
%%


\subsection{error}

\textbf{Error} is defined as the difference between the measured value and the true value. Since the true value of the quantity cannot be know neither can a measurement's error. It is recognised that sources of error fall into two categories, depending upon how they affect measurement.


Errors causing values to be randomly distributed above and below the true value are termed \textbf{random erros}. Errors with a consistent fault, such as being consistently above or below, is referred to as \textbf{systematic}, since it causes all measured values to be consistently under- or over-estimated.


In situations where we believe that the measured value is close to the true value, we say that the measured value is \textbf{accurate}. Note it is impossible to truly quantify accuracy. When values obtained by repeated measurements of a particular quantity exhibit little variability, we say that those values are \textbf{precise}. Precision, like accuracy, is a \textbf{qualitative} term. Care must be exercised when measurements are precise since, if a systematic error has not been accounted for, all the values could be misleading.


\subsection{Uncertainty}

Errors are key and unavoidable ingredients of the measurement process. Their net effect is to create an \textbf{uncertainty} in the value of the measurand. Uncertainty has a number and (most often) a unit associated with it.


\subsection{Repeatability and Reproducibility}

Measurements which are made under (as far as is possible) identical conditions are spoken of as being \textbf{repeatable}, resulting in the possibility of the values obtained exhibiting little variation or scatter.


If a measurand is well defined there is an expectation that, wherever a measurement is made and whatever techniques are used, the same value should be obtained for the measurand irrespective of who makes the measurement and which instrument is used. If there is consistency between values obtained by different experimenters, we say that the value is \textbf{reproducible}.



\section{An introduction to uncertainty in measurement}

\subsection{Measurement and error}

Errors that fluctuate because of the variability in our measurements even under what we consider to be the same conditions are called \textbf{random} errors. Random errors arise because of our lack of total control over the environment or measurand.


In brief, the environment has a basic randomness or `noise' that we are unable to eliminate completely. In the presence of only random errors, a sequence of reasonably stable measurements allows for the possibility of obtaining a more accurate value through the mean, where the random errors will tend to cancel out in the limit of large sampling.


However, during any measurement there will probably be an error that remains constant when the measurement is repeated under the same conditions. Such \textbf{systematic} errors cannot be reduced by repeating the measurements and taking their mean. An instrument may have a systematic error other than simply an offset. For example there is a possibility of multiplicative systematic errors, referred to as \textbf{gain errors}.


Systematic errors may be revealed by manufacturer or supplier specifications for a device, or look-up tables of physical constants of materials, and previously reported measurements against higher-accuracy devices. Any discrepancy between this specific information and the result of the present measurement suggests that there is a systematic error in the present measurement.


Random errors can be revealed when we repeat the measurement while trying to keep the conditions constant. Systematic errors can be revealed when we vary the conditions, whether deliberately or unintentionally. In general, the bigger the change, the greater the chance of uncovering systematic errors. 


After the existence and cause of a systematic error have been established, an experimental routine can often be developed that automatically takes it into account and eliminates it from the final result. Since the magnitude of the systematic error cannot be known exactly, this process of elimination must itself leave an uncertainty. We can regard errors associated with the correction as random errors scattered around the correction.


\subsection{Dispersion of Data}

The dispersion of data is characterized numerically by a standard deviation. The uncertainty is usually assigned to an integer multiple of the standard deviation. 

There are different types of uncertainty. The scatter around the mean of repeated measurements contributes a \textbf{Type A} uncertainty to the uncertainty of the mean. A \textbf{Type B} uncertainty may be determined by looking up specific information about a measurand such as that found in a calibration report or data book. The calibrated value can tell us how much systematic error would exist if we ignored the calibration report and obtaining this information is the primary purpose of calibrating a device. The uncertainty of the calibrated value is always Type B. The reason is that no statistical analysis can or needs to be done when reading the report. The values summarised in the report were presumably obtained from repeated measurements with an associated Type A uncertainty. This uncertainty will therefore have a Type A component.

The act of writing the report `fossilises' a Type A uncertainty into a Type B uncertainty. 


If there are $n$ values of a quantity, $x_1,...,x_n$, the approximate standard deviation, $s$, of these $n$ values is given by
%%
\begin{equation}
    s = \sqrt{\frac{\sum_{i=1}^n(x_i-\overline{x})^2}{n-1}}
\end{equation}
%%
where $\overline{x}$ is the mean of the $n$ measurements. The square, $s^2$, is known as the \textbf{unbiased estimate of the variance} of the entire population of the $x$'s of which our $n$ values, $x_i$, form a sample. In metrology the standard deviation is referred to as the \textbf{standard uncertainty}. 

The standard uncertainty, $u(\overline{x})$, of the mean may be expected to be less than $s$. This is correct if the values $x_i$ are \textbf{uncorrelated}. If they are uncorrelated, then
%%
\begin{equation}
    u(\overline{x}) = s/\sqrt{n}
\end{equation}
%%
If there is, for example, a steady drift in the values over time, then this high correlation implies the experimental standard deviation of the mean will not be significantly less than $s$, and in fact is closely equal to it.


\subsection{Uncertainty in the estimate of uncertainty}

If the standard uncertainty is denoted by $s$, then its own uncertainty is
%%
\begin{equation}
    u(s) \sim \frac{s}{\sqrt{2\nu}}
\end{equation}
%%
where $\nu$ is the number of `degrees of freedom'. $\nu$ is equal to the number of values, $n$, minus the number of quantities determined using the values. In the case where the mean is the only quantity determined using the values, $\nu = n-1$. Expressed as a percentage uncertainty
%%
\begin{equation}
    \frac{u(s)}{s}\times 100\%\sim \frac{1}{\sqrt{2\nu}}\times 100\%
\end{equation}
%%
This equation is particularly useful for the Type B category of uncertainty. Often the value of $\nu$ for a Type $B$ uncertainty will not exceed $10$, implying a reliability in the estimated uncertainty of no better than about $20\%$. 


\subsection{Combining standard uncertainties}

If $y  = f(x_1,...,x_n)$ is the measurand with input quantities $x_i$, the standard uncertainty, $u(y)$, in $y$ resulting from the standard uncertainties $u(x_i)$ is obtained by addition in quadrature:
%%
\begin{equation}
    u(y) = \sqrt{\sum_{i=1}^n\left(\frac{\partial y}{\partial x_i}\right)^2u(x_i)^2}
\end{equation}
%%
This equation is \textbf{only valid} if the $x_i$ are mutually uncorrelated. A correlation exists if, for example, two or more of the $x_i$ have been measured using the same instrument that has a systematic error with a significant associated uncertainty.


No distinction between Type A and B is made when evaluating the standard uncertainty of the measurand, $y$.




\section{Statistical Concepts}


The \textbf{population} refers to the number of possible measured values. We can usually only ever draw a sample, hoping and expecting that the sample is representative of the population. The quantities of interest obtained from the sample, \textbf{sample statistics}, should reliably represent corresponding parameters in the population, \textbf{population parameters}. 



Suppose we have a random variable described by a measurable function $X:(\Omega,\mathcal{F},\mathbb{P})\rightarrow E$ between measurable spaces, where $\mathbb{P}$ is a probability and $\mathcal{F}$ is a $\sigma$-algebra on $\Omega$. If $E = \R$ with the borel or Lebesgue measure, then the expected value of $X$ is given via integration with respect to its probability measure:
%%
\begin{equation}
    E[X] = \int_\Omega X\,d\mathbb{P}
\end{equation}
%%
In particular, this function is linear. Given a function of random variables, $f:\mathcal{M}(\Omega)\rightarrow \mathcal{M}(\Omega)$, we may define its moment $E[f(X)]$. A particular moment of interest is given by the population variance, $X\mapsto (X-E[X])^2$. The sample variance, given by the mean of this function over a sample data set, is a biased estimate of the true variance $\sigma^2$. In order to obtain an unbiased estimate we must replace the devisor by $n-1$ in the computation of the sample mean.


The standard deviation is given by the square root of the variance. The sample standard deviation, as previously defined, is now only an approximately unbiased estimate of hte population standard deviation. Note that the variance and standard deviation, whether of a sample or of a population, are \textbf{location-independent}.


\subsection{Residuals and degrees of freedom}

Suppose we calculate the mean, $\overline{X}$ of $n$ values $X_i$. Using the mean, we calculate the $n$ resulting residuals $r_i$, where 
%%
\begin{equation}
    r_i := X_i-\overline{X}
\end{equation}
%%
In general, for a sample of size $n$ the sum of the residuals is $0$. This implies that the residuals are not independent as they satisfy a joint constraint. If $n-1$ values are known, the $n$th is immediate. This is why we say that the $n$ residuals have degrees of freedom 
%%
\begin{equation}
    \nu = n-1
\end{equation}
%%


\subsection{Least-squares modeling}

In least squares we consider a modeling function $f(x,\beta)$ where $\beta$ consists of adjustable parameters and $x$ consists of independent variables in a measurement setting. If $y$ are the dependent or measured quantities corresponding to the $x$, we define `residuals' $\epsilon_i := y_i - f(x_i,\beta)$, and define the quantity
%%
\begin{equation}
    S := \sum_{i=1}^n\epsilon_i^2
\end{equation}
%%
We then adjust the $\beta$ parameters in order to minimize $S$, which can be approached using gradients and other methods. We can, for example, use this to estimate the mean when $f(x,\beta)$ is a constant function. 


Whenever a straight line is fitted by least-squares, the residuals have two degrees of freedom fewer than the number of original values, so now we have $\nu = n-2$.


In general, for a linear fit, $y=a+bx$, the intercept and slope have the following forms from least-square fitting:
%%
\begin{equation}
    a = \frac{\left(\sum_{i=1}^ny_i\right)\left(\sum_{i=1}^nx_i^2\right)-\left(\sum_{i=1}^nx_iy_i\right)\left(\sum_{i=1}^nx_i\right)}{n\sum_{i=1}^nx_i^2-\left(\sum_{i=1}^nx_i\right)^2}
\end{equation}
%%
and
%%
\begin{equation}
    b = \frac{n\sum_{i=1}^nx_iy_i-\left(\sum_{i=1}^nx_i\right)\left(\sum_{i=1}^ny_i\right)}{n\sum_{i=1}^nx_i^2-\left(\sum_{i=1}^nx_i\right)^2}
\end{equation}
%%
while the residuals are computed as $\epsilon_i = y_i-a-bx_i$, and their root-mean-square value as
%%
\begin{equation}
    s = \sqrt{\frac{\sum_{i=1}^n\epsilon_i^2}{n-2}}
\end{equation}
%%


\subsection{Standard uncertainties of estimates}

The standard uncertainty, $s_{\overline{X}}$, of the mean for $n$ mutually uncorrelated values is given by $s/\sqrt{n}$, where $s$ is the rms value of the residuals.

When a straight line is fitted to data the standard uncertainties of the intercept $a$ and slope $b$ are
%%
\begin{equation}
    s_a = s\sqrt{\frac{\sum_{i=1}^nx_i^2}{D}}
\end{equation}
%%
and
%%
\begin{equation}
    s_b = s\sqrt{\frac{n}{D}}
\end{equation}
%%
where $D = n\sum_{i=1}^nx_i^2-\left(\sum_{i=1}^nx_i\right)^2$.


In general, the difference between our sample size, $n$, and the number of parameters, $q$, in our least-squares fit is the degrees of freedom, $\nu = n-q$. The smaller the number of degrees of freedom, the less reliable our least-squares fit. We need, in other words, more `redundancy'.

\subsection{Covariance and correlation}


If in a linear fit, $y = a+bx$, the value of $b$ is considerably greater in absolute magnitude than its own standard uncertainty, then $x$ and $y$ have a significant mutual correlation. The mutual covariance of two random variables $X$ and $Y$ defined on the same parameter space is
%%
\begin{equation}
    E[(X-E[X])(Y-E[Y])]
\end{equation}
%%
which for a prescribed sample is
%%
\begin{equation}
    \text{cov}(X,Y) = \frac{\sum_{i=1}^n(X_i-\overline{X})(Y_i-\overline{Y})}{n-1}
\end{equation}
%%
Then the \textbf{linear correlation coefficient} can be expressed as
%%
\begin{equation}
    r := \frac{\text{cov}(X,Y)}{\sqrt{\text{var}(X)\text{var}(Y)}}
\end{equation}
%%
$r \in [-1,1]$, where $-1$ is perfect negatie correlation, and $+1$ is perfect positive correlation. $r$, excepts for its sign, is independent of the slope, and instead depends on the scatter of the data.


In general, independence implies zero correlation, but zero correlation need not imply independence.


\section{Systematic error}


%%%%%%%%%%%%%%%%%%%%%%%%%%%%%%%%%%%%% Part 0.5
\part{Scientific Communication}

%%%%%%%%%%%%%%%%%%%%% Chapter 2.1
\chapter{General Advice}



\section{Writing the Laboratory Report}

\subsection{Where to Begin}

You must consider four things:
\begin{itemize}
    \item[1)] Audience 
    \item[2)] Format
    \item[3)] Mechanics
    \item[4)] Politics
\end{itemize}

Keep in mind that the \textbf{goal of scientific writing is either to inform or persuade}.


\textbf{Audience:} Your document must make a connection with your audience. It is important to decide who will read your report.


\textbf{Goal=Inform:} Efficiency is the style that communicates the most amount of information in the least amount of reading time. Avoid writing a murder mystery where you save the ``best for last." Place important details where they stand out.


\textbf{Goal = Persuade:} Efficiency presents logical arguments in the most convincing manner. You may withhold your conclusions until later in the text. This allows you time to establish credibility with your audience.

Consider the following:
\begin{itemize}
    \item Who will read the document?
    \item What do they know about the subject?
    \item Why will they read the document? What information is sought?
    \item How will they read the document? Will they read from start to finish?
\end{itemize}


\textbf{Format:} While you cannot control the format, you can control the style; including word choices, complexity of illustrations, etc.


\textbf{Mechanics:} This entails the rules of grammar and punctuation.


\textbf{Politics:} In an ideal world you ``stay honest." However, political constraints will force you to face your ethical responsibilities as a scientist and act accordingly.


\subsection{Stylistic Tools}

The first tool you should consider is the \textbf{structure} of the document. This includes the organization of details, depth of the presentation, transitions between ideas, and emphasis. If the structure fails, your readers are lost.


\textbf{Language} is another important stylistic tool. The thoughtful selection of words and the arrangement of those words in sentences can make the difference between a useful paper and a document that misses the mark completely. Consider the following:
\begin{itemize}
    \item \textbf{Precision:} Write what you mean
    \item \textbf{Clarity:} avoid saying what you don't mean
    \item \textbf{Forthright:} be sincere and straightforward
    \item \textbf{Use familiar language}
    \item \textbf{Make every word count}
    \item \textbf{Fluid writing:} smooth transition between different sections and ideas.
\end{itemize}

The integration of \textbf{illustrations} can also help to convey your topic effectively.


\textbf{Structure:} Basically, a scientific paper has a beginning, middle, and end.

The \textbf{beginning} of the documet has one task: to prepare the reader for understanding the document's middle. The single most important component of the beginning is the \textbf{title}. The title identifies the field of study and goes on to separate the document from all others in the field. Your title should say what the document is about and should entice your audience to actually read through the paper. Avoid phrases that your audience might not recognize.

You must also provide a \textbf{summary} or \textbf{abstract} for your report. \textbf{This component should give away the show from the beginning}. The summary is where you emphasize the most important details. Your summary should describe the goals of the work, give a descriptive summary of the results, and how these results were achieved. The first sentence of the summary orients the audience toward the identity of the work.

In writing your \textbf{introduction} you could consider:
\begin{itemize}
    \item \textbf{What} exactly is the work? (Goal or Objective)
    \item \textbf{Why} is the work important? (Motivation)
    \item \textbf{What} is needed to understand the work?
    \item \textbf{How} will the work be presented?
\end{itemize}

Note that the introduction may not have to answer all of these questions. However, it should specify the scope (aspects of the project) and limitations (assumptions that restrict the boundaries of the work). You should demonstrate why your work is important and provide some incentive for the reader to read your work or convince funding agencies to give you money. You should try to instill this importance/curiosity in your readers. 

The amount of background material you provide depends on your audience. Providing historical background should be done only \textbf{if it serves your readers}. In many documents, other kinds of information, such as definitions of key terms, are more important. Some background information may be more appropriate elsewhere in the document. The introduction is the background material that applies to the entire document.

It is good practice to map the document in the introduction, providing the reader with an idea of how the ideas and topics in the document will flow.


The \textbf{middle} is where the bulk of information is found and the outcomes are presented and interpretations are developed. You may take a \textbf{chronological approach}. You may choose to use a \textbf{spatial} approach, such that you develop the ideas according to the physical shape or form of the object. A common approach in Science is to follow the \textbf{flow of a variable}. 

You should envision a path before you start writing and read your document to see if it works for the subject matter and audience. 

Section titles should be clear and you should avoid one-word headings. Consider developing titles that parallel each other.


At the \textbf{end} of the document you provide closure, an analysis of the most important results, and future perspectives on your work. In the end of the document, your should treat the results as a whole. Do not introduce new material at this stage. Leave your readers with a sense of the impact of your work on the proverbial ``Big Picture."


\subsection{Format}

Although the exact style and names might be different, you should find that the contents of the sections are equivalent. 

\begin{itemize}
    \item[1.] \textbf{Title:} The title should be short and straightforward to appeal to a general reader but detailed enough to properly reflect the contents of the article. Think about keywords and use recognizable, searchable terms. Avoid the use of non-standard abbreviations and symbols.
    \item[2.] \textbf{Authorship:} Full names and affilitations for all the authors should be included. Everyone who made a significant contribution to the conception, design or implementation of the work should be listed as co-authors. THe corresponding author has the responsibilsity to include all (and only) co-authors. The corresponding author also signs a copyright licence on behalf of all the authors. If there are more than 10 co-authors on the manuscript, the corresponding author should provide a statement to specify the contribution of each co-author. Identify co-corresponding authors on your manuscript's first page and also mention this in your comments to the editor and/or cover letter.
    \item[3.] \textbf{Abstract:} The abstract should be a single paragraph (50-250 words) that summarises the content of the article. It should set out briefly and clearly the main objectives and results of the work; it should give the reader a clear idea of what has been achieved. Like your title, make sure you use recognisable searchable terms and keywords.
    \item[4.] \textbf{Keywords:} Think carefully of those terms that will help identify what you did and where the work is relevant. Avoid abbreviations and plural forms if possible. Be specific and avoid overly general terms and concepts.
    \item[5.] \textbf{Introduction:} An introduction should `set the scene' of the work. Context is \textbf{EVERYTHING} and your introductino should clearly explain both the nature of the problem under investigation and its background. What are you doing and why is it significant? How is this work related to our current knowledge and what new knowledge will your work generate? What might be the broader implications being careful not to overstate this? It should start off general and then focus in to the specific research question you are investigating. Describe if there are any challenges or controversies involved in the topic. Ensure you include all relevant references. Define any important terms or concepts that might not be familiar to your readers. THe concluding sentence or two should be a concise statement of what the paper will accomplish. What \textbf{EXACTLY} is being tested or investigated?
    \item[6.] \textbf{Experimental:} You should provide descriptions of the experiments in enough detail so that a skilled researcher is able to repeat them. Standard techniques and methods used throughout the work should just be stated at the beginning of the section; descriptions of these are not needed. Any unusual hazards about the procedures or equipment should be clearly identifies.
    \item[7.] \textbf{Results \& Discussion:} Your results should be organized into an orderly and logical sequence. Only the most relevant results should be described in the text; to highlight the most important points. Figures, tables, and equations should be used for purposes of clarity and brevity. Data should not be reproduced in more than one form, for example in both figures and tables, without good reason. 

        The purpose of the discussion is to explain the meaning of your results and why they are important. You should state the impact of your results compared with recent work and relate it back to the problem or question you posed in your introduction. Ensure claims are backed up by evidence and explain any complex arguments.
    \item[8.] \textbf{Conclusions:} State the most important outcome of your work clearly and concisely. Place the outcome in some context; the ``big picture." Relate your outcome to the goal and motivation described in the introduction. Discuss where there are opportunities for future work. Do not repeat ideas from the introduction. A conclusion can be short.
    \item[9.] \textbf{Conflicts of interest:} Ensure a conflicts of interest statement is included in your manuscript here. If no conflicts exist, please state that `There are no conflicts to declare'. 
    \item[10.] \textbf{Acknowledgements:} Contributors (that are not included as co-authors) may be acknowledged; they should be as brief as possible. All sources of funding should be declared.
    \item[11.] \textbf{References:} THe list of papers, book chapters, theses, reports, webpages, etc. is provided at the end of your paper. In this course, references are cited in the order in which they appear in the paper. For a webpage citation use the following format:
        \begin{quotation}
            Last, FM. (Year, Month, Date Published). Article title [Type of blog post]. Retrieved (on Year, Month, Date) from URL.
        \end{quotation}
\end{itemize}

\subsection{Paper elements}

\begin{itemize}
    \item[1.] \textbf{Footnotes:} Footnotes relating to the title and/or authors, including affilitations, should appear at the very bottom of the first page of the article.
    \item[2.] \textbf{Data tables:} You will usually not include ALL your data in the final report. Select the data which are most relevant. You need to include sufficient evidence to demonstrate the methodology that was used is robust and appropriate. You want to show that you have followed sound metrological principles, for example calibration and assessment of uncertainties. You will also want to arrange the results in a manner that conveys the outcomes clearly and compactly. Some results may be better suited to presentation as a figure rather than a table, and you need to decide what is more effective while remaining transparent and honest.
        The table will contain columns of information and a useful identifier should be centered above each column. Make sure to include units if necessary. Uncertainty should be incorporated into th etable entries. THis can be done for each individual entry, as a separate column, or, if the uncertainties are the same for all entries, at the top or bottom of the column. A table may have an associated caption, although this is not always done. There must be a reference to the table in the main body of the text. The font size for the table may be smaller than the main body in order to accommodate all the entries. It is not good practice to split data tables, and it may be better to change the orientation of the table from portrait to landscape to avoid splitting the table.
    \item[3.] \textbf{Figures and images:} It is worth spending some effort to create meaningful images. Figures are always accompanied by a figure caption that is attached to the figure itself. Figures never have titles. The caption should be a summary of what is being shown/plotted and anything of significance. The full explanation of the figure is in the body of the paper. If you are providing a plot or graph, make sure to include axis titles with units. Use a font size that is easy to read, which might be a few points large than you use in the document. 

        Uncertainties should be provided on the plots when appropriate. Place major and minor tick marks on the plot. Use a range for the axes that uses the full width and height of the plot so that you avoid squishing the data into a very small area of the figure.
    \item[4.] \textbf{Equations:} The equations should be labeled and in a separate line of the text. The equation itself is right justified and the equation number is left justified. 
\end{itemize}

\subsection{Other considerations}


\textbf{Inclusive language:} Inclusive language acknowledges diversity, conveys respect to all people, is sensitive to differences, and promotes equal opportunities. Authors should ensure that writing is free from bias.




\section{How To Speak}


\section{Writing A Scientific Paper}

\subsection{10 Steps}

\textbf{Step 1: Write a Vision Statement} - What is the key message of your paper? Be able to articulate it in one sentence, because it's a sentence you'll come back to a few times throughout the paper. Think of your paper as a press release: what would the subhead be? 

The vision statement should guide your next important decision: where are you submitting? Once you choose a journal, check the website for requirements with regards to formatting, length limits, and figures.


\textbf{Step 2: Don't Start at the Beginning} - If you start with the introduction, by the time everything else is written, you will likely have to rewrite both sections.


\textbf{Step 3: Storyboard the Figures} - Figures are the best place to start, because they form the backbone of your paper. The first figure should inspire the reader to want to learn about your discovery.

Consider ``storyboarding" where all figures are laid out on boards. One approach is to put the vision statement on the first slide of a powerpoint, and all your results on subsequent slides. To start, simply include all data, without concern for order or importance. Subsequent passes can evaluate consolidation of data sets and relative importance. The figures should be arranged in a logical order to support your hypothesis statement. Notably, this order may or may not be the order in which you took the data.


\textbf{Step 4: Write the Methods Section} - The methods section is the most important to write accurately. ANy results in your paper should be replicable based on the methods section. If you've developed an entirely new experimental method, write it out in excruciating detail, including setup, controls, and protocols, also manufacturers and part numbers, if appropriate. If you're building on a previous study, there's no need to repeat all of those details; that's what references are for.

The methods section is simply a record of what you did (no results!). 


\textbf{Step 5: Write the Results and Discussion Section} - This section(s) should form the bulk of your paper-by storyboarding your figures, you already have an outline! 

A good place to start is to write a few paragraphs about each figure, explaining: 1. the result (this should be void of interpretation), 2. the relevance of the result to your hypothesis statement (interpretation is beginning to appear), and 3. the relevance to the field (this is completely your opinion). Whenever possible, you should be quantitative and specific, especially when comparing to prior work. Additionally, any experimental errors should be calculated and error bars should be included on experimental results along with replicate analysis.

You can use this section to help readers understand how your research fits in the context of other ongoing work and explain how your study adds to the body of knowledge. This section should smoothly transition into the conclusion.


\textbf{Step 6: Write the conclusion} - Summarize everything you have already written. Emphasize the most important findings from your study and restate why they matter. State what you learned and end with the moost important thing you want the reader to take away from the paper-again, your vision statement. 


\textbf{Step 7: Now Write the Introduction} - The introduction sets the stage of your article. The introduction gives a view of your researh from 30,000 feet: it defines the problem in the context of a larger field; it reviews what other research groups have done to move forward on the problem (the literature review); and it lays out your hypothesis, which may include your expectations about what the study will contribute to the body of knowledge. THe majority of your references will be located in the introduction.


\textbf{Step 8: Assemble References} - References serve multiple roles in a manuscript:
%%
\begin{itemize}
    \item[1)] To enable a reader to get more detailed information on a topic that has been previously published.
    \item[2)] To support statements that are not common knowledge or may be contentious.
    \item[3)] To recognize others working in the field, such as those who came before you and laid the groundwork for your work as well as more recent discoveries. The selection of these papers is where you need to be particularly conscientious. You need to make sure that your references include both foundational papers as well as recent works.
\end{itemize}


\textbf{Step 9: Write the Abstract} - The abstract is the elevator pitch for your article. Most abstracts are 150-300 words, or approximately 10-20 sentences. It should describe the importance of the field, the challenge that your research addresses, how your research solves the challenge, and its potential future impact. It should include any key quantitative metrics.


\textbf{Step 10: The Title Comes Last} - The title should capture the essence of the paper.  If someone was interested in your topic, what phrase or keywords would they type into a search engine?


\subsection{First-Class Paper Writing}

\textbf{Keep your message clear}. This is even more important when there is a multidisciplinary group of authors. Authors should sit together (preferably in person) and seek consensus --- not only in the main message, but also in the selection of data, the visual presentation and information necessary to transmit a strong message.


Writers should put their results into a global context to demonstrate what makes those results significant or original. 

There is a narrow line between speculation and evidence-based conclusions. Speculation is okay in the discussion, but only to a degree! In the conclusion, include a one-or-two sentence statement on the research you plan to do in the future and on what else needs to be explored.


\textbf{Create a logical framework}. In each paragraph the first sentence defines the context, the body contains the new idea, and the final sentence offers a conclusion. For the whole paper, the introduction sets the context, the results present the content, and the discussion brings home the conclusion.

It is crucial to focus your paper on a single key message, which you communicate in the title. Everything in the paper should logically and structurally support that idea.


\textbf{State your case with confidence}. Answering one central question --- What did you do? --- is the key to finding the structure of a piece. Every section of the manuscript needs to support that one fundamental idea.


\textbf{Beware the curse of `zombie nouns'}. Scientific writing should be factual, concise, and evidence-based, but it should also be creative, original, and engaging.

We should engage readers' emotions and avoid formal, impersonal language. Still, there's a balance. Once the paper has a clear message, try some vivid language to help to tell the story.


\textbf{Prune that purple prose}. Make the writing only as complex as it needs to be. 

They need to explain why the findings are interesting and how they affect a wider understanding of the topic. Authors should also reassess the existing literature and consider whether their findings open the door for future work. And, in making clear how robust their findings are, they must convince readers that they’ve considered alternative explanations. 


\textbf{Aim for a wide audience}. Articles with clear, succinct, declarative titles are more likely to get picked up by social media or the popular press. Make your point clearly and concisely---if possible in non-specialist language, so that readers from other fields can quickly make sense of it.


%%%%%%%%%%%%%%%%%%%%%%%%%%%%%%%%%%%%% Part 1
\part{Phys 497 Laboratories}

\input{Phys497Labs}

%%%%%%%%%%%%%%%%%%%%%%%%%%%%%%%%%%%%% Part 2
\part{Circuit Theory}


%%%%%%%%%%%%%%%%%%%%%% Chapter 2.1
\chapter{Introductory Examples}


\section{Direct Current and Circuit Laws}

Current is the flow of chare. We will look at it as conventional current (positive flow). Voltage and current are constant for \Emph{direct current} (DC). Two various types of two-terminal circuits elements may be classified according to their terminal voltage-current $v(i)$ relationships: \begin{itemize}
    \item $v(i) = $ constant
    \item $v = v(t), i = i(t)$ - ideal voltage source (independent voltage source)
\end{itemize}

\begin{defn}
    The waveform $v(t)$ represents the voltage produced by the source. ``Ideal source" means the voltage is maintained regardless of the drawn current.
\end{defn}

\begin{defn}
    An \Emph{ideal battery} has its voltage drop between the terminals being $V_0$.
\end{defn}

\begin{figure}[H]
    \centering
    \includegraphics[scale = 0.6]{Images/TH1.PNG}
\end{figure}

\begin{defn}[Ohm's Law]
    \Emph{Ohm's Law} states that for a resistor connected between terminals $A$ and $B$, the voltage drop from $A$ to $B$ is equal to the resistance multiplied by the current flowing from $A$ to $B$ through the resistor. \begin{equation*}
        I = \frac{V}{R}\implies V = IR
    \end{equation*}
    and \begin{equation*}
        v(t) = Ri(t)\implies i(t) = \frac{1}{R}v(t)
    \end{equation*}
    for \Emph{ohmic materials} (such as wires)
\end{defn}

\subsection{Resistance}

Resistance depends on the geometry and the material of the conductor. For a specific temperature, $R \propto \frac{L}{A}$, specifically \begin{equation*}
    R = \rho\frac{L}{A}
\end{equation*}
where $\rho$ is the \Emph{resistivity} ($\Omega m$).

\begin{table}[H]
    \centering
    \caption{Material Resistivity}
    \begin{tabular}{|cc|}
        \hline 
        Material & Resistivity $\Omega m$ \\ \hline
        Silver & $1.59\times 10^{-8}$ \\ \hline
        Copper & $ 1.68\times 10^{-8}$ \\ \hline 
        Aluminum & $2.65\times 10^{-8}$ \\ \hline
        Tungsten & $5.6\times 10^{-8}$ \\ \hline
        Graphite & $(3$-$60)\times 10^{-5}$ \\\hline
        Silicon & $0.1-60$ \\ \hline
        Hard Rubber & $(1$-$100)\times 10^{13}$ \\ \hline
    \end{tabular}
\end{table}

\begin{defn}
    The power absorbed by a resistor can be calculated according to \begin{equation*}
        P(t) = v(t)i(t) = i^2(t)R = \frac{v^2(t)}{R}
    \end{equation*}
\end{defn}
For DC current this reduces to $P = IV = I^2R = \frac{V^2}{R}$.

\subsection{Circuit Laws}

\begin{thm}[Kirchhoff's Current Law]
    The algebraic sum of all the currents entering and leaving a node is equal to zero. \begin{equation*}
        \sum_{i=1}^nI_i = 0
    \end{equation*}
    \begin{figure}[H]
        \centering
        \includegraphics[scale = 0.8]{Images/TH2.PNG}
    \end{figure}
    This is really \Emph{conservation of charge}.
\end{thm}

\begin{thm}[Kirchhoff's Voltage Law]
    The algebraic sum of all the voltage drops around any closed path is zero at every distance and time: \begin{equation*}
        \sum_{i=1}^nV_i = 0
    \end{equation*}
    This is a form of \Emph{energy conservation}.
\end{thm}

For a circuit and any pair of nodes $j$ and $k$, the voltage drop $v_{jk}$ from node $j$ to node $k$ is $$v_{jk} = v_j - v_k$$


\begin{thm}[Resistors in Series]
    If we have a line of $n$ resistors in series (i.e. the same current passes through all of them), then the equivalent resistance of the collection is their sum: $$R_{eq}^{series} = \sum_{i=1}^nR_i$$
    \begin{figure}[H]
        \centering
        \includegraphics[scale = 0.8]{Images/TH3.PNG}
    \end{figure}
\end{thm}

Now, consider a circuit that has $n$ resistors in series connected to a battery, $V_{in}$, with current $I$. Then the equivalent resistance across the series of resistors is $R_{eq} = V_{in}/I$, so the voltage drop across any individual resistor $V_i$ is $$V_i = R_iI = R_i \frac{V_{in}}{\sum_{j=1}^nR_j}$$

\begin{defn}
    $n$ resistors are said to be in parallel if the same voltage drop happens across them:
    \begin{figure}[H]
        \centering
        \includegraphics[scale = 0.8]{Images/TH4.PNG}
    \end{figure}
    In this case different currents go through each resistor, say current $I_i$ goes through resistor $R_i$, and we have $$I_T = \frac{V_{in}}{R_{eq}} = \sum_{i=1}^nI_i = \sum_{i=1}^n\frac{\Delta V_i}{R_i}$$ where by Kirchhoff's voltage law $V = \Delta V_i$ for each $I$, so $$\frac{1}{R_{eq}} = \sum_{i=1}^n\frac{1}{R_i}$$
\end{defn}

\subsection{Voltmeters and Ammeters}

\begin{defn}
    A voltmeter measures voltage (potential difference) between two points along the circuit. We connect them in \Emph{parallel} to the device. To connect them we create a node, which means some current will always go through the resistor. The goal is to send as little current as possible through the voltmeter, therefore voltmeters have \Emph{very high internal resistances}.
\end{defn}

\begin{defn}
    Ammeters measure current through the circuit. We connect them in \Emph{series} with other elements. In order not to affect the circuit, we want them to have the smallest internal resistance possible.
\end{defn}

\subsection{Th\'{e}venin Theorems}

\begin{thm}
    Given an arbitrary two-terminal linear network $N$ consisting of resistances and independent sources, then, for almost all such $N$, there exists an equivalent two-terminal network consisting of a resistance $R_{th}$ in series with an independent voltage source $V_{oc}(t)$.
\end{thm}

The voltage $V_{oc}(t)$ (open-circuit voltage) is what appears across the two terminals of $N$ when no other network is attached. Resistance $R_{th}$ (Th\'{e}venin Equivalent Resistance) is the equivalent resistance of $N$ when all independent sources are deactivated.

The procedure for calculations is as follows:

\begin{enumerate}
    \item Remove the load resistor $R_L$ or component concerned. (Create an opening between two terminals)
    \item Remove the active sources: Remove voltage sources by shortening them, and remove current sources by opening them
    \item Find $R_{th}$ using rules for regular equivalent resistances.
    \item Find $V_{oc}$ by the usual circuit analysis methods.
    \item Find the current flowing through the load resistor $R_L$.
\end{enumerate}


\section{RLC Circuits}

\subsection{Phasors}

First recall we can consider an oscillation $s(t) = A\cos(\omega t+\phi_0)$ as the real part of a complex function $$\tilde{s}(t) = Ae^{i(\omega t+\phi_0)}$$ Similarly, $y(t) = A\sin(\omega t+\phi_0)$ can be seen as the imaginary part of the same function. $\tilde{s}(t)$ is our \Emph{phasor}. Note that waves add as phasors, which add vectorally (superposition principle). In general, any oscillation can be broken into a sum of simple harmonic oscillations (i.e. Fourier series). 


\subsection{Capacitors}

Now, the capacitor usually consists of a pair of parallel plates (possibly with a dielectric between them) storing opposite charges on either side, and hence containing a potential difference between the plates.

\begin{defn}
    \Emph{Capacitance} is the ratio of the magnitude of the charge on one of the plates of the capacitor to the absolute value of the potential difference between them is defined as \begin{equation*}
        C = \frac{q}{|\Delta V|},\;\;1\;F = \frac{1\;C}{1\;V}
    \end{equation*}
\end{defn}
For a parallel-plate capacitor of area $A$ and plate separation $d$ we can compute the capacitance by applying Gauss's Law $\oint\vec{E}\cdot d\vec{A} = \frac{q_{in}}{\varepsilon_0}$, by $$\oint\vec{E}\cdot d\vec{A} = EA = \frac{q}{\varepsilon_0} = \frac{\sigma A}{\varepsilon_0}$$ so $E = \frac{\sigma}{\varepsilon_0}$, so $\Delta V = Ed = \frac{\sigma d}{\varepsilon_0} = \frac{qd}{A\varepsilon_0}$, and $C = \frac{A\varepsilon_0}{d}$.

In order to determine the electric potential energy stored in a capacitor, we look at energy required to transfer charge from one conductor to another. Consider moving an infinitesimal amount of charge $dq'$, so the potential difference is essentially constant during the transfer of one increment $dU^E = -dW = Vdq' = \frac{q'}{C}dq'$, so $U^E = \frac{1}{2}\frac{q^2}{C} = \frac{1}{2}C\Delta V^2 = \frac{1}{2}q\Delta V$.

\subsection{Inductor}

A consequence of Faraday's Law, $\oint \vec{E}\cdot d\vec{l} = -\frac{d\Phi_B}{dt}$, where $\Phi_B$ is the magnetic flux, is that when a current through a conducting loop changes it induces an emf (potential difference, $\varepsilon_{ind}$) in the loop itself. \Emph{Inductance} $L$ of the loop (or solenoid) is defined as a proportionality constant between the emf andthe rate of change of current: $$\varepsilon = -L\frac{dI}{dt},\;\;\;1\;H = \frac{1\;Vs}{1\;A}$$
A device that has an appreciable inductance is called an \Emph{inductor}. It describes the change of magnetic flux associated with a change of current in a particular loop/solenoid.

Work must be done on an inductor to establish the current through it because the change in current causes an induced emf that opposes this change. Consider $dW$, work done \emph{on} the inductor when an amount of charge $dq$ is being moved through it, so $dW = -\varepsilon dq$, $\frac{dW}{dt} = -\varepsilon I = LI\frac{dI}{dt}$, and so $W = \int dW = \frac{1}{2}LI^2$ and $U^B = \frac{1}{2}LI^2$.

\subsection{AC Current}

For a conductor, at any instance $C = \frac{q}{v_c(t)}$ so $q = Cv_c(t) = CV_c\sin(\omega t)$. Current is $$i(t) = CV_c\omega \cos(\omega t) = CV_c\omega\sin\left(\omega t + \frac{\phi}{2}\right) = I_c\sin\left(\omega t+\frac{\phi}{2}\right)$$ As $I_C = CV_C\omega$, $V_C = \frac{I_C}{C\omega}$, so $Z_C = \frac{1}{\omega C}$.

First, $v_L(t) = L\frac{di}{dt} = V_L\sin(\omega t)$, so $$i(t) = -\frac{V_L}{\omega L}\cos(\omega t) = \frac{V_L}{\omega L}\sin\left(\omega t -\frac{\pi}{2}\right)$$ Then $$I_L = \frac{V_L}{\omega L}\implies Z_L = \omega L$$

\subsection{Series RLC Circuit}

First, for a circuit consisting of an inductor, resistor, capacitor, and an AC voltage source, we have the relation $\varepsilon = v_R(t) + v_L(t) + v_C(t)$. Current is the same for all three elements, so $$V_0\sin(\omega t) = \frac{1}{C}\int idt + L\frac{di}{dt}+iR$$ and differentiating $$V_0\omega \cos(\omega t) = \frac{i}{C} + L\frac{d^2i}{dt^2} + R\frac{di}{dt}$$ The solution to these DEs is $$i(t) = \frac{V_0\sin(\omega t-\phi)}{\sqrt{R^2 + \left(\omega L - \frac{1}{\omega C}\right)^2}}$$ where $$\phi = \tan^{-1}\left[\frac{\omega L - \frac{1}{\omega C}}{R}\right]$$




%%%%%%%%%%%%%%%%%%%%%% Chapter 2.2
\chapter{AC Sources, Waveforms}

Recall that ``AC" stands for \Emph{alternating current} - current whose value oscillates sinusoidally about a mean value of zero through time. Recall also from Fourier transform theory that any well-behaved signal of any shape can be synthesized as a linear combination of pure AC signals of appropriately chosen frequencies, amplitudes, and phases.

First, in a circuit diagram we notate an AC voltage source by:

\begin{figure}[H]
    \centering
    \includegraphics[scale = 0.8]{Images/ACSymbol.PNG}
    \caption{Circuit symbol for an ideal AC voltage source. $\pm$ signs show the polarity of the output voltage at $t = 0$.}
    \label{fig:ACSymb}
\end{figure}

The defining property of an ideal AC voltage source is that it produces a sinusoidally varying voltage signal of constant amplitude and frequency across its terminals, no matter what device is connected across them $$v_s(t) = v_+ - v_- = \varepsilon_0\cos(\omega t)$$
where $\varepsilon_0$ and $\omega$ are positive real constants. $\varepsilon_0$ is the constant \Emph{voltage amplitude} of the source, measured in volts $(V)$ SI units. $\omega$ is the \Emph{angular frequency} of the source, measured in radians per second. The waveform executes one complete cycle in a time interval equal to the period $T$: $$T = \frac{1}{f} = \frac{2\pi }{\omega}$$ This is illustrated in the Figure:

\begin{figure}[H]
    \centering
    \includegraphics[scale = 0.8]{Images/ACVoltageSource.PNG}
    \caption{Output voltage signal of the ideal AC voltage source of Figure \ref{fig:ACSymb}.}
    \label{fig:ACSource}
\end{figure}

Often we will ground the lower terminal of an ideal voltage source, so that all potentials in the circuit are measured relative to that ground. 

\begin{defn}
    The \Emph{root-mean-square voltage}, or $V_{rms}$, of an AC voltage source is the square root of the time-averaged square of the voltage signal. For the AC voltage source this is just $$V_{rms} = \frac{1}{\sqrt{2}}\varepsilon_0$$
\end{defn}

Sometimes the amplitude will be given in terms of the rms voltage. A third way of specifying the voltage amplitude is the \Emph{peak-to-peak voltage swing}, or maximum voltage minus minimum voltage for one cycle of the voltage waveform. For an AC waveform this is simply twice the amplitude.

\section{Loaded Ideal AC Voltage Generator}

Consider the AC circuit in which an ideal AC voltage source of the type described previously is connected across a circuit element, referred to as the \Emph{load}. The load is a ``black box" with two terminals. If the load contains only some combination of inductance, resistance, and capacitance, then it is said to be a \Emph{passive linear load}.

\begin{prop}
    When considering an AC circuit with a passive linear load as the only component, the laws of physics dictate that the current $i(t)$ in the circuit must be a constant-amplitude sinusoidal function of time with the same angular frequency as the ideal voltage source: $$i(t) = I_0\cos(\omega t+\phi)$$
    $I_0$ is the \Emph{current amplitude}, measured in amperes $(A)$, and $\phi$ is the \Emph{phase angle}.
\end{prop}

When the load is \Emph{pure resistance} $R$, the law of the resistor dictates that $v_s = iR$ (Ohm's Law). Thus, in the resistive case, $\phi = 0$, the current has amplitude $I_0 = \frac{\varepsilon_0}{R}$, and the current is exactly in phase with the source voltage $v_s(t)$. THe presence of capacitors and/or inductors is what gives us a phase shift.

\section{Three Fundamental Circuits}

An important problem in AC circuit theory is finding the current in the circuit when the load network is specified. We shall first investigate the simple case where the load consists of a single fundamental circuit element: a resistor, a capacitor, or an inductor. In the case of a resistor, we have the following law:

\begin{thm}[Law of the Resistor]
    The voltage across a resistor is given by $$v_R(t) = i(t)R$$ Putting $v_R = v_s(T) = \varepsilon_0\cos(\omega t)$, we obtain $$i(t) = \frac{\varepsilon_0}{R}\cos(\omega t)$$ Thus, we have current amplitude $I_0 = \frac{\varepsilon_0}{R}$ and current phase $\phi = 0$.
\end{thm}

For a capacitor we have the following law:

\begin{thm}[Law of the Capacitor]
    The voltage across a capacitor with capacitance $C$ and storing charge $q(t)$ is $$v_C(t) = \frac{q(t)}{C}$$ or $$i(t) = \frac{dq}{dt} = C\frac{dv_C}{dt}$$ With $v_C = v_s(t) = \varepsilon_0\cos(\omega t)$, $i(t)$ is $$i(t) = -\omega C\varepsilon_0\sin(\omega t) = \omega C\varepsilon_0\cos(\omega t+\pi/2)$$ The current amplitude is $I_0 = \omega C\varepsilon_0$, and the current phase is $\phi = \pi/2$.
\end{thm}

Finally, the law for inductors is as follows:

\begin{thm}[Law of the Inductor]
    The voltage across an inductor with inductance $L$ is given by $$v_L(t) = L\frac{di(t)}{dt}$$ or $$i(t) = \frac{1}{L}\int_0^tv_L(t')dt'$$ Putting $v_L = v_s(t) = \varepsilon_0\cos(\omega t)$ and solving for $i(t)$ gives $$i(t) = \frac{\varepsilon_0}{\omega L}\sin(\omega t) = \frac{\varepsilon_0}{\omega L}\cos(\omega t - \pi/2)$$ The current amplitude is $I_0 = \frac{\varepsilon_0}{\omega L}$ and the current phase is $\phi = -\pi/2$.
\end{thm}

We can summarise these results as follows: \begin{itemize}
    \item When an AC voltage signal of amplitude $V_0$ and angular frequency $\omega$ appears across any linear circuit element, the current in the element is guaranteed to be an AC current signal of the same angular frequency $\omega$ as the voltage signal.
    \item When the device is a \Emph{resistor} of resistance $R$, the current is exactly in phase with the voltage across the resistor $(\phi = 0)$, and the current amplitude is given by $I_0 = V_0/R$.
    \item When the device is a \Emph{capacitor} of capacitance $C$, the current is $\pi/2$ radians ahead of the voltage across the capacitor $(\phi = \pi/2)$; the current amplitude is frequency dependent and given by $I_0 = \omega CV_0$
    \item When the device is an \Emph{inductor} of capacitance $C$, the current is $\pi/2$ radians behind the voltage across the inductor $(\phi = -\pi/2)$; the current amplitude is frequency dependent and given by $I_0 = V_0/(\omega L)$
\end{itemize}

Note that in the case of a conductor, \Emph{the source voltage directly determines the charge on the upper plate of the capacitor}, not the current, as was the case with the resistive load. The current is then the resulting time derivative of the capacitor charge $q$. Indeed, the current is the scaled derivative of the source voltage. 


%%%%%%%%%%%%%%%%%%%%%% Chapter 2.3
\chapter{Phasor Analysis}

A \Emph{phasor} is simply a complex number, a constant independent of time, that contains information on both the amplitude and phase of the AC signal. 

\section{Phasors and Complex Sinusoids}

Recall that we can use Euler's relation to express a complex exponential as $$e^{j\theta} = \cos\theta + j\sin\theta$$ where $j = \sqrt{-1}$, and that the modulus of $e^{j\theta}$ is $1$. If we set $j\theta = j\omega t$ we will have made what is called a time-dependent \Emph{complex sinusoid of unit amplitude}, $e^{j\omega t}$.

Recall that for a single loop circuit, one with a linear load, we have the current $$i(t) = I_0\cos(\omega t+\phi)$$ 

\begin{defn}
    We define the time-independent \Emph{complex amplitude} or \Emph{phasor} associated with $i(t)$ as $$\hat{I} = I_0e^{j\phi}$$
    Combining the phasor $\hat{I}$ with the complex sinusoid $e^{j\omega t}$ we can express the current as $$i(t) = \text{Re}\left(\hat{I}e^{j\omega t}\right) = I_0\cos(\omega t+\phi)$$
\end{defn}

Using our phasor $\hat{I}$, we can recover the amplitude and phase of the physical current as follows: $$I_0 = |\hat{I}| = \sqrt{(\text{Re}\hat{I})^2+(\text{Im}\hat{I})^2};\;\;\;\phi = \tan^{-1}\left[\frac{\text{Im}\hat{I}}{\text{Re}\hat{I}}\right]$$

\section{Kirchhoff's Rules}

\begin{thm}[Kirchhoff's Point Rule]
    For any junction in a circuit with $n$ currents going into it, $i_1(t),...,i_n(t)$, where $i_j(t) < 0$ if it is conventional current going out of the junction, and $i_j(t) > 0$ if it is conventional current going into the junction, then $$\sum_{j=1}^ni_j(t) = 0$$ This is a statement of conservation of charge.
\end{thm}

If this junction is part of a circuit network driven by a single AC voltage source of angular frequency $\omega$, each of the currents will itself be a sinusoidal function of time with the same angular freuency $\omega$ as the source. Using complex phasor amplitudes, $\hat{I}_k = I_ke^{j\phi_k}$, we have that $$\text{Re}\left[\sum_{k=1}^n\hat{I}_ke^{j\omega t}\right] = 0$$ First, note that $\sum_{k=1}^n\hat{I}_k = \sum_{k=1}^nI_ke^{j\phi_k} = I_Se^{j\phi_S}$ for some $I_S$ and $\phi_S$ since all complex numbers can be expressed in polar form. Then $\text{Re}\left[I_Se^{j\phi_S}e^{j\omega t}\right]  = 0$. Since this holds for all $t$, we must have that $I_S = 0$ (setting $t = \frac{-\phi_S}{\omega}$), and it follows that $I_Se^{j\phi_S}e^{j\omega t} = 0$. Thus we have the complex equation $$\sum_{k=1}^n\hat{I}_ke^{j\omega t} = \sum_{k=1}^nI_ke^{j\phi_k}e^{j\omega t} = 0$$ Since $e^{j\omega t} \neq 0$, this is equivalent to the following:

\begin{thm}[Kirchhoff's point rule (phasors)]
    Kirchhoff's point rule is equivalent to $$\sum_{k=1}^n\hat{I}_k = \sum_{k=1}^nI_ke^{j\phi_k} = 0$$ where $\hat{I}_k$ is the phasor of the $k$th current going into the junction.
\end{thm}

\noindent Next we have Kirchhoff's loop rule:

\begin{thm}[Kirchhoff's Loop Rule]
    If one encounters $n$ voltage drops/gains, $v_k(t)$, when going around a closed loop with zero resistance wires, Kirchhoff's loop rule state $$\sum_{k=1}^nv_k(t) = 0$$ This is a statement of the law of conservation of energy.
\end{thm}

If the voltage drops are AC signals of phasor amplitudes $\hat{V}_k = V_ke^{j\phi_k}$, the loop rule can be transformed as with the junction rule:

\begin{thm}[Kirchhoff's Loop Rule (phasors)]
    Kirchhoff's loop rule is equivalent to $$\sum_{k=1}^n\hat{V}_k = \sum_{k=1}^nV_ke^{j\phi_k} = 0$$ where $\hat{V}_k$ is the phasor of the $k$th voltage drop/gain, if all voltage drops/gains are AC signals with the same angular frequency $\omega$.
\end{thm}

\section{Phasor Summary}

We summarize the properties of phasors as follows: 

\begin{itemize}
    \item A \Emph{phasor} is a complex constant (no time dependence) associated with a voltage or current waveform in an AC circuit.
    \item Any complex number $z = x+jy$ can be written in polar form $z = |z|e^{j\phi}$ with $\phi$ its phase constant or argument, and $|z|$ its modulus or magnitude.
    \item For an AC current signal $i(t) = I_{max}\cos(\omega t+\phi)$, the associated phasor is $\hat{I} = I_{max}e^{j\phi}$. If we have the phasor $\hat{I}$ we can recover the amplitude and phase of $i(t)$ as $I_{max} = |\hat{I}|$ and $\phi = \text{arg}\hat{I}$.

        Similarly, for any AC voltage signal $v(t) = V_{max}\cos(\omega t+\theta)$, the associated phasor is $\hat{V} = V_{max}e^{j\theta}$. If we have the phase $\hat{V}$ we can recover the amplitude and phase of $v(t)$ as $V_{max} = |\hat{V}|$ and $\theta = \text{arg}\hat{V}$.
    \item When phasors add, their real and imaginary parts add to produce the phasor sum.
\end{itemize}

Recall that complex numbers behave exactly like two-dimensional real vectors under addition. We can use this fact to draw vector addition diagrams, called \Emph{phasor addition diagrams} in our context.


\section{Complex Impedance}

Consider a circuit consisting of an ideal AC voltage source of frequency $f$ connected across a load containing nothing but capacitors, resistors, and inductors, connected to each other in some arbitrary fashion. Such a load is called a \Emph{passive linear load}, since it contains no sources of emf. Recall that in this case $v(t) = \varepsilon_0 \cos\omega t$ and $i(t) = I_{max}\cos(\omega t+\phi)$. The phasors, $\hat{V}$ and $\hat{I}$, representing the load are $\hat{V} = \varepsilon_0$ and $\hat{I} = I_{max}e^{j\phi}$. 

\begin{defn}
    The \Emph{complex impedance} $Z$ of a load is the ratio of its voltage phasor to its current phasor: $$Z = \frac{\hat{V}}{\hat{I}}$$ The unit of $Z$ is the \Emph{ohm} $(1\;\Omega = 1\;V/A)$ 
\end{defn}
The impedance of a passive load is generally complex and frequency-dependent. We can always write the impedance in rectangular form $$Z = R + jX$$ The real part and imaginary part of the impedance $Z$ are both purely real quantities with units of ohms. They are referred to as the \Emph{resistance} ($R$), and \Emph{reactance} $(X)$ of the load. 

Given the source voltage phasor and the impedance $Z$ of the load, we can find the complex current phasor using the form of $Z$. The amplitude and phase relative to the voltage source of the current are then $$I_{max} = |\hat{I}| = \frac{\varepsilon_0}{|Z|},\;\text{ and }\;\phi = \text{arg}\hat{I} = -\text{arg}Z$$ Observe that through this process, we have simplified the fundamental laws by Kirchhoff and Ohm for AC circuits to ones purely involving complex numbers, giving the same form as the DC case.






%%%%%%%%%%%%%%%%%%%%%% Chapter 2.4
\chapter{Impedances in Series}

We wish to find impedance expressions for circuits with pure resistance, pure capacitance, and pure inductance.

\section{Impedances of Simple Linear Devices}

First we review our time-dependent relations for current of simple circuits with only resistance, capacitance, or inductance, as well as their associated phasor relations in the following table:

\begin{table}[H]
    \centering
    \caption{Voltage-current relations in the time and phasor domains (assumes a voltage waveform $v(t) = V_{max}\cos\omega t$ across the load)}
    \begin{tabular}{c|c|c}
        Load type & time-dependent relation & phasor relation \\ \hline
        Resistive load $(R)$ & $i(t) = \frac{V_{max}}{R}\cos\omega t$ & $\hat{I} = \frac{\hat{V}}{R}$ \\
        Capacitative load $(C)$ & $i(t) = C\omega V_{max}\cos(\omega t + \pi/2)$ & $\hat{I} = C\omega \hat{V}e^{j\pi/2} = j\omega C\hat{V}$ \\ 
        Inductive load $(L)$ & $i(t) = \frac{V_{max}}{\omega L}\cos(\omega t-\pi/2)$ & $\hat{I} = \frac{\hat{V}}{\omega L}e^{-\pi/2} = \frac{\hat{V}}{j\omega L}$ \\ \hline
    \end{tabular}
    \label{tag:simpleLoads}
\end{table}

Note that our definition of complex impedance gives $\hat{I} = \frac{\hat{V}}{Z}$, We collect these results in the following table:


\begin{table}[H]
    \centering
    \caption{Impedances of basic circuit elements ($\omega = $ angular frequency of source)}
    \begin{tabular}{c|c}
        circuit element & impedance, $Z$ \\ \hline 
        Resistor $(R)$ & $R$ \\ 
        Capacitor $(C)$ & $\frac{1}{j\omega C}$ \\
        Inductor $(L)$ & $j\omega L$ \\ \hline
    \end{tabular}
    \label{tag:simpleLoadsImpedances}
\end{table}

\section{The Voltage Divider}

Note that many results from DC circuit theory can be applied to AC circuit theory in the phasor domain. The addition law for a series connection of impedances, for instance, is \Emph{impedances in series add to give the equivalent impedance:} $$Z_{eq} = \sum_{i=1}^nZ_i$$
Note that a string of impedances is considered to be connected in series only when the impedances carry \Emph{exactly the same current}. The common current in the series combination is given by $$\hat{I} = \frac{\hat{V}}{Z_{eq}} = \frac{\hat{V}}{\sum_{i=1}^nZ_i}$$ where $\hat{V}$ is the voltage phasor for the voltage source. The source voltage is divided across the series of impedances so that the \Emph{voltage across the $k$th element in the series is proportional to its own impedance}: $$\hat{V}_k = \hat{I}Z_k = \frac{\hat{V}}{Z_{eq}}Z_k = \frac{Z_k}{\sum_{i=1}^nZ_i}\hat{V}$$ 
\begin{note}
    Note that in the last equation, $\hat{V}_k,Z_k$ and the constant of proportionality $\hat{I}$ are all complex quantities in general. In other words we are adding phasor voltage drops $\hat{V}_k$ to get the source voltage rise $\hat{V}$. Kirchhoff's loop rule dictates that the phasor sum of the voltage drops is equal to the source voltage phasor $$\sum_{i=1}^N\hat{V}_i = \hat{V}$$ and therefor the magnitudes of the phasor sum and the source voltage must be the same $$\left|\sum_{i=1}^N\hat{V}_i\right| = |\hat{V}|$$ but the sum of the individual amplitudes of the sources voltages generally exceeds the amplitude of the phasor sum $$\sum_{i=1}^N|\hat{V}_i| \geq |\hat{V}|$$
    Equality holds only if all the phasors in the summation are collinear and pointing in the same direction. This only occurs if all the elements in the combination have impedances with the same complex phase.
\end{note}


\section{The RC Series Circuit}

We now investigate the RC series circuit, depicted in Figure \ref{fig:RCSeries}:

\begin{figure}[H]
    \centering
    \includegraphics[scale = 0.8]{Images/RCSeries.PNG}
    \caption{RC series circuit.}
    \label{fig:RCSeries}
\end{figure}

We assume the source voltage phasor is purely real and equal to $\hat{V}_S$; equivalently, we will measure the phases of all oscillating quantities relative to the phase of the source voltage, which has phase zero. We first obtain the equivalent complex impedance of the RC series combination in our circuit: $$Z_{eq} = Z_R + Z_C = R + \frac{1}{j\omega C} = \frac{1+j\omega RC}{j\omega C}$$ 
Then the current phasor in the circuit is found using the definition of impedance: $$\hat{I} = \frac{\hat{V}}{Z_{eq}} = \frac{j\omega C}{1+j\omega RC}\hat{V}_S$$
The amplitude and phase of the current phasor are then given by \begin{equation*}
    \hat{I} = \frac{\omega C}{\sqrt{1+\omega^2R^2C^2}}V_S
\end{equation*}
and \begin{equation*}
    \phi = \text{arg}\hat{I} = \text{arg}(j\omega C) - \text{arg}(1+j\omega RC) + \text{arg}V_S = \frac{\pi}{2} - \tan^{-1}(\omega RC)
\end{equation*}
Now we can find the phasor voltage dropped across the resistor and capacitor by using the first form of the voltage divider equations: $$\hat{V}_R = \hat{I}R = \frac{j\omega RC}{1+j\omega RC}\hat{V}_S$$
and so $$|\hat{V}_R| = \frac{\omega RC}{\sqrt{1+\omega^2R^2C^2}}V_S$$
and $$\text{arg}\hat{V}_R = \text{arg}\hat{I}+\text{arg}R = \text{arg}\hat{I} = \frac{\pi}{2} - \tan^{-1}(\omega RC)$$
Similarly, for the capacitor, $$\hat{V}_C = \hat{I}Z_C = \frac{\hat{I}}{j\omega C} = \frac{1}{1+j\omega RC}\hat{V}_S$$ with amplitude $$|\hat{V}_C| = \frac{1}{\sqrt{1+\omega^2R^2C^2}}V_S$$
and phase $$\text{arg}\hat{V}_C = \text{arg}\hat{I} + \text{arg}Z_C = \text{arg}\hat{I} - \pi/2 = - \tan^{-1}(\omega RC)$$

\begin{figure}[H]
    \centering
    \includegraphics[scale = 0.8]{Images/RCGraph.PNG}
    \caption{Graphs of current and voltage amplitudes and phases as a function of angular frequency.}
    \label{fig:RCGraph}
\end{figure}

At low frequencies, $\omega \ll 1/RC$, the impedance $1/j\omega C$ of the capacitor is high, keeping the current amplitude small. The amplitude of the capacitor voltage is close to the source amplitude $V_S$ at these low frequencies. At high frequencies, $\omega \gg 1/RC$, the capacitor impedance is small and the amplitude of the current is limited by the resistor to an asymptotic value of $V_S/R$; the voltage drop across the resistor is approximately $V_S$, with the capacitor voltage drop being very small. At all frequencies the voltage signal dropped across the resistor always leads the voltage signal dropped across the capacitor by $\pi/2$ of phase. Since the phase difference is $\pi/2$, and we have the phasor relation $\hat{V}_S = \hat{V}_R + \hat{V}_C$, we have that $\hat{V}_R$ and $\hat{V}_C$ are orthogonal so pythagoras' theorem applies and $V_S^2 = V_C^2+V_R^2$.

\begin{rmk}[Analog filters]
    The RC circuit is useful as an \Emph{analog filter} to remove unwanted high or low frequency components from a time-continuous signal that contains a spectrum of frequencies. The filter can be said to have a \Emph{characteristic angular frequency} $\omega_C = 1/RC$ that can be seen as separating low frequencies from high frequencies. If we measure an output voltage signal across the resistor, we find that we strongly reject frequencies lying well below $\omega_C$ and retainthose that lie well above $\omega_C$. This is called a \Emph{high pass filter}. On the other hand, if we measure the output signal across the capacitor, we will have a \Emph{low pass filter} which also has its uses. Serious filter design usually requires cascading a number of RC or RLC circuits in a carefully designed series to get reasonable filter preformance. Such filters typically rely on inserting operational amplifiers in the cascade for successful implementation.
\end{rmk}

\begin{rmk}[Transient versus Steady State]
    The ideas of AC circuits that we have developed assume taht the AC source that drives the circuit has been connected for a time sufficiently long enough for exponentially decaying signals, called \Emph{transients}, to have died away, leaving only the simple \Emph{steady state} sinusoidal AC signals which have constant amplitude and phase, and which will persist unchanged until the source is disconnected at some future time, after which new transients emerge which bring down the steady state outputs smoothly and exponentially to zero. When we are thinking in terms of analog filters, we are thinking of the \Emph{steady state response} of the RC circuit.
\end{rmk}






%%%%%%%%%%%%%%%%%%%%%% Chapter 2.5
\chapter{The Series RLC Circuit}

\section{Differential Equation for the RLC circuit}

The RLC circuit is an AC circuit consisting of a capacitor resistor, and inductor all connected in series:

\begin{figure}[H]
    \centering
    \includegraphics[scale = 0.8]{Images/RLCCircuit.PNG}
    \caption{The series RLC circuit, driven by an AC voltage source.}
    \label{fig:RLCCircuit}
\end{figure}

Applying Kirchhoff's loop rule (in the time domain) to the series RLC circuit in Figure \ref{fig:RLCCircuit} gives the following equation expressed in terms of the time-varying charge $q = \int idt$ stored in teh capacitor: \begin{equation*}
    L\frac{d^2q}{dt^2} + R\frac{dq}{dt} + \frac{q}{C} = V_s\cos\omega t
\end{equation*}
Note that this differential equation takes the form of a damped harmonic oscillator with a harmonic driving term: \begin{equation*}
    \ddot{x} + 2\gamma\dot{x} + \omega_0^2x = A\cos\omega t
\end{equation*}
The damping constant for us is $\frac{R}{2L}$, and the natural frequency of the undamped oscillator is $\omega_0 = \frac{1}{\sqrt{LC}}$. The RLC circuit also exhibits \Emph{resonance} (a large current amplitude when the driving frequency $\omega$ is equal to the natural frequency $\omega_0 = 1/\sqrt{LC}$).

\section{Resonance in the Series RLC Circuit}

We assume that the voltage source of Figure \ref{fig:RLCCircuit} has been connected for a long time so that any transient currents have died away, and the AC current in the circuit has reached its \Emph{steady state}. In terms of voltage and current phasors we can write our DE as \begin{equation*}
    \hat{I}(Z_C+Z_R+Z_L) = \hat{V}_s
\end{equation*}
where the impedance terms are $Z_C = 1/j\omega C, Z_R = R, Z_L = j\omega L$. Assigning a phase of zero to the source voltages, thus making $\hat{V}_s$ purely real, results in the following expression for the complex current amplitude in the circuit: \begin{equation*}
    \hat{I} = \frac{V_s}{\frac{1}{j\omega C}+R+j\omega L} = \frac{V_s}{R + j\left[\omega L - \frac{1}{\omega C}\right]}
\end{equation*}
The real amplitude $I$ and phase $\phi$ of the AC current are then \begin{equation*}
    I = \frac{V_s}{\sqrt{R^2 + \left[\omega L - \frac{1}{\omega C}\right]^2}};\;\;\phi = -\tan^{-1}\left[\frac{\omega L}{R} - \frac{1}{\omega RC}\right]
\end{equation*}
Note the current amplitude has a maximum, with respect to $\omega$, when $\omega = \omega_0 = \frac{1}{\sqrt{LC}}$. Thus, the natural angular frequency $\omega_0$ of the undamped oscillator is also the \Emph{resonance frequency} of the RLC circuit. At resonance, $\omega = \omega_0$, the current amplitude achieves its maximum value of $V_s/R$. Also the value of $\phi$ is zero at resonance; this means that the current is in phase with the source voltage when the current amplitude is maximum.

Since the current amplitude at resonance is proportional to $1/R$, changing $R$ while holding $L$ and $C$ fixed will change the height of the peak of the resonance curve without changing the resonance angular frequency, $\omega_0 = 1/\sqrt{LC}$.

One useful time scale in the RLC circuit is $1/\omega_0 = \sqrt{LC}$. Another comes from defining a dimensionless parameter $Q$ called the \Emph{quality factor} of the circuit: \begin{equation*}
    Q = \frac{\sqrt{L/C}}{R}
\end{equation*}
Circuits of given $L$ and $C$ with high $Q$ will have small values of $R$. Low-$Q$ circuits will have high values of $R$. We now define the \Emph{normalized angular frequency} $\overline{\omega}$: \begin{equation*}
    \overline{\omega} = \frac{\omega}{\omega_0} = \omega\sqrt{LC}
\end{equation*}
Using these definitions in the current amplitude and phase equations we can write \begin{equation*}
    I = \frac{V_s/R}{\sqrt{1+Q^2\left[\overline{\omega} - \frac{1}{\overline{\omega}}\right]^2}};\;\;\phi = -\tan^{-1}\left[Q\left(\overline{\omega} - \frac{1}{\overline{\omega}}\right)\right]
\end{equation*}

At resonance, the reactance of the RLC circuit disappears, and the circuit acts as a pure resistance (the impedance becomes purely real). The source voltage is dropped entirely across the resistor, and the current amplitude is $V_s/R$ irrespective of the values of $L$and $C$. It can be seen that increasing $Q$ for given values of $L$ and $C$, by decreasing the resistance $R$ in the circuit, will sharpen the resonance curve. 

\section{Voltage Characteristics}

We now wish to look at the AC voltage signals that appear across each of the elements in the RLC circuit, as represented by their three phasors: \begin{align*}
    \hat{V}_R &= \hat{I}Z_R = \hat{I}R \\
    \hat{V}_C &= \hat{I}Z_C = \hat{I}\frac{1}{j\omega C} \\
    \hat{V}_L &= \hat{I}Z_L = \hat{I}j\omega L
\end{align*}
The phasor $\hat{I} = I(\omega)e^{j\phi(\omega)}$ is common to all three expressions. It follows that \begin{itemize}
    \item The AC voltage signal dropped across the \Emph{resistor} has amplitude $V_R(\omega) = I(\omega)R$ and phase $\phi_R = \phi(\omega)$ (because $R$ is a positive real quantity)
    \item The AC voltage signal across the \Emph{capacitor} has amplitude $V_C(\omega) = I(\omega)/\omega C$ and phase $\phi_C = \phi(\omega) - \pi/2$ (because $\omega C$ is a positive real quantity and $1/j$ has phase $-\pi/2$)
    \item The AC voltage signal across the \Emph{inductor} has amplitude $V_L = \omega LI(\omega)$ and phase $\phi_L = \phi(\omega) + \pi/2$ (because $\omega L$ is a positive real quantity and $j$ has phase $\pi/2$)
\end{itemize}

Using the results for $I(\omega)$ and $\phi(\omega)$ found previously we then have \begin{align*}
    V_R(\omega) &= \frac{V_sR}{\sqrt{R^2+\left[\omega L-\frac{1}{\omega C}\right]^2}};\;\;\;\phi_R(\omega) = \phi(\omega) = -\tan^{-1}\left[\frac{\omega L}{R} - \frac{1}{\omega RC}\right] \\
    V_C(\omega) &= \frac{V_s/\omega C}{\sqrt{R^2+\left[\omega L-\frac{1}{\omega C}\right]^2}};\;\;\;\phi_C(\omega) = -\tan^{-1}\left[\frac{\omega L}{R} - \frac{1}{\omega RC}\right] - \frac{\pi}{2} \\
    V_L(\omega) &= \frac{\omega LV_s}{\sqrt{R^2+\left[\omega L-\frac{1}{\omega C}\right]^2}};\;\;\;\phi_L = -\tan^{-1}\left[\frac{\omega L}{R} - \frac{1}{\omega RC}\right] + \frac{\pi}{2}
\end{align*}





%%%%%%%%%%%%%%%%%%%%%% Chapter 2.6
\chapter{Impedances in Parallel}


%%%%%%%%%%%%%%%%%%%%%% Chapter 2.7
\chapter{Models of Real AC Sources; Th\'{e}venin's Theorem}


%%%%%%%%%%%%%%%%%%%%%% Chapter 2.8
\chapter{Power Dissipation}






%%%%%%%%%%%%%%%%%%%%%%%%%%%%%%%%%%%%% Part 3
\part{Operational Amplifiers, Linear Systems, and the Fourier Transform}


%%%%%%%%%%%%%%%%%%%%%% Chapter 3.1
\chapter{Operational Amplifiers}



%%%%%%%%%%%%%%%%%%%%%% Chapter 3.2
\chapter{Linear Two-ports: Transfer Function, Impulse Response}


%%%%%%%%%%%%%%%%%%%%%% Chapter 3.3
\chapter{Time and Frequency Domains}


%%%%%%%%%%%%%%%%%%%%%% Chapter 3.4
\chapter{Data Sampling}


%%%%%%%%%%%%%%%%%%%%%% Chapter 3.5
\chapter{Analog Filter Design}



%%%%%%%%%%%%%%%%%%%%%%%%%%%%%%%%%%%%% Part 4
\part{Semiconductors, Diodes, and Transistors}



%%%%%%%%%%%%%%%%%%%%%%%%%%%%%%%%%%%%% Part 
\part{Laboratory Designs and Theory}

%%%%%%%%%%%%%%%%%%%%%% Chapter .1


\begin{subappendices}

    \section{Traveling Waves Theory}


    \section{X-Ray Theory}

    \section{Spectroscopy Theory}


\end{subappendices}




%%%%%%%%%%%%%%%%%%%%%% - Appendices
\begin{appendices}
    

    

    %%%%%%%%%%%%%%%%%%%% Section A.2
    \section{Graphing and Lab Reports}

    Tables should be numbered and captioned \Emph{above}. Figures are numbered and captioned \Emph{below} - explain the figure as a whole, what the data points represent, and what the line (if there is one represents). That is explan all elements in the graph in its caption.

    A standard lab report should have the following:

    \begin{itemize}
        \item \textbf{Cover Page:} Includes a title, names, UCIDs (School IDs), Section, Group number, TA's name, Date, and Location
        \item \textbf{Objective:} 2-3 sentences stating the purpose of the experiment/goals to be accomplished\begin{itemize}
                \item ``to determine..."
                \item ``to calculate..."
                \item ``to compare..."
                \item ``to investigate..."
        \end{itemize}
        \item \textbf{Theory:} (\Emph{NOT} a textbook chapter) Should include the most important theory relevant to the experiment. Quote \Emph{all} equations you will use, and \Emph{reference sources}.
        \item \textbf{Experimental Methods (Methods and Materials):} Lists are welcome. Brief descriptions. Feel free to include sketches, and figures (with captions) of the experimental set up if necessary. Briefly describe materials and equipment you will use.
        \item \textbf{Data:} In tables (numbered \& labeled). Include uncertainties and units. 
        \item \textbf{Data Analysis:} Any and all needed calculations. Sample uncertainty calculations. Additional tables (comparisons, graphing, tables, slope/intercept etc.) and graphs.
        \item \textbf{Discussions:} Comparisons of data. Reasoning, discrepancies/agreements, commentary, sources of errors, classification of errors as systematic and random. Compare the obtained values to the expected ones (possibly using a test like the t-test). Comment on the agreement and on possible sources of errors (both random and systematic). Make sure the errors are reasonable given your results.
        \item \textbf{Conclusions:} (Include final values) Appropriate commentary on the agreement/disagreement of derived values. Report your values again and comment on their agreement with accepted values
        \item \textbf{References:}
    \end{itemize}

    \begin{defn}
        A \Emph{Hypothesis} is a supposition or proposed explanation made on the basis of limited evidence as a starting point for the future investigation.
    \end{defn}







    %%%%%%%%%%%%%%%%%%%% Section A.3
    \section{Presentation Tips}

    The fundamentals of a good talk are a\begin{itemize}
        \item Good story: Clear beginning (describe the BIG PICTURE), clear plan (describe your topics and sub-topics if any), and clear answers (present the results and supporting evidence)
        \item Good slides: The three T's - Tell them what you are going to tell them (have a title page and outline slide), tell them (present the main slides), tell them what you told them (have a summary slide)
        \item Good presence: Dress for success (avoid wearing distracting items), engage the audience (look and talk to the audience), don't read the slides (and try to avoid reading from notes), body positioning and body language, and talk \Emph{WITH} not to.
    \end{itemize}

    In a normal talk you would usually budget $1$ slide per minute, maximum. Text should be at least $22$ pt, and you should avoid full sentences or lots of text on any given slide. Images should convey at most one or two concepts. Use reasonable colour contrast. Don't use unnecessary flashiness. KEEP IT SIMPLE SILLY.

    Good powerpoint design:

    \begin{itemize}
        \item Keep it simple/consistent
        \item Use meaningful visuals
        \item Use predetermined colors/templates
        \item Don't overcomplicate/confuse
        \item Be distinctive or be related
    \end{itemize}

    When building a slide deck, have most of the work done already and let the template do the heavy lifting. Take your audience on a visual journey - go bolder on visuals and let the audience focus on you, not text. Know your key-takeaways - leave the audience something to take-away and make it obvious.

    Repetition is a powerful and often useful tool in longer talks/presentations to remind audiences of your key takeaways. Progress bars or numbers can be powerful visual tools for guiding the audience. Use space appropriately, and leave suitable white-space between visuals/text. Less can often be more - use line animation when suitable to help guide visually. Make sure to establish a clear heirarchy in slides to guide the audience's eyes - animation can be a useful guide. Try to align visuals appropriately with suitable white spacing between columns.

    Know your audience - what is their knowledge level? what level of professionalism do they expect?

    
    %%%%%%%%%%%%%%%%%%%% Section A.4
    \section{Paper Tips}

    Basic notes/points: \begin{itemize}
        \item Remember to reference the theory section properly
        \item The literature review is thorough - we compare with multiple pear-reviewed papers.
        \item Explain what every variable in an equation means
        \item Every statement that is not your own stated in the theory should be supported by pear-reviewed papers
        \item All equations must be referenced, and definitions should be referenced
        \item If steps aren't shown, the result must be referenced. 
        \item Reference textbooks for definitions of basic terms.
        \item It is recommended to have 3-5 references per page of the report.
    \end{itemize}


    \subsection{Writing a Research Proposal}

    What is it? It is a document intended to convince others that you have interesting and important research. Why do you need it? To convince the reader of the value of your project/idea, and your \Emph{competence}. It is also to give yourself an opportunity to think through your research project, refine your ideas, and predict any challenges that may arise.

    How to write one: (follow the instructions if given)i break down your proposal into point form. Choose your headings/subheadings. Lay out in point form what you will discuss. Use appropriate clean plain language.

    \subsection{References}

    Follow the \Emph{Journal of Applied Physics} guidelines, \url{https://aip.scitation.org/jap/authors/manuscript}, \url{https://publishing.aip.org/authors/preparing-your-manuscript}, \url{https://aip.scitation.org/journal/jap}, \url{https://publishing.aip.org/wp-content/uploads/2021/03/AIP_Style_4thed.pdf}, and \url{https://paperpile.com/s/journal-of-applied-physics-citation-style/}. It is often best to use alphabetically listed references. For miscellaneous items that don't always appear it may be a good idea to see how APA does it before returning to the JAP guidelines. Choose how the numbering looks to be consistent.

    \subsubsection{Bibliographies}

    For an annotated bibliography, look at the \Emph{Purdue} and \Emph{Cornell} guides for how to build and annotate the bibliographies. Annotations are for you. Mention if its useful to you and if it is trustworthy. You cite yourself like any other source (in square brackets put that it is a lab manual and unpublished). 

    An annotated bibliography is a list of citations, where each citation is followed by an annotation - a brief (100-150 words) descriptive and evaluative paragraph. 


    \subsection{Example-Outline}

    A large centered or left aligned title followed by names of the authors with footnotes referencing their affiliations. Date the paper, and include a centered clear abstract (at most 10 sentences), with keywords after the abstract. An experimental paper should include:
    \begin{itemize}
        \item \Emph{Introduction:}
        \item \Emph{Methods:}
        \item \Emph{Results and Discussions:}
        \item \Emph{Conclusions:}
    \end{itemize}

    Methods may be replaced with \Emph{Theory} if a non-experimental paper is being designed.



    \subsection{Peer Reviewing}

    Helpful feedback follows the four following principles: \begin{itemize}
        \item \Emph{Goal Referenced:} Is linked to the goal specified by the reciever. Improving/developing presentation/writing skills.
        \item \Emph{Transparent:} Making the ``how, what and why" behind editorial decisions clear to everyone - if possible/provided include page/line numbers for specific feedback. Content of the review is known not the identity of the reviewer.
        \item \Emph{Actionable:} Comment on things that actually can be improved. Provide feedback that strengthens the work.
        \item \Emph{User-Friendly:} Be objective. Back up each statement with an example. Be honest.
    \end{itemize}




\end{appendices}


\end{document}


%%%%%% END %%%%%%%%%%%%%
