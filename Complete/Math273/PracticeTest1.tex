%%%
\documentclass[12pt]{article}
%%%%%%%%%%%%%% PACKAGES
%
% amsmath, amsthm, makeidx are included in the amsart document class
%
%
\usepackage[margin=1in]{geometry}
\usepackage{enumitem}
%set margins to 1 inch
\usepackage{amsmath}
\usepackage{amssymb}
\usepackage{amsfonts}
\usepackage{amsthm}% AMS theorem environments and proof environment -- load after amsmath
\renewcommand\qedsymbol{$\blacksquare$}
\usepackage{latexsym}
\usepackage{graphicx}    % standard LaTeX graphics tool
\usepackage{xypic}      % commutative diagrams
\usepackage[pdfstartview=FitH,pdfpagemode=UseNone]{hyperref} %hyper-reference sections, \ref and citations
\usepackage{pbox}
\usepackage{tikz} % for drawing technical diagrams, graphing etc.
\usetikzlibrary{calc}
\usepackage{pgfplots}
\pgfplotsset{compat=1.11}
\usepackage[makeroom]{cancel}
%%%%%%%%%%%%%%%%% HEADERS %%%%%%%%%%%%%%
%
%   UPDATE THIS INFORMATION
%
% Here you can change the header information to display
%    the assignment number (top left); and 
%     your name and UCID on the top right of each of the pages 
%
%
\usepackage{fancyhdr}
\pagestyle{fancy}
\lhead{MATH 273 Practice Test \# $1$}
\rhead{December 7th, 2020}
\chead{}
%\lfoot{Author's Name}
%\cfoot{}
%\rfoot{Page \thepage}

%%%%%%%%%%%%%%% ENVIRONMENTS
\newtheorem{thm}{Theorem}[section]
\newtheorem*{thm*}{Theorem}
\newtheorem{lem}[thm]{Lemma}  %%%% [thm] means number in sequence with Theorem
\newtheorem{cor}[thm]{Corollary}
\newtheorem{prop}[thm]{Proposition}
%%% definition style
\theoremstyle{definition}
\newtheorem{defn}[thm]{Definition}
\newtheorem{eg}[thm]{Example}
\newtheorem{xca}[thm]{Exercise}
\newtheorem{conj}[thm]{Conjecture}
%%% remark style
\theoremstyle{remark}
\newtheorem{rmk}[thm]{Remark}
\newtheorem*{qst}{Question}
\newtheorem{obs}[thm]{Observation}
\newtheorem*{note}{Note} %%%%%%%%%% no numbering for notes
\numberwithin{equation}{section}
\newcounter{questionctr}
\newenvironment{question}[1]
        {
        \bigskip\noindent
        \refstepcounter{questionctr}
        \textsc{\textbf{Question \thequestionctr} (#1 points).}
        \newline
        }
        {\par\bigskip
}

%%%%%%%%%%%% MACROS
%% new commands 
\newcommand{\ip}[2]{\langle #1 , #2 \rangle}    %this is to get the correct brackets for inner product
\newcommand{\abs}[1]{\lvert#1\rvert}
\newcommand\diag{\operatorname{diag}}   %%%%%%%%% diag matrix
\newcommand\tr{\mbox{tr}\,}   %%%%%%%%% trace
\newcommand\C{\mathbb C}    %%%%%%%%% the set of complex numbers
\newcommand\R{\mathbb R}    %%%%%%%%% the set of real numbers
\newcommand\Z{\mathbb Z}    %%%%%%%%% the set of integers
\newcommand\N{\mathbb N}    %%%%%%%%% the natural numbers
\newcommand\Q{\mathbb Q}    %%%%%%%% the rational numbers
%% math operators
\DeclareMathOperator{\ran}{Im} %% image
\DeclareMathOperator{\aut}{Aut} %% automorphism
\DeclareMathOperator{\spn}{span} %% span
\DeclareMathOperator{\ann}{Ann} %% annihilator 
\DeclareMathOperator{\rank}{rank} %% rank
\DeclareMathOperator{\ch}{char}        %% characteristic
\DeclareMathOperator{\ev}{\bf{ev}}    %% evaluation
\DeclareMathOperator{\sgn}{sgn} %% sign
\DeclareMathOperator{\id}{Id} %% identity
\DeclareMathOperator{\supp}{Supp}%%%%%%%% support
%%%%%%%%%%%%%% END MACROS

%---BEGIN DOCUMENT----------
% 
\begin{document}

% Write comments following a % symbol

%--------------------- COVER PAGE


%--------------------- PROBLEM SOLUTIONS

\clearpage % Start solutions on a new page

\begin{question}{4}
        Prove that any natural number $n > 1$ is either prime or composite.
\end{question}

\vspace{10cm}

\begin{question}{3}
        Let $p > 2$ be a prime. Find all solutions to the congruence equation $x^2 \equiv x \mod p$.
\end{question}

\vspace{10cm}


\begin{question}{4}
        Define a sequence by $F_0 = 0$, $F_1 = 1$, and $F_n = F_{n-1} + F_{n-2}$ for $n \geq 2$. Prove by induction that \begin{equation*}
                F_n = \frac{(\alpha_+)^n-(\alpha_-)^n}{\alpha_+-\alpha_-}
        \end{equation*}
        where $\alpha_+ = \frac{1+\sqrt{5}}{2}$ and $\alpha_- = \frac{1-\sqrt{5}}{2}$.
\end{question}

\vspace{10cm}


\begin{question}{3}
        Prove that for all sets $A$ and $B$, $(A\backslash B)\cup(B\backslash A) = (A\cup B)\backslash (A \cap B)$. 
\end{question}

\vspace{10cm}


\begin{question}{4}
        Prove that for all $n \in \N$, given sets $A_1,...,A_n$, we have that: $$\left(\bigcup\limits_{i=1}^nA_i\right)^c = \bigcap\limits_{i=1}^nA_i^c$$
\end{question}

\vspace{10cm}

\begin{question}{3}
        Prove that a finite union of countable sets is countable.
\end{question}

\vspace{10cm}


\begin{question}{6}
        Prove that if A is an infinite set, then A is countable if and only if there exists a surjection from $\N$ to A or an injection from A to $\N$
\end{question}

\vspace{10cm}


\begin{question}{5}
        Prove that if $x,y \in \Z$ and $xy = 0$, then $x = 0$ or $y=0$.
\end{question}

\vspace{10cm}


\begin{question}{5}
        Prove that the rationals are countable.
\end{question}

\vspace{10cm}


\begin{question}{6}
        Find a Cauchy sequence $\{a_n\}$ of rational numbers such that $\{(a_n)^2\}$ converges to 2.
\end{question}

\vspace{10cm}


\begin{question}{6}
        Prove that the Nested Interval Property and the Least Upper Bound Property are equivalent
\end{question}

\vspace{10cm}


\begin{question}{4}
        Suppose that $n \in \N$ and $z \in \C$ so that $z^n = 1$. Find and prove a formula for $$\prod\limits_{k=1}^{n-1}z^k$$
\end{question}




%-------------------Bibliography---------------------
% Uncomment to create a bibliography

%\bibliographystyle{amsalpha}
%\begin{thebibliography}{H}
%\bibitem[H]{H} Robin Hartshorne, Algebraic Geometry. Springer-Verlag New York Inc.\, GTM \textbf{52}, (1977).
%\end{thebibliography}


%-------------------End document----------------------------
\end{document}
