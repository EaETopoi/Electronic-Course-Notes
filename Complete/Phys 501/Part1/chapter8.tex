%%%%%%%%%%%%%%%%%%%%% chapter.tex %%%%%%%%%%%%%%%%%%%%%%%%%%%%%%%%%
%
% sample chapter
%
% Use this file as a template for your own input.
%
%%%%%%%%%%%%%%%%%%%%%%%% Springer-Verlag %%%%%%%%%%%%%%%%%%%%%%%%%%
%\motto{Use the template \emph{chapter.tex} to style the various elements of your chapter content.}
\chapter{Gravitational Collapse and Black Holes}
\label{GravCol} % Always give a unique label
% use \chaptermark{}
% to alter or adjust the chapter heading in the running head



\abstract{To be completed once done}

\section{The Schwarzschild Black Hole}
\label{sec:schwarzBlack}

\subsection{Eddington-Finkelstein Coordinates}

The geometry outside a spherically symmetric gravitational collapse is the time-independent Schwarzschild geometry. The singularity in the Schwarzschild metric at $r = 2M$ turns out not to be a singularity in the geometry of spacetime, but a singularity in Schwarzschild coordinates. It is a \textbf{coordinate singularity}.

BDefine a coordinate $v$ by $$v = t+r+2M\log\left|\frac{r}{2M}-1\right|$$
Starting from either $r < 2M$ or $r > 2M$ and transforming $t$ to $v$ in the line element for Schwarzschild coordinates $(t,r,\theta,\phi)$ gives the result $$ds^2 = -\left(1-\frac{2M}{r}\right)^2+2dvdr+r^2(d\theta^2+\sin^2\theta d\phi^2)$$
This line element is fit for describing physics outside, at, and inside the Schwarzschild radius.

At large $r$, the metric approaches a flat metric. The metric is now non-diagonal with off-diagonal entries $g_{vr} = g_{rv} = 1$.

At $r = 0$ the metric is singular in both the Schwarzschild and Eddington-Kinkelstein coordinate systems. $r = 0$ is a place of infinite spacetime curvature and infinite gravitational forces---a real singularity.

\subsection{Light Cones of the Schwarzschild Geometry}

Consider radial light rays, i.e. ones which satisfy $d\theta = d\phi = 0$ and $ds^2 = 0$, so $$-\left(1-\frac{2M}{r}\right)dv^2+2dvdr = 0$$
An immediate consequence is that some radial light rays move along the curves $$v = \text{const}.\;\text{(ingoing radial light rays)}$$
As $t$ increases, $r$ must decrease to keep $v$ constant. The other possible solution is $$-\left(1-\frac{2M}{r}\right)dv + 2dr = 0$$
which can be solved to obtain $$v-2\left(r+2M\log\left|\frac{r}{2M}-1\right|\right) = \text{const}$$
This corresponds to radial light rays which are outgoing if $r > 2M$ and ingoing if $r < 2M$. The curve $r = 2M$ also satisfies these equations, describing light rays that are neither ingoing nor outgoing but instead are stationary.

The surface $r = 2M$ is called the \textbf{event horizon}.


\subsection{Geometry of the Horizong and Singularity}

The horizon $r = 2M$ is a three-dimensional null surface in spacetime. Its normal vector points in the $r$-direction and is a null vector. The horizon is generated by those radial light rays that neither fall into the singularity nor escape to infinity.

A $v = $ const slice of the horizon is a two-surface with the metric $d\Sigma^2 = (2M)^2(d\theta^2+\sin^2\theta d\phi^2)$. This is the geometry of a sphere with area $A = 16\pi M^2$, which is called the \textbf{area of the horizon}. Inside $r = 2M$ surfaces of constant $r$ are spacelike. In particular, the singularity $r = 0$ is a spacelike surface. The $r = 0$ singularity in the Schwarzschild is not a place in space; it is a moment in time.


\section{Collapse to a Black Hole}


The radially moving particles at the surface of the collapsing star follow timelike world lines that lie inside the light cone at each point of spacetime they pass through, just like any other particle.



\section{Kruskal-Szekeres Coordinates}


\subsection{Relation to Schwarzschild Coordinates}

\textbf{Kruskal-Szekeres} coordinates are denoted by $(V,U,\theta,\phi)$. The $\theta,\phi$ coordinates are the same as the Schwarzschild polar angles, but Schwarzschild $t$ and $r$ are traded for $V$ and $U$ according to the transformations \begin{align*}
    U &= \left(\frac{r}{2M}-1\right)^{1/2}e^{r/4M}\cosh\left(\frac{t}{4M}\right) \\
    V &= \left(\frac{r}{2M}-1\right)^{1/2}e^{r/4M}\sinh\left(\frac{t}{4M}\right)
\end{align*}
for $r > 2M$, and \begin{align*}
    U &= \left(1-\frac{r}{2M}\right)^{1/2}e^{r/4M}\sinh\left(\frac{t}{4M}\right) \\
    V &= \left(1-\frac{r}{2M}\right)^{1/2}e^{r/4M}\cosh\left(\frac{t}{4M}\right)
\end{align*}
for $r < 2M$. The metric in these coordinates takes the form \begin{equation*}
    \boxed{ds^2 = \frac{32M^3}{r}e^{-r/2M}\left(-dV^2+dU^2\right)+r^2\left(d\theta^2+\sin^2\theta d\phi^2\right)}
\end{equation*}
Here $r$ is considered as a function of $V$ and $U$, $r = r(V,U($, defined implicitly by $$\left(\frac{r}{2M}-1\right)e^{r/2M} = U^2-V^2$$
Lines of constant $r$ are curves of constant $U^2-V^2$ and therefore hyperbolas in the $UV$ plane. The value $r = 2M$ corresponds to either of the straight lines $V = \pm U$. The value $r = 0$ correspons=ds to the hyperbola $$V = \sqrt{U^2+1}$$
Lines of constant $t$ are described by $$V = \tanh(t/4M)U$$ 
for $r > 2M$ and $$U = \tanh(t/4M)V$$
for $r < 2M$.

For the entire range of $(V,U,\theta,\phi)$, the metric component $g_{VV}$ is negative, whereas $g_{UU}, g_{\theta\theta},g_{\phi\phi}$ are always positive. The direction along $V$ is always timelike, and the direction along $U$ is always spacelike.              

\subsection{Light Cones, Horizon, and Spherical Collapse}

Radial light rays are described by the curves $$V = \pm U+\text{const}$$
Radial particle world lines must lie within the $45^{\circ}$ lines with slopes that are greater than unity. Indeed, $|dV/dU| > 1$ for a particle wordl line whether it is moving radially or not.



% Problems or Exercises should be sorted chapterwise
\section*{Problems}
\addcontentsline{toc}{section}{Problems}
%
% Use the following environment.
% Don't forget to label each problem;
% the label is needed for the solutions' environment
\begin{prob}
\label{prob1}
A given problem or Excercise is described here. The
problem is described here. The problem is described here.
\end{prob}

% \begin{prob}
% \label{prob2}
% \textbf{Problem Heading}\\
% (a) The first part of the problem is described here.\\
% (b) The second part of the problem is described here.
% \end{prob}

%%%%%%%%%%%%%%%%%%%%%%%% referenc.tex %%%%%%%%%%%%%%%%%%%%%%%%%%%%%%
% sample references
% %
% Use this file as a template for your own input.
%
%%%%%%%%%%%%%%%%%%%%%%%% Springer-Verlag %%%%%%%%%%%%%%%%%%%%%%%%%%
%
% BibTeX users please use
% \bibliographystyle{}
% \bibliography{}
%


% \begin{thebibliography}{99.}%
% and use \bibitem to create references.
%
% Use the following syntax and markup for your references if 
% the subject of your book is from the field 
% "Mathematics, Physics, Statistics, Computer Science"
%
% Contribution 
% \bibitem{science-contrib} Broy, M.: Software engineering --- from auxiliary to key technologies. In: Broy, M., Dener, E. (eds.) Software Pioneers, pp. 10-13. Springer, Heidelberg (2002)
% %
% Online Document

% \end{thebibliography}

