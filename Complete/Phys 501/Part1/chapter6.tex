%%%%%%%%%%%%%%%%%%%%% chapter.tex %%%%%%%%%%%%%%%%%%%%%%%%%%%%%%%%%
%
% sample chapter
%
% Use this file as a template for your own input.
%
%%%%%%%%%%%%%%%%%%%%%%%% Springer-Verlag %%%%%%%%%%%%%%%%%%%%%%%%%%
%\motto{Use the template \emph{chapter.tex} to style the various elements of your chapter content.}
\chapter{Geodesics}
\label{geod} % Always give a unique label
% use \chaptermark{}
% to alter or adjust the chapter heading in the running head



\abstract{To be completed once done}

\section{The Geodesic Equation}
\label{sec:geodEq}

Both experimentally and theoretically, the curved spacetimes of general relativity are explored by studying how test particles and light rays move through them.

\begin{nthm}{Variational Principle for Free Test Particle Motion}
    The world line of a free test particle between two timelike separated points extremizes the proper time between them.
\end{nthm}

Extremal proper time world lines are called \textbf{geodesics}, and the equations of motion that determine them comprise the \textbf{geodesic equation}.

For a general metric $g_{\alpha\beta}$, we consider the variational principle $$\delta\int(-g_{\alpha\beta}dx^{\alpha}dx^{\beta})^{1/2} = 0$$
which has equations of motion $$\frac{d^2x^{\alpha}}{d\tau^2} = -\Gamma_{\beta\gamma}^{\alpha}\frac{dx^{\beta}}{d\tau}\frac{dx^{\gamma}}{d\tau}$$
where $\Gamma_{\beta\gamma}^{\alpha}$ are the Christoffel symbols for the metric. We can describe a world line parametrically by giving the four coordinates $x^{\alpha}$ as a function of a parameter $\sigma$ that varies between $\sigma = 0$ at a point $A$ to $\sigma = 1$ at a point $B$. We can rewrite the proper time between $A$ and $B$ as $$\tau_{AB} = \int_0^1d\sigma\left(-g_{\alpha\beta}(x)\frac{dx^{\alpha}}{d\sigma}\frac{dx^{\beta}}{d\sigma}\right)^{1/2}$$
Applying Euler Lagrange equations we have that $$\frac{d}{dt}\left[\frac{\partial \mathcal{L}}{\partial \dot{x}^{\alpha}}\right] = \frac{\partial\mathcal{L}}{\partial x^{\alpha}}$$
The Christoffel symbols $\Gamma_{\beta\gamma}^{\alpha}$ are constructed from the metric and its first derivatives. In terms of the four-velocity $u^{\alpha} = dx^{\alpha}/d\tau$ we have $$\frac{du^{\alpha}}{d\tau} = -\Gamma_{\beta\gamma}^{\alpha}u^{\beta}u^{\gamma}$$
The Christoffel symbols may be taken to be symmetric in the lower two indices. We can express the Christoffel symbols by \begin{equation*}
    g_{\alpha\delta}\Gamma_{\beta\gamma}^{\delta} = \frac{1}{2}\left(\frac{\partial g_{\alpha\beta}}{\partial x^{\gamma}} + \frac{\partial g_{\alpha\gamma}}{\partial x^{\beta}} - \frac{\partial g_{\beta\gamma}}{\partial x^{\alpha}}\right)
\end{equation*}


\section{Solving the Geodesic Equation--Symmetries and Conservation Laws}
\label{sec:solvGeodEq}

The normalization of the four-velocity gives the equation $$-1 = u\cdot u = g_{\alpha\beta}\frac{dx^{\alpha}}{d\tau}\frac{dx^{\beta}}{d\tau}$$

Energy is conserved when there is a symmetry under displacements in time, linear momentum is conserved when there is a symmetry under displacements in space, and angular momentum is conserved when there is a symmetry under rotations.

A \textbf{Killing vector} is a general way of characterizing symmetry in any coordinate system. A symmetry implies a conserved quantity along a geodesic. In an arbitrary coordinate system, a conserved quantity along a geodesic is a Killing vector $\xi$ such that $$\xi\cdot u = \text{constant}$$


\section{Null Geodesics}

Light rays move along null world lines for which $ds^2 = 0$. If $x^{\alpha}(\lambda)$ is the path of a light ray parameterized by some $\lambda$ and $u^{\alpha} := dx^{\alpha}/d\lambda$ is the tangent vector, then $$u\cdot u = g_{\alpha\beta}(x)\frac{dx^{\alpha}}{d\lambda}\frac{dx^{\beta}}{d\lambda} = 0$$
The equation for \textbf{null geodesics} is $$\frac{d^2x^{\alpha}}{d\lambda^2} = -\Gamma_{\beta\gamma}^{\alpha}\frac{dx^{\beta}}{d\lambda}\frac{dx^{\gamma}}{d\lambda}$$
Note the affine parameter $\lambda$ is not a spacetime distance. Rather it is a parameter chosen so that the equation above takes the form of the geodesic equation.



% Problems or Exercises should be sorted chapterwise
\section*{Problems}
\addcontentsline{toc}{section}{Problems}
%
% Use the following environment.
% Don't forget to label each problem;
% the label is needed for the solutions' environment
\begin{prob}
\label{prob1}
A given problem or Excercise is described here. The
problem is described here. The problem is described here.
\end{prob}

% \begin{prob}
% \label{prob2}
% \textbf{Problem Heading}\\
% (a) The first part of the problem is described here.\\
% (b) The second part of the problem is described here.
% \end{prob}

%%%%%%%%%%%%%%%%%%%%%%%% referenc.tex %%%%%%%%%%%%%%%%%%%%%%%%%%%%%%
% sample references
% %
% Use this file as a template for your own input.
%
%%%%%%%%%%%%%%%%%%%%%%%% Springer-Verlag %%%%%%%%%%%%%%%%%%%%%%%%%%
%
% BibTeX users please use
% \bibliographystyle{}
% \bibliography{}
%


% \begin{thebibliography}{99.}%
% and use \bibitem to create references.
%
% Use the following syntax and markup for your references if 
% the subject of your book is from the field 
% "Mathematics, Physics, Statistics, Computer Science"
%
% Contribution 
% \bibitem{science-contrib} Broy, M.: Software engineering --- from auxiliary to key technologies. In: Broy, M., Dener, E. (eds.) Software Pioneers, pp. 10-13. Springer, Heidelberg (2002)
% %
% Online Document

% \end{thebibliography}

