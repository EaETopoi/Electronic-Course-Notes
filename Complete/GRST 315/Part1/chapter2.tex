%%%%%%%%%%%%%%%%%%%%% chapter.tex %%%%%%%%%%%%%%%%%%%%%%%%%%%%%%%%%
%
% sample chapter
%
% Use this file as a template for your own input.
%
%%%%%%%%%%%%%%%%%%%%%%$% Springer-Verlag %%%%%%%%%%%%%%%%%%%%%%%%%%
%\motto{Use the template \emph{chapter.tex} to style the various elements of your chapter content.}
\chapter{Performing Sappho}
\label{PerfSappho} % Always give a unique label
% use \chaptermark{}
% to alter or adjust the chapter heading in the running head


%%% Questions to think about
%The \textbf{Thesis} or general sense of the article is ...

%The \textbf{method} the author uses to argue their point is ...

%In their \textbf{analysis} the author uses tools such as ...
% How do they look at the evidence? Do they place it in some theoretical framework? (i.e. gender studies, music studies, etc.)

%Additionally they conclude ...
% How does this compare to others throughout time? What is the societal context?

%What connections does the author portray with regard to \textbf{space}, \textbf{relationships}, \textbf{occupation}, and \textbf{religion}.


\abstract{Ferrari aims to provide insight into how and where Sappho's poems were performed, using a combination of literary sources, visual sources, and arceological sources. Ferrari also compares and contrasts his cocnlusions with the conclusions published in the scholarship. In particular, Ferrari agues and concludes that the stage directions found with regard to Sappho's work indicate references to night festivals, temples, shrines, groves, altars, sacrifices, musical instruments, choral groups, cloakcs, crowns, and so on, all of which seem to be organically linked to the festivals---weddings, rites in honour of Aphrodite, the Adonia (festival celebrated annually by women in ancient Greece to mourn the death of Adonis, the consort of Aphrodite), cultic ceremonies for Hera at Messon and for the Nereids in the bay of Pyrrha, and so on. This animates the communal life of Mytilene (capital of Lesbos) and the island of Lesbos as a whole. As method Ferrari primarily analyses poems and poem fragments of Sappho with an emphasis on possible translations of the greek text and what they imply in context.}

\section{Questions and Remarks}
\label{sec:QR2}

\begin{qst}
    What does it mean to ``perform Sappho"?
\end{qst}

\begin{qst}
    What is the Book of Epithalamia?
\end{qst}
\textbf{Answer.} It is a wook of wedding songs


\begin{qst}
    Who is Charaxus, and why do they matter?
\end{qst}
\textbf{Answer.} They are Sappho's brother.

\begin{qst}
    What is Pathography?
\end{qst}



\section{First Reading}
\label{sec:FirRead2}


\subsection{Viewpoints}

\textbf{Epithalamia} (wedding songs) and \textbf{Contrasto} (dialogue poem) are poems of Sappho which were performed for Adonis' death, both of which can be assigned to choral performance before a large audience.

According to Fr\"{a}nkel, Sappho and her followers worshipped the gods with songs and dances not only during festive occasions but also according to sudden, personal impulses; for Page, Sappho performed her poems `informally to her companions'; for Merkelbach, nearly all of Sappho's poems were intended for a group of girls taht she directed; for West, Sappho's poems were ``music and song, for public as well as private performance'; and for Aloni there were three groups of poems: \textbf{ritual songs}, \textbf{poems destined to heterogenous people}, \textbf{poems addressed to a narrow female audience}. The existence of such a circle, in any form, has also been denied by other scholars.

That teenage girls, not peers, were Sappho's privileged interlocutors and customary emotional reference can be inferred from clues in certain poem fragments.

\subsection{Night Parties}

Evidence can be found in portraits and illustrations from the time, though Sappho appears rarely.

\subsection{The Book of Epithalamia}



\subsection{Interiors and Externals}

The dichotomy that characterises the poem was accompanied by a contrast between spaces internal and external to Sappho's house---an opposition between private activity related to elegance, beauty, and the pleasure of sleeping together, and activities tied to festive occasions and musical performances.

\subsection{Cretan Aphrodite}

The sacred shrine, the insistence on vegatation, the presence of a group of devotees celebrating a rite that includes a state of trance, the sharing of a drink connected with immortality---all these elements suggest a mystery rite in honour of Aphrodite.

Crete is the place from which Aphrodite should arrive.

\subsection{Worries for Charaxus}

Charaxus is Sappho's brother.


\subsection{Old Age}

The enumeration of the symptoms of old age culminates in the impossibility of dancing on the part of the speaker. THis implies Sappho must have been inviting the young girls of her group to dance, not to listen or play. Old age prevents the poet from leading the dances of the chorus as she once used to do:

\begin{quotation}
    Come to the splendid sanctuary of ox-eyed Hera, girls of Lesbos, whirling the delicate steps of your feet. Form there a beautiful chorus in honor of the goddess; Sappho will lead you with the golden lyre in her hands. Blessed you in the joy of your dance: surely you will believe you are listening to the sweet song of Calliope herself.
\end{quotation}



\subsection{Sing for Us!}



\subsection{Erotic Pathography}


\subsection{Mourning Becomes Not Sappho}



\subsection{Towards a Conclusion}

The stage directions emergin from Sappho's remains do not refer either to home performances within a small circle of girls or to those sympotic meetings reflected in the poems of Alcaeus and elsewhere in Greek lyric poetry. Instead, references to night festivals, temples, shrines, groves, altars, sacrifices, musical instruments, choral groups, cloacks, crowns, and so on seem organically linked to the festivals---weddings, rites in honour of Aphrodite, the Adonia, cultic ceremonies for Hera at Messon adn for the Nereids in the bay of Pyrrha, and so on---which animated the communal life of Mytilene and of the island of Lesbos. On the one hand poems, usually choral, marking the key moments of public events; on the other, monodic (eventually with dancing accompaniment) or choral poems related to emotions, situations, and relationships that interested Sappho and her adepts in the context of cultic occasions.



%
% \begin{acknowledgement}
% If you want to include acknowledgments of assistance and the like at the end of an individual chapter please use the \verb|acknowledgement| environment -- it will automatically render Springer's preferred layout.
% \end{acknowledgement}
%
% \section*{Appendix}
% \addcontentsline{toc}{section}{Appendix}
%


% Problems or Exercises should be sorted chapterwise
\section*{Problems}
\addcontentsline{toc}{section}{Problems}
%
% Use the following environment.
% Don't forget to label each problem;
% the label is needed for the solutions' environment
\begin{prob}
\label{prob1}
A given problem or Excercise is described here. The
problem is described here. The problem is described here.
\end{prob}

% \begin{prob}
% \label{prob2}
% \textbf{Problem Heading}\\
% (a) The first part of the problem is described here.\\
% (b) The second part of the problem is described here.
% \end{prob}

%%%%%%%%%%%%%%%%%%%%%%%% referenc.tex %%%%%%%%%%%%%%%%%%%%%%%%%%%%%%
% sample references
% %
% Use this file as a template for your own input.
%
%%%%%%%%%%%%%%%%%%%%%%%% Springer-Verlag %%%%%%%%%%%%%%%%%%%%%%%%%%
%
% BibTeX users please use
% \bibliographystyle{}
% \bibliography{}
%


% \begin{thebibliography}{99.}%
% and use \bibitem to create references.
%
% Use the following syntax and markup for your references if 
% the subject of your book is from the field 
% "Mathematics, Physics, Statistics, Computer Science"
%
% Contribution 
% \bibitem{science-contrib} Broy, M.: Software engineering --- from auxiliary to key technologies. In: Broy, M., Dener, E. (eds.) Software Pioneers, pp. 10-13. Springer, Heidelberg (2002)
% %
% Online Document

% \end{thebibliography}

