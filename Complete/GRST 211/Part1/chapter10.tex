%%%%%%%%%%%%%%%%%%%%% chapter.tex %%%%%%%%%%%%%%%%%%%%%%%%%%%%%%%%%
%
% sample chapter
%
% Use this file as a template for your own input.
%
%%%%%%%%%%%%%%%%%%%%%%$% Springer-Verlag %%%%%%%%%%%%%%%%%%%%%%%%%%
%\motto{Use the template \emph{chapter.tex} to style the various elements of your chapter content.}
\chapter{Shoulders and Arms}
\label{Shoulders} % Always give a unique label
% use \chaptermark{}
% to alter or adjust the chapter heading in the running head


%%% Questions to think about
%The \textbf{Thesis} or general sense of the article is ...

%The \textbf{method} the author uses to argue their point is ...

%In their \textbf{analysis} the author uses tools such as ... 

%Additionally they conclude ...

%What connections does the author portray with regard to \textbf{space}, \textbf{relationships}, \textbf{occupation}, and \textbf{religion}.


\abstract{}


\section{Notes}
\label{sec:NOTE10}



\subsection{Suffixes}

\begin{longtable}{c | p{0.4\textwidth} | p{0.4\textwidth}}
    \caption{Suffixes for the shoulders and arms.}
    \hline
    Suffix & Meaning(s) & Example(s) \\ \hline
        -ated & `composed of,' `having' & constapated \\
    \label{tab:Ch10Suffix}
\end{longtable}


\subsection{Bases}

\begin{longtable}{c | p{0.4\textwidth} | p{0.4\textwidth}}
    \caption{Bases for the shoulders and arms.}
    \hline
    Base & Meaning(s) & Example(s) \\ \hline
        SCAPUL- & `shoulder blade,' `scapula' & scapular (SCAPUL-ar) - pertaining to the shoulder blade, scapulodynia (SPACUL-ODYN-ia) - condition of pain in the shoulder blade \\
        BRACHI- & `arm' & brachiocephalic (BRACHI-O-CEPHAL-ic) - pertaining to the head and arm, cervicobrachial (CERVIC-O-BRACHI-al) - pertaining to the arm and neck, macrobrachia (MACR-O-BRACH(I)-ia) - condition of the arm that is (abnormally) large \\
        MEL- & `limb' & amelia (a-MEL-ia) - condition of without a limb, erythromelalgia (ERYTHR-O-MEL-algia) - painful condition of a limb (accompanied by) redness of the part, megalomelia (MEGAL-O-MEL-ia) - condition of a limb that is (abnormally) large \\
        HUMER- & `upper arm bone,' `humerus' & humeral (HUMER-al) - pertaining to the upper bone of the arm \\
        ULN- & `ulna' & ulnad (ULN-ad) - toward the ulna, ulnar (ULN-ar) - pertaining to the ulna \\
        RADI- & `ray,' `spoke of a wheel,' `radius' & radiate (RADI-ate) - to (extend like) spokes of a wheel, radiodermatitis (RADI-O-DERMAT-itis) - inflammation of the skin (caused by exposure to) rays (of electromagnetic radiation), radioulnar (RADI-O-ULN-ar) - pertaining to the ulnar and radius \\
        CUB-, CUBIT-, CUMB- & `to lie down,' & cubation (CUB-ation) - process of lying down, decubital (de-CUBIT-al) - pertaining to  (occuring) from lying down, procumbent (pro-CUMB-ent) - pertaining to forward lying down \\
        CUBIT- & `elbow,' `forearm'& cubital (CUBIT-al) - pertaining to the elbow or forearm, brachiocubital (BRACHI-O-CUBIT-al) - pertaining to the elbow and arm \\
        ANCON- & `elbow' & anconad (ANCON-ad) - toward the elbow, anconitis (ANCON-itis) - inflammation of the elbow, anconeal (ANCON-eal) - pertaining to the elbow, \\
        GLEN- & `socket of a joint' & glenoid (GLEN-oid) - resembling a socket of a joint \\
        GLEN- & `glenoid' (the shallow depression on the scapula that engages with the head of the humerus bone) & glenohumeral (GLEN-O-HUMER-al) - pertaining to the humerus and the glenoid \\
        CORAC- & `crow's beak' & coracoid (CORAC-oid) - resembling a crow's beak \\
        CORAC- & `coracoid process' (Process on the scapula that is shaped like a crow's beak, where a process is a bony projection that provides attachment points for muscles and ligaments) & coracoclavicular (CORAC-O-CLAVICUL-ar) - pertaining to the clavicle and the coracoid process \\
        ACROMI- & `acromion' & subacromial (sub-ACROMI-al) - pertaining to below the acromion, acromioclavicular (ACROMI-O-CLAVICUL-ar) - pertaining to the clavicle and acromion \\
        CAPIT-, CIPIT- & `head' & capitate (CAPIT-ate) - having a head, capitulum (CAPIT-ulum) - small head, capitellum (CAPIT-ellum) - small head, bicipital (BI-CIPIT-al) - pertaining to heads - two of them, tricipital (TRI-CIPIT-al) - pertaining to heads - three of them \\
        DELT- & `Greek letter delta' & deltoid (DELT-oid) - shaped like the Greek letter delta \\
        TRAPEZ- & `table' & trapezoid (TRAPEZ-oid) - resembling a table, trapezium (TRAPEZ-ium) - small table \\
        TRAPEZI- & `trapezium' (a four-sided figure with no parallel sides) & trapeziform (TRAPEZI-form) - having the form of a trapezium, trapeziectomy (TRAPEZI-ectomy) - surgical removal of the trapezium \\
        OST-, OSTE- & `bone' & dysostosis (dys-OST-osis) - condition of abnormal bone, osteopenia (OSTE-O-penia) - deficiency of bone, osteolipochondroma (OSTE-O-LIP-O-CHONDR-oma) - tumor of the cartilage (made up of) fat and bone tissue \\
        TEND-, TENS-, TENT- & `to stretch' & extendable (ex-TEND-able) - able to be out stretched, tensor (TENS-or) - things that stretches, extensor (ex-TENS-or) - things that out stretches, distention (dis-TENT-ion) - condition of apart stretching \\
        TEN-, TEND-, TENDIN-, TENDON-, TENON-, TENONT- & `tendon' & tenodynia (TEN-odynia) - painful condition of a tendon, tendotomy (TEND-O-tomy) - surgical cutting of a tendon, tendinoplasty (TENDIN-O-plasty) - surgical reshaping of a tendon, tendonitis (TENDON-itis) - inflammation of a tendon, tenonectomy (TENON-ectomy) - surgical removal of a tendon, tenontology (TENONT-O-logy) - study of tendons \\
        LIG- & `to bind,' `to tie' & ligature (LIG-ature) - system composed of (something) to bind, ligate (LIG-ate) - to bind, ligation (LIG-ation) - process of applying ligature \\
        LIGAMENT- & `ligament' (bands of fibrous connective tissue that bind bones to other bones) & ligamentary (LIGAMENT-ary) - pertaining to ligament, osseoligamentous (OSSE-O-LIGAMENT-ous) - pertaining to ligament and bone \\
        DESM- & `band,' `ligament' & desmitis (DESM-itis) - inflammation of a ligament, desmopathy (DESM-O-pathy) - disease of the ligaments \\
        SYNDESM- & `ligament' & syndesmorrhaphy (SYNDESM-O-rrhaphy) - surgical suture of a ligament, syndesmotomy (SYNDESM-O-tomy) - surgical cutting of a ligament \\
        FASCI- & `band,' `fascia' & fasciation (FASCI-ation) - process of applying bands, fasciectomy (FASCI-ectomy) - surgical removal of (strips of) a fascia, fasciodesis (FASCI-O-desis) - surgical fusion of a fascia) \\
        MY-, MYS-, MYOS- & `muscle' & myalgia (MY-algia) - painful condition of a muscle, myogenesis (MY-O-genesis) - production of muscle, myopathy (MY-O-pathy) - disease of muscle, epimysial (epi-MYS-ial) - pertaining to on the surface of a muscle, myositis (MYOS-itis) - inflammation of a muscle \\
        MUSCUL- & `muscle' & intramuscular (intra-MUSCUL-ar) - pertaining to within the muscle, cervicomuscular (CERVIC-O-MUSCUL-ar) - pertaining to the muscles of the neck, musculature (MUSCUL-ature) - system composed of muscles \\
        FLECT-, FLEX- & `to bend' & reflection (re-FLECT-ion) - act of bending backward, flexion (FLEX-ion) - act of bending, flexor (FLEX-or) - thing that bends \\
        AGON- & `struggle,' `contest,' `action' & agonism (AGON-ism) - condition of action, antagonism (ant-AGON-ism) - condition of opposite action, agony (AGON-y) - state of struggle \\
        DUC-, DUCT- & `to lead,' `to draw' & abducent (ab-DUC-ent) - pertaining to away from (the midline) drawing, adduction (ad-DUCT-ion) - act of toward (the midline) drawing, circumduction (circum-DUCT-ion) - act of around drawing \\
        LAMIN- & `thin plate,' `lamina' & laminar (LAMIN-ar) - pertaining to a lamina, laminated (LAMIN-ated) - composed of laminae (thin plates), laminitis (LAMIN-itis) - inflammation of a lamina (thin plate) \\
        LAMELL- & `thin plate,' `lamella' (diminutive form of lamina) & lamellar (LAMELL-ar) - pertaining to a lamella \\
        ARTHR- & (i) `joint': (ii) `speech' & arthritis (ARTHR-itis) - inflammation of a joint, arthropathology (ARTHR-O-PATH-O-logy) - study of the diseases of joints, dysarthria (dys-ARTHR-ia) - condition of abnormal speech \\
        ARTICUL- & (i) `joint': (ii) `speech sound' & interarticular (inter-ARTICUL-ar) - pertaining to between joints, articulation (ARTICUL-ation) - process of being jointed, process of (forming) speech sounds, disaticulation (dis-ARTICUL-ation) - process of separate joint, biarticulate (BI-ARTICUL-ate) - having joints - two of them \\
        CRIST- & `crest' & cristate (CRIST-ate) - having a crest, intercristal (inter-CRIST-al) - pertaining to between crests \\
        PHY- & `to grow' & physis (PHY-sis) - process of growing, epiphysis (epi-PHY-sis) - condition of upon (the rest) growing, diaphysis (dia-PHY-sis) - condition of in a line growing \\
        PHYS- & (i) `growth,' `nature': (ii) `breath,' `inflation,' `swollen' & physics (PHYS-ics) - science of nature, physocephaly (PHYS-O-CEPHAL-y) - state of head (being) swollen \\
        SPIN- & `thorn,' `spine' & spinate (SPIN-ate) - having spines, spinalgia (SPIN-algia) - painful condition of the spine, infraspinous (infra-SPIN-ous) - pertaining to below a spine \\
        TOP- & `place,' `position' & topophobia (TOP-O-phobia) - abnormal fear of (certain) places, dystopic (dys-TOP-ic) - pertaining to abnormal positionm actopia (ec-TOP-ia) - condition of outside the (correct) position, adenectopy (ADEN-ec-TOP-y) - condition of outside the (correct) position gland \\
        TA-, TAS- & `to stretch,' `stretching,' `tension' & iridotasis (IRID-O-TA-sis) - condition of stretching the iris, entasia (en-TAS-ia) - condition of inward tension, ectasia (ec-TAS-ia) - condition of outward stretching \\
        EDE-, OEDE- & `to swell' & edesis (EDE-sis) - conditon of swelling, arthredema (ARTHE-edema) - swelling of the joints, lipedema (LIP-edema) - swelling (caused by subcutaneous) fat \\
        EME- & `to vomit' & antiemetic (anti-EME-tic) - pertaining to against the act of vomiting, hyperemesis (hyper-EME-sis) - condition of more than normal vomiting, dysemesia (dys-EME-sia) - condition of painful vomiting \\
        PHRAG- & `to block,' `to obstruct' & emphraxis (em-PHRAG-sis) - condition of inward blocking, laryngemphraxis (LARYNG-em-PHRAG-sis) - condition of inward blocking of the larynx, adenemphraxis (ADEN-em-PHRAG-sis) - condition of the inward blocking of a gland \\
        DIAPHRAGM- & `diaphragm' (a variety of thin partitions) & diaphragmatic (DIAPHRAGM-atic) - pertaining to a diaphragm, diaphragmalgia (DIAPHRAGM-algia) - painful condition of the diaphragm \\
        LEP-, LEPS- & `taking hold,' `to seize' & analeptic (ana-LEP-tic) - pertaining to an upward taking hold, psycholepsy (PSYCH-O-LEPS-y) - state of seizure of the mind, catalepsy (cata-LEPS-y) - state of complete seizure \\
        MALAC- & `soft' & malacodermous (MALAC-O-DERM-ous) - pertaining to skin that is soft, malacosarcosis (MALAC-O-SARC-osis) - condition of (muscular) tissue that is soft, osteomalacia (OSTE-O-MALAC-ia) - condition of softening of the bones, craniomalacia (CRANI-O-MALAC-ia) - condition of softening of the cranium \\
        PHTHI-, PHTHIS- & `to waste away,' `to decay' & ophthalmophthisis (OPHTHALM-O-PHTHI-sis) - condition of decay of the eye, laryngophthisis (LARING-O-PHTHI-sis) - condition of wasting in the larynx, phthisic (PHTHIS-ic) - pertaining to wasting \\
        PLEG- & `blow,' `stroke' & glossoplegia (GLOSS-O-PLEG-ia) - condition of stroke of the tongue, quadriplegic (QUADR-I-PLEG-ic) - pertaining to a stroke of the four (limbs), prosopediplegia (PROSOP-O-DI-PLEG-ia) - condition of stroke on two sides of the face \\
        PLEX-, PLEC- & `blow,' `stroke,' `seizure' & cataplexy (cata-PLEX-y) - condition of down stroke, phrenoplexia (PHREN-O-PLEX-ia) - condition of seizure of the mind, cataplectic (cata-PLECT-ic) - pertaining to down stroke \\
        PTO- & `falling' & proptosis (pro-PTO-sis) - condition of forward falling, blepharoptosis (BLEPHAR-O-PTO-sis) - condition of falling of the eyelid, glossoptosia (GLOSS-O-PTO-sia) - condition of falling of the tongue \\
        STA-, STAS- & `to stand,' `standing' & astasia (a-STA-sia) - condition of not standing, coprostasis (COPR-O-STA-sis) - condition of standing of the feces, stasiphobia (STAS-I-phobia) - abnormal fear of standing, isostasy (IS-O-STAS-y) - condition of standing equal \\
        STHEN- & `strength' & hypersthenia (hyper-STHEN-ia) - condition of more than normal strength, asthenia (a-STHEN-ia) - condition of without strength, anisosthenic (anis-O-STHEN-ic) - pertaining to unequal strength \\
    \label{tab:Ch10Base}
\end{longtable}




\subsection{Compound Suffixes}


\begin{longtable}{c | p{0.4\textwidth} | p{0.4\textwidth}}
    \caption{Compound suffixes for the shoulders and arms.}
    \hline
    Suffix & Meaning(s) & Example(s) \\ \hline
        -asthenia & `weakness' & \\
        -ectasia, -ectasis & `expansion,' `widening,' `dilation' & \\
        -ectopia, -ectopy & `displacement' & \\
        -emesia, -emesis & `vomiting' & \\
        -emphraxis & `obstruction' & \\
        -lepsis, -lepsy & `seizure' & epilepsy \\
        -malacia & `softening' & \\
        -(o)edema & `swelling' & \\
        -phthisis & `wasting' & \\
        -plegia & `paralysis' & paraplegia \\
        -plexia, -plexy & `seizure' & \\
        -ptosia, -ptosis & `drooping,' `prolapse' & \\
        -stasia, -stasis, -stasy & `stoppage,' `stagnation' & \\
        -stenosis & `narrowing,' `contraction' & \\
    \label{tab:Ch10Suffix}
\end{longtable}

\section{Questions and Remarks}
\label{sec:QR10}






%
% \begin{acknowledgement}
% If you want to include acknowledgments of assistance and the like at the end of an individual chapter please use the \verb|acknowledgement| environment -- it will automatically render Springer's preferred layout.
% \end{acknowledgement}
%
% \section*{Appendix}
% \addcontentsline{toc}{section}{Appendix}
%


% Problems or Exercises should be sorted chapterwise
\section*{Problems}
\addcontentsline{toc}{section}{Problems}
%
% Use the following environment.
% Don't forget to label each problem;
% the label is needed for the solutions' environment
\begin{prob}
\label{prob1}
A given problem or Excercise is described here. The
problem is described here. The problem is described here.
\end{prob}

% \begin{prob}
% \label{prob2}
% \textbf{Problem Heading}\\
% (a) The first part of the problem is described here.\\
% (b) The second part of the problem is described here.
% \end{prob}

%%%%%%%%%%%%%%%%%%%%%%%% referenc.tex %%%%%%%%%%%%%%%%%%%%%%%%%%%%%%
% sample references
% %
% Use this file as a template for your own input.
%
%%%%%%%%%%%%%%%%%%%%%%%% Springer-Verlag %%%%%%%%%%%%%%%%%%%%%%%%%%
%
% BibTeX users please use
% \bibliographystyle{}
% \bibliography{}
%


% \begin{thebibliography}{99.}%
% and use \bibitem to create references.
%
% Use the following syntax and markup for your references if 
% the subject of your book is from the field 
% "Mathematics, Physics, Statistics, Computer Science"
%
% Contribution 
% \bibitem{science-contrib} Broy, M.: Software engineering --- from auxiliary to key technologies. In: Broy, M., Dener, E. (eds.) Software Pioneers, pp. 10-13. Springer, Heidelberg (2002)
% %
% Online Document

% \end{thebibliography}

