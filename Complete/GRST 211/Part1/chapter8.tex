%%%%%%%%%%%%%%%%%%%%% chapter.tex %%%%%%%%%%%%%%%%%%%%%%%%%%%%%%%%%
%
% sample chapter
%
% Use this file as a template for your own input.
%
%%%%%%%%%%%%%%%%%%%%%%$% Springer-Verlag %%%%%%%%%%%%%%%%%%%%%%%%%%
%\motto{Use the template \emph{chapter.tex} to style the various elements of your chapter content.}
\chapter{Mouth}
\label{Mouth} % Always give a unique label
% use \chaptermark{}
% to alter or adjust the chapter heading in the running head


%%% Questions to think about
%The \textbf{Thesis} or general sense of the article is ...

%The \textbf{method} the author uses to argue their point is ...

%In their \textbf{analysis} the author uses tools such as ... 

%Additionally they conclude ...

%What connections does the author portray with regard to \textbf{space}, \textbf{relationships}, \textbf{occupation}, and \textbf{religion}.


\abstract{}


\section{Notes}
\label{sec:NOTE8}


\subsection{Prefixes}

\begin{longtable}{c | p{0.4\textwidth} | p{0.4\textwidth}}
    \caption{Prefixes for the Mouth.}
    \hline
    Prefix & Meaning(s) & Example(s) \\ \hline
        ab-, a- & `away from' & abnormal \\
        ad-, ac-, af- etc. & `toward,' `near' & \\
        retro- & `behind,' `backward' & retroactive \\
    \label{tab:Ch8Prefix}
\end{longtable}



\subsection{Suffixes}

\begin{longtable}{c | p{0.4\textwidth} | p{0.4\textwidth}}
    \caption{Suffixes for the mouth.}
    \hline
    Suffix & Meaning(s) & Example(s) \\ \hline
        -culum & `small' & curriculum \\
        -tion & `act of,' `process of' & action \\
        -ulum & `small' & \\
        -ulus & `small' & \\
        -uncle & `small & \\
    \label{tab:Ch8Suffix}
\end{longtable}


\subsection{Bases}

\begin{longtable}{c | p{0.4\textwidth} | p{0.4\textwidth}}
    \caption{Bases for the mouth.}
    \hline
    Base & Meaning(s) & Example(s) \\ \hline
        OR-, OS- & `mouth,' `opening' & suboral (sub-OR-al) - pertaining to below the mouth, orad (OR-ad) - toward the mouth, aborad (ab-OR-ad) - toward away from the mouth (i.e. in a direction away from the mouth), osculum (os-culum) - small opening (i.e. a pore) \\
        GEUS- & `to tast,' `sense of taste' & ageusia (a-GEUS-ia) - condition of without sense of taste, hypergeusesthesia (hyper-GEUS-ESTHES-ia) - condition of more than normal sensation to taste \\
        GUST- & `to taste' & gustation (GUST-ation) - process of tasting, gustatory (GUST-atory) - pertaining to taste \\
        GLOSS- & `tongue,' `language' & glossectomy (GLOSS-ectomy) - surgical removal of the tongue, glossodynia (GLOSS-odynia) - painful condition of the tongue, aglossostomia (a-GLOSS-O-STOM-ia) - condition of without mouth and tongue, xenoglossia (XEN-O-GLOSS-ia) - condition of (knowing) a language that is foreign (i.e. the supposed ability to communicate in an unlearned language) \\
        GLOTT- & (i) `tongue,' `language': (ii) `glottis' & hypoglottal (hypo-GLOTT-al) - pertaining to below the tongue \\
        LINGU- & `tongue,' `language' & lingual (LINGU-al) - pertaining to the tongue, pertaining to language, faciolingual (FACI-O-LINGU-al) - pertaining to the tongue and face, retrolingual (retro-LINGU-al) - pertaining to behind the tongue, pertaining to the backward part of the tongue, sublingual (sub-LINGU-al) - pertaining to underneath the tongue \\ 
        PAPILL- & `nipple,' `papilla' (any small nipple-like projection) & retinopapillitis (RETIN-O-PAPILL-itis) - inflammation of the (optic) papilla and the retina, papilliferous (PAPILL-I-FER-ous) - pertaining to bearing papillae (i.e. pertaining to having nipples, papules, or pimples), papillectomy (PAPILL-ectomy) - surgical removal of a papilla \\
        FREN- & `bridle,' `rein' & frenum (FREN-um) - structure of a rein \\
        FREN- & `frenum' (a connecting fold of a membrane that limits the movement between a fixed part and a movable part) & frenate (FREN-ate) - having a frenum, frenotomy (FREN-O-tomy) - surgical cutting of a frenum, frenulum (FREN-ulum) - small frenum \\
        CAR-, CARN- & `flesh,' `meat' & caruncle (CAR-uncle) - small (piece of) flesh, carnose (CARN-ose) - having the quality of flesh, carnophobia (CARN-O-phobia) - abnormal fear of meat \\
        BUCC- & `cheek' & intrabuccal (intra-BUCC-al) - pertaining to inside the cheek, bucconasal (BUCC-O-NAS-al) - pertaining to the nose and cheeks, buccolingual (BUCC-O-LINGU-al) - pertaining to the tongue and cheeks \\
        LABI-,LABR- & `lip' & labiate (LABI-ate) - having lips, labioplasty (LABI-O-plasty) - surgical reshaping of the lips, labiograph (LABI-O-graph) - instrument used to measure lip (movement during speaking), labral (LABR-al) - pertaining to a lip \\
        CHEIL- & `lip' & acheilia (a-CHEIL-ia) - condition of without lips, cheilectomy (CHEIL-ectomy) - surgical removal of a lip, cheilitis (CHEIL-itis) - inflammation of the lip \\
        STAPHYL- & `grape,' `bunch of grapes' & \\
        STAPHYL- & `uvula' & staphylotomy (STAPHYL-O-tomy) - surgical cutting of the uvula, staphyloschisis (STAPHYL-O-SCHI(S)-sis) - condition of a split uvula \\
        PALAT- & `roof of the mouth,' `palate' & palatine (PALAT-ine) - pertaining to the palate, palatoglossal (PALAT-O-GLOSS-al) - pertaining to the tongue and palate, palatoschisis (PALAT-O-SCHI(S)-sis) - condition of a split palate \\
        URAN- & `roof of the mouth,' `palate' & uranoplasty (URAN-O-plasty) - surgical reshaping of the palate, uranorrhaphy (URAN-O-rrhaphy) - surgical suture of the palate, uranoschisis (URAN-O-SCHI(S)-sis) - condition of split palate \\
        GNATH- & `jaw' & dysgnathic (dys-GNATH-ic) - pertaining to abnormal jaw (development), perignathic (per-GNATH-ic) - pertaining to around the jaw, prognathous (pro-GNATH-ous) - having a forward (jutting) jaw \\
        MAXILL- & `upper jaw bone,' `maxilla' & admaxillary (ad-MAXILL-ary) - pertaining to near the upper jaw bone \\
        hemimaxillectomy (hemi-MAXILL-ectomy) - surgical removal of half the upper jaw bone, inframaxillary (infra-MAXILL-ary) - pertaining to below the upper jaw bone \\
        MANDIBUL- & `lower jaw bone,' `mandible' & mandibular (MANDIBUL-ar) - pertaining to the lower jaw bone, mandibulectomy (MANDIBUL-ectomy) - surgical removal of the lower jaw bone, mandibulofacial (MANDIBUL-O-FACI-al) - pertaining to the facial and lower jaw bones \\
        CONDYL- & `knuckle,' `knob' & \\
        CONDYL- & `condyle' (the rounded bump at the end of a bone where it meets with another bone) & condylar (CONDYL-ar) - pertaining to a condyle, condylectomy (CONDYL-ectomy) - surgical removal of the condyle, condyllotomy (CONDYL-O-tomy) - surgical cutting of a condyle \\
        RAM- & `branch' & ramiform (RAM-I-form) - having the form of a branch \\
        RAM- & `ramus' (the individual divided parts, or branches, of a structure) & ramitis (RAM-itis) - inflammation of a ramus, ramulus (RAM-ulus) - small ramus \\
        GINGIV- & `gum' & labiogingival (LABI-O-GINGIV-al) - pertaining to the gums and libs, gingivalgia (GINGIV-algia) - painful condition of the gums, gingivoglossitis (GINGIV-O-GLOSS-itis) - inflammation of the tongue and gums \\
        DENT- & `tooth' & interdental (inter-DENT-al) - pertaining to between the teeth, denticle (DENT-I-cle) - small tooth, dedentition (de-DENT-I-tion) - process of (becoming) without teeth \\
        ODONT- & `tooth' & endodontics (end-ODONT-ics) - study of inside the teeth, odontonecrosis (ODONT-O-NECR-osis) - process of the death of a tooth \\
        SALIV- & `spit,' `saliva' & salivary (SALIV-ary) - pertaining to saliva, salivation & SALIV-ation & process of (producing) saliva \\
        SIAL- & `spit,' `saliva,' `salivary gland' & asialia (a-SIAL-ia) - condition of without saliva, sialic (SIAL-ic) - pertaining to saliva, sialostenosis (SIAL-O-STEN-osis) - abnormal condition of narrowing of the salivary gland \\
        PTY- & `spit,' `saliva' & ptysis (PTY-sis) - act of spitting \\
        PTYAL- & `saliva,' `salivary gland' & hyperptyalism (hyper-PTYAL-ism) - condition of excessive saliva, ptyalogenic (PTYAL-O-genic) - producing saliva, ptyalography (PTYAL-O-graphy) - process of recording salivary gland \\
        MULT- & `many,' `mcuh' & multicavous (MULT-I-CAV-ous) - full of many empty spaces, multidentate (MULT-I-DENT-ate) - having teeth - many of them multocular (MULT-OCUL-ar) - pertaining to eyes, many of them \\
        POLY- & `many,' `much' & polyblennia (POLY-BLENN-ia) - condition of excessive mucus, polycoria (POLY-COR-ia) - condition of having more than one pupil in the eye, polyhidria (POLY-HIDR-ia) - condition of excessive sweating \\
        HEMI- & `half' & hemiglossectomy (HEMI-GLOSS-ectomy) - surgical removal of half the tongue \\
        hemicephalgia (HEMI-CEPH-algia) - painful condition of half the head \\
        hemiopalgia (HEMI-OP-algia) - painful condition of one eye \\
        SEMI- & `half,' `partly' & semilenticular (SEMI-LENTICUL-ar) - having the character of the lens of the eye - (in one) half (i.e. a lens that is convex like the lens of the eye on one side only) \\
        OLIG- & `few,' `scanty' & olighidrosis (OLIG-HIDR-osis) - abnormal condition of sweating that is scanty, oligoptyalism (OLIG-O-PTYAL-ism) - condition of saliva that is scanty, oligotrichosis (OLIG-O-TRICH-osis) - abnormal condition of the hair that is scanty \\
        PLUR- & `many,' `more' & plurocular (PLUR-OCUL-ar) - pertainin to eyes - many of them \\
        PLEO-, PLEIO- & `more,' `excessive,' `multiple' & pleochroic (PLEO-CHRO-ic) - pertaining to colors that are multiple, pleocytosis (PLEO-CYT-osis) - abnormal condition of the cells that are excessive pleiomorphous (PLEIO-MORPH-ous) - pertaining to forms that are many \\
        PAN-,  PANT- & `all,' `entire' & pansclerosis (PAN-SCLER-osis) - abnormal condition of hardening of an entire (organ or part), panencephalitis (PAN-ENCEPHAL-itis) - inflammation of the brain - all of it, pantophobia (PANT-O-phobia) - abnormal fear of all things, pantaphobia (PANT-a-phobia) - abnormal non-fear of all things (i.e. total fearlessness) \\
        MACR- & `large,' `long' & marcoblepharia (MACR-O-BLEPHAR-ia) - condition of the eyelid that is (abnormally) large, macrocephalic (MACR-O-CEPHAL-ic) - pertaining to a head that is large, macrognathia (MACR-O-GNATH-ia) - condition of the jaw that is large \\
        MAGN- & `large,' `great' & magnisonant (MAGN-I-SON-ant) - pertaining to a sound that is large \\
        MICR- & `small' & microdontia (MICR-ODONT-ia) - condition of the teeth that are small, microsomia (MICR-O-SOM-ia) - condition of the body that is small, microscopic (MICR-O-SCOP-ic) - pertaining to viewing small \\
        MEGA-, MEGAL- & `large,' `great' & megaprosopia (MEGA-PROSOP-ia) - condition of the face that is large, megalgia (MEG(A)-algia) - painful condition (that is) large, rhinomegaly (RHIN-O-MEGAL-y) - state of a large nose \\
        PEN- & `deficiency,' `decrease' & penalgesia (PEN-algesia) - sensation of pain decrease, sarcopenia (SARC-O-PEN-ia) - condition of deficiency of flesh, lipopenia (LIP-O-PEN-ia) - condition of deficiency of fats, leukocytopenia (LEUK-O-CYT-O-PEN-ia) - condition of deficiency of cells that are white \\
        HOL- & `whole,' `entire' & holotrichous (HOL-O-TRICH-ous) - having hair, or hair-like structures, over the entire (body), holosomatric (HOL-O-SOMAT-ic) - pertaining to the body - the whole of it \\
        HOM-, HOME- & `same,' `similar' & homodontic (HOM-ODONT-ic) - pertaining to teeth that are all the same, homogensis (HOM-O-genesis) - production of (offspring) similar (to parents), homeopathy (HOME-O-pathy) - treatment of disease with similars \\
        IS- & `equal,' `same' & isochromatic (IS-O-CHROM-atic) - pertaining to a color that is the same, isocoria (IS-O-COR-ia) - condition of the pupils being the same (size), isothymia (IS-O-THYM-ia) - condition of emotion that is equal \\
        ANIS- & `unequal,' `different' & anisocoria (ANIS-O-COR-ia) - condition of the pupils being unequal, anisognathous (ANIS-O-GNATH-ous) - having jaws of unequal (size), anisopia (ANIS-OP-ia) - condition of sight that is different (in each eye) \\
        ALL- & `other,' `different' & allesthesia (ALL-ESTHE-sia) - condition of sensation at a different (place from the stimulus), allochorism (ALL-O-CHRO-ism) - condition of color being different \\
        HETER- & `other,' `different' & heterodontic (HETER-ODONT-ic) - pertaining to teeth that are different, heterotrichosis (HETER-O-TRICH-osis) - abnormal condition of hair that is different (colors) \\
    \label{tab:Ch8Base}
\end{longtable}


All the word parts presented here that indicate size and quantity are classified as BASES.

\subsection{Diminutive suffixes}

\begin{longtable}{c | p{0.4\textwidth} | p{0.4\textwidth}}
    \caption{Dimminutive suffixes.}
    \hline
    Suffix & Meaning(s) & Example(s) \\ \hline
        -culus, -cula, -culum & `small' & \\
        -ellus, -ella, -ellum & `small' & \\
        -illus, -illa, -illum & `small' & \\
        -ulus, -ula, -ulum & `small' & \\
        -cle & `small' & \\
        -idium & `small' & \\
        -il & `small' & \\
        -ium & `small' & \\
        -ole & `small' & \\
        -ule & `small' & \\
        -uncle & `small' & \\
        -unculus & `small' & \\
    \label{tab:Ch8Suffix2}
\end{longtable}


\subsection{Compound Suffixes}

\begin{longtable}{c | p{0.4\textwidth} | p{0.4\textwidth}}
    \caption{Compound suffixes for the mouth.}
    \hline
    Suffix & Meaning(s) & Example(s) \\ \hline
        -geusia & `condition of sense of taste' & \\
        -megaly & `enlargement' & \\
        -penia & `deficiency' & \\
        -schesis & `fissure' & \\
    \label{tab:Ch8Suffix3}
\end{longtable}

\section{Questions and Remarks}
\label{sec:QR8}






%
% \begin{acknowledgement}
% If you want to include acknowledgments of assistance and the like at the end of an individual chapter please use the \verb|acknowledgement| environment -- it will automatically render Springer's preferred layout.
% \end{acknowledgement}
%
% \section*{Appendix}
% \addcontentsline{toc}{section}{Appendix}
%


% Problems or Exercises should be sorted chapterwise
\section*{Problems}
\addcontentsline{toc}{section}{Problems}
%
% Use the following environment.
% Don't forget to label each problem;
% the label is needed for the solutions' environment
\begin{prob}
\label{prob1}
A given problem or Excercise is described here. The
problem is described here. The problem is described here.
\end{prob}

% \begin{prob}
% \label{prob2}
% \textbf{Problem Heading}\\
% (a) The first part of the problem is described here.\\
% (b) The second part of the problem is described here.
% \end{prob}

%%%%%%%%%%%%%%%%%%%%%%%% referenc.tex %%%%%%%%%%%%%%%%%%%%%%%%%%%%%%
% sample references
% %
% Use this file as a template for your own input.
%
%%%%%%%%%%%%%%%%%%%%%%%% Springer-Verlag %%%%%%%%%%%%%%%%%%%%%%%%%%
%
% BibTeX users please use
% \bibliographystyle{}
% \bibliography{}
%


% \begin{thebibliography}{99.}%
% and use \bibitem to create references.
%
% Use the following syntax and markup for your references if 
% the subject of your book is from the field 
% "Mathematics, Physics, Statistics, Computer Science"
%
% Contribution 
% \bibitem{science-contrib} Broy, M.: Software engineering --- from auxiliary to key technologies. In: Broy, M., Dener, E. (eds.) Software Pioneers, pp. 10-13. Springer, Heidelberg (2002)
% %
% Online Document

% \end{thebibliography}

