\documentclass[12pt, a4paper, oneside, openright, titlepage]{book}
\usepackage[utf8]{inputenc}
\raggedbottom
\usepackage{import}


%%%%%%%%%%%%%%%%% Book Formatting Comments:

%%%%%%%%%%%%%%%%%%%%%%%%%%%%%%%%%%%%% for Part

%%%%%%%%%%%%%%%%%%%%%% for chapter

%%%%%%%%%%%%%%%%%%%% for section




%%%%%% PACKAGES %%%%%%%
\usepackage{hyperref}
\hypersetup{
    colorlinks,
    citecolor=black,
    filecolor=black,
    linkcolor=black,
    urlcolor=black
}
\usepackage{amsmath} % Math display options
\usepackage{amssymb} % Math symbols
%\usepackage{amsfonts} % Math fonts
%\usepackage{amsthm}
\usepackage{mathtools} % General math tools
\usepackage{array} % Allows you to write arrays
\usepackage{empheq} % For boxing equations
% \usepackage{mathabx}
% \usepackage{mathrsfs}
\usepackage{nameref}
\usepackage{wrapfig}

\usepackage{soul}
\usepackage[normalem]{ulem}

\usepackage{txfonts}
\usepackage{cancel}
\usepackage[toc, page]{appendix}
\usepackage{titletoc,tocloft}
\setlength{\cftchapindent}{1em}
\setlength{\cftsecindent}{2em}
\setlength{\cftsubsecindent}{3em}
%\setlength{\cftsubsubsecindent}{4em}
\usepackage{titlesec}

%\titleformat{\section}
%  {\normalfont\fontsize{25}{15}\bfseries}{\thesection}%{1em}{}
%\titleformat{\section}
%  {\normalfont\fontsize{20}{15}\bfseries}%{\thesubsection}{1em}{}
%\setcounter{secnumdepth}{1}  
  
  

%\newcommand\numberthis{\refstepcounter{equation}\tag{\theequation}} % For equation labelling
\usepackage[framemethod=tikz]{mdframed}

\usepackage{tikz} % For drawing commutative diagrams
\usetikzlibrary{cd}
\usetikzlibrary{calc}
\tikzset{every picture/.style={line width=0.75pt}} %set default line width to 0.75p

\usepackage{datetime}
\usepackage[margin=2cm]{geometry}
\setlength{\parskip}{1em}
\usepackage{makeidx}         % allows index generation
\usepackage{graphicx}       % standard LaTeX graphics tool
\usepackage{multicol}        % used for the two-column index
\usepackage[bottom]{footmisc}% places footnotes at page bottom

\usepackage{newtxtext}       % 
\usepackage{newtxmath}       % selects Times Roman as basic font
\usepackage{float}
\usepackage{fancyhdr}
\setlength{\headheight}{15pt} 
\pagestyle{fancy}
\lhead[\leftmark]{}
\rhead[]{\leftmark}

%\usepackage{enumitem}

\usepackage{url}
\allowdisplaybreaks

%%%%%% ENVIRONMENTS %%%
\definecolor{purp}{rgb}{0.29, 0, 0.51}
\definecolor{bloo}{rgb}{0, 0.13, 0.80}



%%\newtheoremstyle{note}% hnamei
%{3pt}% hSpace above
%{3pt}% hSpace belowi
%{}% hBody fonti
%{}% hIndent amounti
%{\itshape}% hTheorem head fonti
%{:}% hPunctuation after theorem headi
%{.5em}% hSpace after theorem headi
%{}% hTheorem head spec (can be left empty, meaning ‘normal’)i





% %%%%%%%%%%%%% THEOREM DEFINITIONS

\spnewtheorem{axiom}{Axiom}[chapter]{\bfseries}{\itshape}

\spnewtheorem{construction}{Construction}[chapter]{\bfseries}{\itshape}

\spnewtheorem{props}{Properties}[chapter]{\bfseries}{\itshape}


\renewcommand{\qedsymbol}{$\blacksquare$}


\numberwithin{equation}{section}

\newenvironment{qest}{
    \begin{center}
        \em
    }
    {
    \end{center}
    }

%%%%%% MACROS %%%%%%%%%
%% New Commands
\newcommand{\ip}[1]{\left\langle#1\right\rangle} %%% Inner product
\newcommand{\abs}[1]{\lvert#1\rvert} %%% Modulus
\newcommand\diag{\operatorname{diag}} %%% diag matrix
\newcommand\tr{\mbox{tr}\.} %%% trace
\newcommand\C{\mathbb C} %%% Complex numbers
\newcommand\R{\mathbb R} %%% Real numbers
\newcommand\Z{\mathbb Z} %%% Integers
\newcommand\Q{\mathbb Q} %%% Rationals
\newcommand\N{\mathbb N} %%% Naturals
\newcommand\F{\mathbb F} %%% An arbitrary field
\newcommand\ste{\operatorname{St}} %%% Steinberg Representation
\newcommand\GL{\mathbf{GL}} %%% General Linear group
\newcommand\SL{\mathbf{SL}} %%% Special linear group
\newcommand\gl{\mathfrak{gl}} %%% General linear algebra
\newcommand\G{\mathbf{G}} %%% connected reductive group
\newcommand\g{\mathfrak{g}} %%% Lie algebra of G
\newcommand\Hbf{\mathbf{H}} %%% Theta fixed points of G
\newcommand\X{\mathbf{X}} %%% Symmetric space X
\newcommand{\catname}[1]{\normalfont\textbf{#1}}
\newcommand{\Set}{\catname{Set}} %%% Category set
\newcommand{\Grp}{\catname{Grp}} %%% Category group
\newcommand{\Rmod}{\catname{R-Mod}} %%% Category r-modules
\newcommand{\Mon}{\catname{Mon}} %%% Category monoid
\newcommand{\Ring}{\catname{Ring}} %%% Category ring
\newcommand{\Topp}{\catname{Top}} %%% Category Topological spaces
\newcommand{\Vect}{\catname{Vect}_{k}} %%% category vector spaces'
\newcommand\Hom{\mathbf{Hom}} %%% Arrows

\newcommand{\map}[2]{\begin{array}{c} #1 \\ #2 \end{array}}
\newcommand{\inner}[2]{\left\langle #1, #2\right\rangle}
\newcommand{\norm}[1]{\left|\left| #1\right|\right|}

\newcommand{\Emph}[1]{\textbf{\ul{\emph{#1}}}}




%% Math operators
\DeclareMathOperator{\ran}{Im} %%% image
\DeclareMathOperator{\aut}{Aut} %%% Automorphisms
\DeclareMathOperator{\spn}{span} %%% span
\DeclareMathOperator{\ann}{Ann} %%% annihilator
\DeclareMathOperator{\rank}{rank} %%% Rank
\DeclareMathOperator{\ch}{char} %%% characteristic
\DeclareMathOperator{\ev}{\bf{ev}} %%% evaluation
\DeclareMathOperator{\sgn}{sign} %%% sign
\DeclareMathOperator{\id}{Id} %%% identity
\DeclareMathOperator{\supp}{supp} %%% support
\DeclareMathOperator{\inn}{Inn} %%% Inner aut
\DeclareMathOperator{\en}{End} %%% Endomorphisms
\DeclareMathOperator{\sym}{Sym} %%% Group of symmetries


%% Diagram Environments
\iffalse
\begin{center}
    \begin{tikzpicture}[baseline= (a).base]
        \node[scale=1] (a) at (0,0){
          \begin{tikzcd}
           
          \end{tikzcd}
        };
    \end{tikzpicture}
\end{center}
\fi

\usepackage[most]{tcolorbox}
%%%%%%%%%%%%%% ENVIRONMENTS
%%% tcolorbox environments

%%% \tcolorboxenvironment: An existing environment is redefined to be allowed to be boxed inside another tcolorbox during use
\tcolorboxenvironment{proof}{% `proof' from `amsthm'  
blanker,breakable=true,left=0.3em,
borderline west={3pt}{-3pt}{black!35}}


%%% titled environments
\newtcolorbox[auto counter, number within=section]{nthm}[2][]{
colback=red!5!white,
colframe=red!5!white,
attach title to upper={},
title={\textbf{Theorem \thetcbcounter\  (#2)\ }},
coltitle={black},
sharp corners,
breakable=true,
enhanced jigsaw,
borderline west = {3pt}{-3pt}{red!75!black},
#1
}

\newtcolorbox[use counter from=nthm, number within=section]{ndefn}[2][]{
colback=blue!5!white,
colframe=blue!5!white,
attach title to upper={},
title={\textbf{Definition \thetcbcounter\  (#2)\ }},
coltitle={black},
sharp corners,
breakable=true,
enhanced jigsaw,
borderline west = {3pt}{-3pt}{blue!75!black},
#1
}

\newtcolorbox[auto counter, number within=section]{eg}[1][]{
colback=white,
attach title to upper={},
title={\underline{Example:}\smallbreak},
coltitle={green!50!black},
sharp corners,
breakable=true,
enhanced jigsaw,
frame hidden, 
borderline west = {3pt}{-3pt}{green!50!black},
#1
}

\newtcolorbox[auto counter, number within=section]{qst}[1][]{
colback=white,
attach title to upper={},
title={\underline{Question?}\smallbreak},
coltitle={blue!50!black},
sharp corners,
breakable=true,
enhanced jigsaw,
frame hidden, 
borderline west = {3pt}{-3pt}{blue!50!black},
#1
}

\newtcolorbox[auto counter, number within=section]{rmk}[1][]{
colback=white,
attach title to upper={},
title={\underline{Remark:}\smallbreak},
coltitle={red!50!black},
sharp corners,
breakable=true,
enhanced jigsaw,
frame hidden, 
borderline west = {3pt}{-3pt}{red!50!black},
#1
}


\newtcolorbox[auto counter, number within=section]{nte}[1][]{
colback=white,
attach title to upper={},
title={\underline{Note:}\smallbreak},
coltitle={red!25!white},
sharp corners,
breakable=true,
enhanced jigsaw,
frame hidden, 
borderline west = {3pt}{-3pt}{red!25!white},
#1
}


%%% untitled environments
\newtcolorbox[use counter from=nthm, number within=section]{thm}[1][]{
colback=red!5!white,
colframe=red!5!white,
attach title to upper={},
title={\textbf{Theorem \thetcbcounter\ }},
coltitle={black},
sharp corners,
breakable=true,
enhanced jigsaw,
borderline west = {3pt}{-3pt}{red!75!black},
#1
} 

\newtcolorbox[use counter from=nthm, number within=section]{defn}[1][]{
colback=blue!5!white,
colframe=blue!5!white,
attach title to upper={},
title={\textbf{Definition \thetcbcounter\ }},
coltitle={black},
sharp corners,
breakable=true,
enhanced jigsaw,
borderline west = {3pt}{-3pt}{blue!75!black},
#1
}

\newtcolorbox[use counter from=nthm, number within=section]{cor}[1][]{
colback=red!5!white,
colframe=red!5!white,
attach title to upper={},
title={\textbf{Corollary \thetcbcounter\ }},
coltitle={black},
sharp corners,
breakable=true,
enhanced jigsaw,
borderline west = {3pt}{-3pt}{red!50!black},
#1
}

\newtcolorbox[use counter from=nthm, number within=section]{prop}[1][]{
colback=orange!5!white,
colframe=orange!5!white,
attach title to upper={},
title={\textbf{Proposition \thetcbcounter} \ },
coltitle={black},
sharp corners,
breakable=true,
enhanced jigsaw,
borderline west = {3pt}{-3pt}{orange!75!black},
#1
}

\newtcolorbox[use counter from=nthm, number within=section]{lem}[1][]{
colback=orange!5!white,
colframe=orange!5!white,
attach title to upper={},
title={\textbf{Lemma \thetcbcounter} \ },
coltitle={black},
sharp corners,
breakable=true,
enhanced jigsaw,
frame hidden, 
borderline west = {3pt}{-3pt}{orange!75!black},
#1
}

\newtcolorbox[auto counter, number within=section]{home}[1][]{
colback=white,
attach title to upper={},
title={\textbf{Exercise:}\ \ },
coltitle={black},
sharp corners,
breakable=true,
enhanced jigsaw,
frame hidden, 
borderline west = {3pt}{-3pt}{black!10!lime},
#1
}


\newdateformat{monthdayyeardate}{%
    \monthname[\THEMONTH]~\THEDAY, \THEYEAR}
%%%%%%%%%%%%%%%%%%%%%%%


%%% Specific Macros %%%


%%%%%% BEGIN %%%%%%%%%%


\begin{document}

%%%%%% TITLE PAGE %%%%%

\begin{titlepage}
    \centering
    \scshape
    \vspace*{\baselineskip}
    \rule{\textwidth}{1.6pt}\vspace*{-\baselineskip}\vspace*{2pt}
    \rule{\textwidth}{0.4pt}
    
    \vspace{0.75\baselineskip}
    
    {\LARGE Statistical Mechanics: A Complete Guide}
    
    \vspace{0.75\baselineskip}
    
    \rule{\textwidth}{0.4pt}\vspace*{-\baselineskip}\vspace{3.2pt}
    \rule{\textwidth}{1.6pt}
    
    \vspace{2\baselineskip}
    Phys 449 \\
    \vspace*{3\baselineskip}
    \monthdayyeardate\today \\
    \vspace*{5.0\baselineskip}
    
    {\scshape\Large Elijah Thompson, \\ Physics and Math Honors\\}
    
    \vspace{1.0\baselineskip}
    \textit{Solo Pursuit of Learning}
    \vfill
    \enlargethispage{1in}
    \begin{figure}[b!]
    \makebox[\textwidth]{\includegraphics[width=\paperwidth, height =10cm]{../Crab.jpg}}
    \end{figure}
\end{titlepage}

%%%%%%%%%%%%%%%%%%%%%%%
\tableofcontents



%%%%%%%%%%%%%%%%%%%%%%%%%%%%%%%%%%%%% Part 1
\part{Thermodynamics}

%%%%%%%%%%%%%%%%%%%%%% Chapter 1.1
\chapter{Energy in Thermal Physics}

%%%%%%%%%%%%%%%%%%%% Section 1.1.1
\section{Basic Notation and Work}

\begin{defn}
    Thermodynamics is a \Emph{phenomenological} description of properties of \Emph{macroscopic systems} in \Emph{thermal equilibrium}.
\end{defn}

By a phenomenological description we mean a discription based on observations and direct experience of the experimenter with the system, considered as a ``black box" (a system whose internal structure is unknown, or is just not considered). The task of thermodynamics is to define appropriate physical quanitities, \Emph{state quantities}, which characterize macroscopic properties of matter, that is \Emph{macrostates}, in a way which is as unambiguous as possible, and to relate these quantities by means of universally valid equations.

\begin{defn}[Systems]
    There are a number of different thermodynamic systems. In particular we define a thermodynamic system, in general, to be an arbitrary amount of matter, the properties of which can be uniquely and completely described by specifying certain macroscopic parameters. We summarize them as follows: \begin{itemize}
        \item \Emph{Thermodynamic or Macroscopic System:} A system consisting of a large number of constituents. For example, a mole of gas (approximately $10^{23}$ particles) can be considered as a macroscopic system.
        \item \Emph{Isolated Thermodynamic System:} A system which exhibits no exchange of any type with the surroundings; no exchange of work, heat, matter, etc. The total energy (mechanical, electrical, etc.) is a conserved quantity for such a system, and can thus be used to characterize the macrostate.
        \item \Emph{Closed Theormodynamic System:} A system which exhibits no exchange of matter with its surroundings. Hence, energy exchange is allowed, so energy is no longer a conserved quantity and can fluctuate due to exchange with the surroundings. The termperature, particle number, and volume of the system can characterize the macrostate.
        \item \Emph{Open Thermodynamic System:} A system for which it is possible for exchange of any type with the surroundings (work, heat, matter, etc.). Energy and particle number are both not conserved. But, temperature and chemical potential can still be used to characterize a macrostate.
    \end{itemize}
\end{defn}

If the properties of a system are the same for any part of it, one calls such a system \Emph{homogeneous}. On the other hand, if the properties change discontinuously at certain marginal surfaces, the system is \Emph{heterogeneous}, with the homogeneous portions of the system called \Emph{phases} and the separating surfaces \Emph{phase boundaries}. 

\begin{defn}
    The macroscopic quantities which describe a system are called state quantities.
\end{defn}

\begin{eg}
    Examples of state quantities are the system's energy $E$, its volume $V$, its particle number $N$, its entropy $S$, its temperature $T$, the pressure $P$, the chemical potential $\mu$, the charge $q$, the dipole momentum, the refractive index, the viscosity, the chemical composition, and the size of phase boundaries.
\end{eg}

However, microscopic quantities do not fall under the umbrella of state quantities.

\begin{defn}[Equilibrium]
    Two thermodynamic systems are in \Emph{equilibrium} if and only if they are in contact such that they can exchange a given conserved quantity (for example particles) and they are relaxed to a state in which there is no average net transfer of that quantity between them anymore. A thermodynamic system $S$ is in equilibrium with itself, if and only if, all its subsystems are in equilibrium with each other. In this case, $S$ is called an \Emph{equilibrium} system.
\end{defn}

From this definition we find that different types of exchanged quanitities can lead to different types of equilibria: 

\begin{table}[H]
    \centering
    \begin{tabular}{c|c}
        \hline
        Exchanged Quantity & Type of Equilibrium \\ \hline \hline
        Particles/Matter & Diffusive Equilibrium \\ 
        Work & Mechanical Equilibrium \\
        Heat & Thermal Equilibrium \\ \hline
    \end{tabular}
\end{table}

\begin{defn}
    \Emph{Complete Thermodynamic Equilibrium} corresponds to a state where all the conserved fluxes between two coupled thermodynamic systems vanish.
\end{defn}

A way of testing if a given system is in a complete thermodynamic equilibrium is if the properties of the system do not change appreciably over the observation time, which is to say the properties reflect the true asymptotic long-term properties after any initial relaxation time is over.

The state of thermodynamic systems in complete thermodynamic equilibrium can be described by a set of independent \Emph{thermodynamic coordinates} or \Emph{state variables}; this fact is based on empirical observations. 

\subsection{Boyle-Mariotte's Experiment}

\begin{law}
    For a given mass of gas at a constant temperature $T$, the volume $V$ is inverseley proportional to the pressure $P$: $V \propto P^{-1}$.
\end{law}

We consider Robert Boyle's (1627-1691) experiment, conducted at room temperature, which should remain constant during the time of the experiment. We use a vertical tube with markings to indicate the volume of gas, and oil at the bottom. By applying pressure on the oil, we also exert a pressure on the air in the tube above the oil, causing the volume to decrease. Drawing the graph of volume versus $P^{-1}$ we obtain a straight line, so $V\cdot P = C$ for some constant $C$, given a fixed temperature. As further measurements show, the constant is proportional to the temperature $T$, so $V\cdot P \propto T$. More precisely, the full equation of a state of simple low-density gase is given by \begin{equation}
    \boxed{PV = Nk_BT}
\end{equation}
where $N$ is the number of gas particles and $k_B \approx 1.381\times 10^{-23}\;J/K$ is a natural constant called the \Emph{Boltzmann's constant}. The above equation is called the \Emph{ideal gas law} or the \Emph{equation of state for an ideal gas} since it relates the three state variables, or thermodynamic coordinates, $P$, $V$, and $T$ at equilibrium.





\subsection{Work}

\begin{defn}
    Given a vector force field $\vec{F}$ defined along a path $\gamma$, the \Emph{work} $W$ is of $\vec{F}$ along $\gamma$ is defined to be: \begin{equation}
        W = \int_{\gamma}\vec{F}\cdot d\vec{r}
    \end{equation}
\end{defn}

\begin{rec}
    Recall that pressure is force per unit area, so we have that \begin{equation*}
        P = \frac{F}{A}
    \end{equation*}
    where $F$ is the normal component of the force to the area $A$. More precisely, pressure is the proportionality constant that relates the force and normal vectors: \begin{equation*}
        d\vec{F}_n = -pd\vec{A}
    \end{equation*}
    where $\vec{F}_n$ denotes the normal component of the force vector $\vec{F}$.
\end{rec}


\begin{eg}
    Consider the compression of a gas by a piston with a constant force of magnitude $F$.
    \begin{center}
    \begin{tikzpicture}[x=0.75pt,y=0.75pt,yscale=-1.2,xscale=1.2]
%uncomment if require: \path (0,398); %set diagram left start at 0, and has height of 398

%Shape: Rectangle [id:dp9125760638904235] 
\draw   (100,110) -- (200,110) -- (200,160) -- (100,160) -- cycle ;
%Straight Lines [id:da19743423815417427] 
\draw    (200,110) -- (230,110) ;
%Straight Lines [id:da17599416360044806] 
\draw    (200,160) -- (230,160) ;
%Straight Lines [id:da37025754978607806] 
\draw [color={rgb, 255:red, 255; green, 27; blue, 0 }  ,draw opacity=1 ]   (200,110) -- (200,160) ;
%Straight Lines [id:da5752016333947421] 
\draw [color={rgb, 255:red, 255; green, 27; blue, 0 }  ,draw opacity=1 ]   (200,135) -- (250,135) ;
%Straight Lines [id:da4588247522685671] 
\draw    (230,127) -- (222,127) ;
\draw [shift={(220,127)}, rotate = 360] [color={rgb, 255:red, 0; green, 0; blue, 0 }  ][line width=0.75]    (4.37,-1.32) .. controls (2.78,-0.56) and (1.32,-0.12) .. (0,0) .. controls (1.32,0.12) and (2.78,0.56) .. (4.37,1.32)   ;
%Curve Lines [id:da22459360243367987] 
\draw [line width=0.75]    (281.89,123.39) .. controls (280.27,107.22) and (224.4,110.81) .. (204.92,121.69) ;
\draw [shift={(203.22,122.72)}, rotate = 326.56] [color={rgb, 255:red, 0; green, 0; blue, 0 }  ][line width=0.75]    (6.56,-1.97) .. controls (4.17,-0.84) and (1.99,-0.18) .. (0,0) .. controls (1.99,0.18) and (4.17,0.84) .. (6.56,1.97)   ;
%Straight Lines [id:da04861500853364609] 
\draw    (200,170) -- (192,170) ;
\draw [shift={(190,170)}, rotate = 360] [color={rgb, 255:red, 0; green, 0; blue, 0 }  ][line width=0.75]    (4.37,-1.32) .. controls (2.78,-0.56) and (1.32,-0.12) .. (0,0) .. controls (1.32,0.12) and (2.78,0.56) .. (4.37,1.32)   ;
%Straight Lines [id:da2692578781738837] 
\draw    (200,167) -- (200,173) ;

% Text Node
\draw (231,122.4) node [anchor=north west][inner sep=0.75pt]  [font=\tiny,color={rgb, 255:red, 246; green, 12; blue, 12 }  ,opacity=1 ]  {$\vec{F}$};
% Text Node
\draw (237,125) node [anchor=north west][inner sep=0.75pt]  [font=\tiny] [align=left] {(const.)};
% Text Node
\draw (270.89,125.89) node [anchor=north west][inner sep=0.75pt]  [font=\tiny] [align=left] {Piston area:};
% Text Node
\draw (311.89,125.4) node [anchor=north west][inner sep=0.75pt]  [font=\tiny,color={rgb, 255:red, 255; green, 0; blue, 0 }  ,opacity=1 ]  {$\mathcal{A}$};
% Text Node
\draw (108.56,114.4) node [anchor=north west][inner sep=0.75pt]  [font=\small]  {$\cdot $};
% Text Node
\draw (128.56,134.4) node [anchor=north west][inner sep=0.75pt]  [font=\small]  {$\cdot $};
% Text Node
\draw (128.56,134.4) node [anchor=north west][inner sep=0.75pt]  [font=\small]  {$\cdot $};
% Text Node
\draw (139.89,121.07) node [anchor=north west][inner sep=0.75pt]  [font=\small]  {$\cdot $};
% Text Node
\draw (159.89,141.07) node [anchor=north west][inner sep=0.75pt]  [font=\small]  {$\cdot $};
% Text Node
\draw (127.56,112.07) node [anchor=north west][inner sep=0.75pt]  [font=\small]  {$\cdot $};
% Text Node
\draw (161.56,117.4) node [anchor=north west][inner sep=0.75pt]  [font=\small]  {$\cdot $};
% Text Node
\draw (181.56,137.4) node [anchor=north west][inner sep=0.75pt]  [font=\small]  {$\cdot $};
% Text Node
\draw (110.22,132.4) node [anchor=north west][inner sep=0.75pt]  [font=\small]  {$\cdot $};
% Text Node
\draw (179.22,114.73) node [anchor=north west][inner sep=0.75pt]  [font=\small]  {$\cdot $};
% Text Node
\draw (142.89,141.4) node [anchor=north west][inner sep=0.75pt]  [font=\small]  {$\cdot $};
% Text Node
\draw (168.22,127.73) node [anchor=north west][inner sep=0.75pt]  [font=\small]  {$\cdot $};
% Text Node
\draw (148.22,112.73) node [anchor=north west][inner sep=0.75pt]  [font=\small]  {$\cdot $};
% Text Node
\draw (123.56,124.07) node [anchor=north west][inner sep=0.75pt]  [font=\small]  {$\cdot $};
% Text Node
\draw (111.56,142.73) node [anchor=north west][inner sep=0.75pt]  [font=\small]  {$\cdot $};
% Text Node
\draw (192,173.4) node [anchor=north west][inner sep=0.75pt]  [font=\tiny]  {$\Delta x$};


\end{tikzpicture}
\end{center}
    so the work is \begin{equation*}
        W = \int_{\gamma}\vec{F}\cdot d\vec{r} = F\int_{\gamma}dr = F\Delta x = PA\Delta x = -P\Delta V
    \end{equation*}
    where $P$ is the pressure and $\Delta V$ is the change in volume. Since the change in volume is negative, the work $W$ done on the system (here the gas) is positive: Energy is added to the system by a force (macroscopic) process (here the piston). $W< 0$ if energy is removed from the system.
\end{eg}


\begin{defn}
    The generalized differential form for work is given by \begin{equation}
        \delta W = \sum_{i=1}^mJ_idq_i
    \end{equation}
    where $q_i$ are the generalized coordinates and the $J_i$ are the conjugate generalized forces such that $J_idq_i$ has units of energy.
\end{defn}

Here are a few examples of generalized coordinates and their corresponding generalized conjugate forces:

\begin{table}[H]
    \centering
    \begin{tabular}{c|c|c}
        \hline
        & Generalized Force $J_i$ & Generalized Coordinate $dq_i$ \\ \hline \hline
        Pressure & $-P$ & $dV$ (change in volume) \\ 
        Surface tension & $\sigma$ & $dS$ (change in surface area) \\
        Magnetic field & $\vec{B}_0$ & $d\vec{m}$ (change in magnetic moment) \\
        Electric field & $\vec{E}$ & $d\vec{P}$ (change in electric dipole moment) \\\hline
    \end{tabular}
\end{table}

\begin{defn}
    Differential changes in a systems property are said to be \Emph{quasi-statis} if the changes occur on time scales much longer than the relaxation time such that the system remains in equilibrium at all times.
\end{defn}

To ensure thermodynamics is a self-consistent description of macroscopic systems in thermal equilibrium we must assume that the differential changes $\delta W$ are quasi-static. This also ensures that we can describe the state of such a thermodynamic system by a set of thermodynamic coordinates at all times.

Depending on the macroscopic generalized force, the work necessary to transfer a thermodynamic system in complete thermodynamic equilibrium from state $A$ to state $B$ might or might not depend on the path taken. For example, if $\gamma_1$ and $\gamma_2$ are two paths from state $A$ to state $B$, it is possible that $\int_{\gamma_1}\delta W \neq \int_{\gamma_2}\delta W$. (Note that this justifies the use of the notation $\delta W$ over $dW$, which would denote an exact differential)

\subsection{Conservative Forces}

\begin{rec}
    Recall that a conservative force is a force $\vec{F}$ with an associated potential energy function $E$ such that $\vec{F} = \nabla E$.
\end{rec}

\begin{prop}
    If the macroscopic generalized force $\vec{J}$, depending on $m$ generalized coordinates $q_i$, is conservative with potential energy $E_{pot}$ which is twice continuously differentiable and the domain of integration is simply connected, then the following are equivalent:
    \begin{itemize}
        \item $\vec{J}(\vec{q}) = -\nabla E_{pot}(\vec{q})$
        \item $dW$ is an exact differential form, so $$\int_{\gamma}dW = \int_{\gamma}\vec{J}\cdot d\vec{r}$$ depends only on the endpoints of $\gamma$.
        \item For any simple closed path $\gamma$, $$\oint_{\gamma}dW = \oint_{\gamma}\vec{J}\cdot d\vec{r} = 0$$
        \item The curl of $\vec{J}$ is trivial over the domain of integration $$\nabla \times \vec{J} = 0$$ provided that $m = 3$
        \item For all $i,j \in \{1,2,...,m\}$, we have that \begin{equation*}
                \frac{\partial J_i}{\partial q_j} - \frac{\partial J_j}{\partial q_i} = 0
        \end{equation*}
        \item The differential $dW$ is exact, and we have that \begin{equation*}
                dW = -\nabla E_{pot}\cdot d\vec{r} = -\sum_{i=1}^m\frac{\partial E_{pot}}{\partial q_i}dq_i = -dE_{pot}
        \end{equation*}
    \end{itemize}
\end{prop}


\begin{defn}
    For a function $A(q_1,q_2,...,q_m)$ the \Emph{total differential} or \Emph{exact differential} of $A$ is given by \begin{equation*}
        dA = \sum_{i=1}^m\frac{\partial A}{\partial q_i}dq_i
    \end{equation*}
    This corresponds to a generalized chain rule: \begin{equation*}
        \frac{dA}{dt} = \nabla A\cdot \frac{d}{dt}\vec{q} = \sum_{i=1}^m\frac{\partial A}{\partial q_i}\frac{dq_i}{dt}
    \end{equation*}
    called the \Emph{total derivative} of $A$ with respect to $t$. An exact differential corresponds to an \Emph{integrable differential form}: \begin{equation*}
        \int_{\gamma}\sum_{i=1}^m\frac{\partial A}{\partial q_i}dq_i = \int_{\gamma}dA = A(\vec{q}_f) - A(\vec{q}_i)
    \end{equation*}
    where $\vec{q}_i$ and $\vec{q}_f$ are the initial and final point of the path $\gamma$, respectively.
\end{defn}

Consequently, an integrable, or exact, differential form does not depend on the path taken and in physics we would consider $A$ to be a potential function. 

\begin{note}
    For non-conservative forces, $\delta W$ is an \Emph{inexact differential form}.
\end{note}

\begin{thm}
    A differential form $$\delta A \equiv \sum_{i=1}^ma_i(q_1,q_2,...,q_m)dq_i$$ for functions $a_i:D \subseteq \R^m\rightarrow \R$ is exact if and only if \begin{equation*}
        \frac{\partial a_i}{\partial q_j} = \frac{\partial a_j}{\partial q_i}
    \end{equation*}
    for all $i,j \in \{1,2,...,m\}$.
\end{thm}

\begin{defn}
    An \Emph{integrating factor} $\mu$ is a factor that makes an inexact differential form exact upon multiplication.
\end{defn}

\begin{thm}
    For $m = 2$, an integrating factor always exists. Specifically, for $\delta A = a_1dx_1+a_2dx_2$, we can define $df:= \mu \delta A = (\mu a_1)dx_1 + (\mu a_2)dx_2$, with $\mu$ determined non-uniquely by the equation \begin{equation*}
        \frac{\partial (\mu a_1)}{\partial x_2} = \frac{\partial (\mu a_2)}{\partial x_1}
    \end{equation*}
\end{thm}




%%%%%%%%%%%%%%%%%%%% Section 1.1.2
\section{Heat and the 1st Law of Thermodynamics}

\begin{defn}
    Recall that work corresponds to the change in energy of a thermodynamic system by a \Emph{macroscopically forced process}. On the other hand the \Emph{heat} $Q$ is the energy added to or removed from a thermodynamic system by a \Emph{spontaneous process}.
\end{defn}

For example, consider the energy transfer between a cooking plate and a pot of water. The origin of this type of process lies in the underlying microscopic dynamics, which we will explore using Statistical Mechanics.

\begin{law}[First Law of Thermodynamics]
    For an isolated thermodynamic system, the total \Emph{internal energy} $U$ is a constant and $dU = 0$. By the definition of internal energy, $dU$ is an exact differential; consequently, $U$ should be unique for a given state. For various systems we have the following: \begin{itemize}
        \item For a closed system, $dU = \delta Q + \delta W$. 
        \item For an open system, $dU = \delta Q + \delta W + \delta E_c$, with $$\delta E_c := \sum_{i=1}^{\alpha}\mu_idN_i$$ where $N_i$ is the number of particles of type $i$ and $\mu_i$ is the \Emph{chemical potential} associated with particles of type $i$ for all $1 \leq i \leq \alpha$ with $\alpha$ being the number of different particle types. The chemical potential corresponds to the energy needed to add a particle of type $i$ to a given thermodynamic system while no other type of energy is exchanged, which is to say $\delta W = 0$ and $\delta Q = 0$.
    \end{itemize}
\end{law}

In summary, the empirical first law is a reformulation of the conservation of energy and requires the inclusion of heat. 



%%%%%%%%%%%%%%%%%%%%%% Chapter 1.2
\chapter{Thermodynamical Systems}



%%%%%%%%%%%%%%%%%%%%%% Chapter 1.3
\chapter{Thermodynamical Potentials and Equilibrium}




%%%%%%%%%%%%%%%%%%%%%%%%%%%%%%%%%%%%% Part 2
\part{Statistical Mechanics}


%%%%%%%%%%%%%%%%%%%%%% Chapter 2.1
\chapter{Microstates and Entropy}



%%%%%%%%%%%%%%%%%%%%%% Chapter 2.2
\chapter{Ensemble Theory and Free Energy}



%%%%%%%%%%%%%%%%%%%%%% Chapter 2.3
\chapter{Boltzmann Statistics and the Canonical Ensemble}



%%%%%%%%%%%%%%%%%%%%%% Chapter 2.4
\chapter{Breakdown of Classical Statistical Mechanics}



%%%%%%%%%%%%%%%%%%%%%%%%%%%%%%%%%%%%% Part 3
\part{Probability Theory}


%%%%%%%%%%%%%%%%%%%%%% Chapter 3.1
\chapter{Characteristics of Probability Theory}


%%%%%%%%%%%%%%%%%%%%%% Chapter 3.2
\chapter{Continuous Random Variables and the Gaussian Distribution}


%%%%%%%%%%%%%%%%%%%%%% Chapter 3.3
\chapter{Information and Entropy}





%%%%%%%%%%%%%%%%%%%%%%%%%%%%%%%%%%%%% Part 4
\part{Real Gases and Phase Transitions}


%%%%%%%%%%%%%%%%%%%%%% Chapter 4.1
\chapter{Kinetic Theory of Gases}



%%%%%%%%%%%%%%%%%%%%%% Chapter 4.2
\chapter{Classification of Phase Transitions}






%%%%%%%%%%%%%%%%%%%%%%%%%%%%%%%%%%%%% Part 5
\part{Quantum Statistics}


%%%%%%%%%%%%%%%%%%%%%% Chapter 5.1
\chapter{Quantum States}


%%%%%%%%%%%%%%%%%%%%%% Chapter 5.2
\chapter{Ideal Quantum Gases}








%%%%%%%%%%%%%%%%%%%%%% - Appendices
\begin{appendices}


\end{appendices}


\end{document}


%%%%%% END %%%%%%%%%%%%%
