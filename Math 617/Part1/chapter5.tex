%%%%%%%%%%%%%%%%%%%%% chapter.tex %%%%%%%%%%%%%%%%%%%%%%%%%%%%%%%%%
%
% sample chapter
%
% Use this file as a template for your own input.
%
%%%%%%%%%%%%%%%%%%%%%%%% Springer-Verlag %%%%%%%%%%%%%%%%%%%%%%%%%%
%\motto{Use the template \emph{chapter.tex} to style the various elements of your chapter content.}
\chapter{Weak Topologies}
\label{WeakTop} % Always give a unique label
% use \chaptermark{}
% to alter or adjust the chapter heading in the running head



\abstract{Summary of material in chapter (to be completed after chapter)}

\section{Duality}
\label{sec:dual}


Let $X$ be a LCS and let $X^*$ denote its dual. $X^*$ has a natural vector space structure. We introduce the \textbf{dual pairing} $$\langle,\rangle:X^*\times X\rightarrow \F$$
defined by $\langle f,x\rangle = f(x)$. In the case of $X$ being replaced by $X^*$, we have $\langle x,x^*\rangle = x^*(x)$ for $x \in X$ embedded into $X^{**}$, and $x^* \in X^*$. Note we can also write $x^*(x) = \langle x^*,x\rangle$, understanding $X$ to not be embedded in $X^{**}$.

\begin{defn}
    If $X$ is a LCS, the \textbf{weak topology} on $X$, denoted by \textbf{wk}, or $\sigma(X,X^*)$, is the topology defined by the family of seminorms $\{p_{x^*}:x^* \in X^*\}$, where $$p_{x^*}(x) = |\langle x,x^*\rangle|$$
    The \textbf{weak-star topology} on $X^*$, denoted \textbf{wk$^*$} or $\sigma(X^*,X)$, is the topology defined by the seminorms $\{p_x:x \in X\}$, where $$p_x(x^*) = |\langle x,x^*\rangle|$$
\end{defn}


So a subset $U$ of $X$ is \textbf{weakly open} if and only if for every $x_0 \in U$, there is an $\epsilon > 0$ and $x_1^*,...,x_n^* \in X^*$ such that $$\bigcap_{k=1}^n\{x \in X:|\langle x-x_0,x_k^*\rangle| < \epsilon\}\subseteq U$$
A \textbf{net} $\{x_i\}_{i \in I}$ in $X$ converges weakly to $x_0$ if and only if $\langle x_i,x^*\rangle \rightarrow \langle x_0,x^*\rangle$ for every $x^*$ in $X^*$.

Note that both $(X,\text{wk})$ and $(X^*,\text{wk}^*)$ are LCS's. Also, $X$ already possesses a topology so that $\text{wk}$ is a second topology on $X$. However, $X^*$ has no topology to being with so that $\text{wk}^*$ is the only topology on $X^*$.

If $\{x_i\}$ is a net in $X$ and $x_i\rightarrow 0$ in $X$, then for every $x^*$ in $X^*$, $\langle x_i,x^*\rangle\rightarrow 0$. So if $\mathcal{T}$ is the topology on $X$, $\text{wk} \subseteq \mathcal{T}$ and each $x^*$ in $X^*$ is weakly continuous.

\begin{thm}
    If $X$ is a LCS, $(X,\text{wk})^* = X^*$.
\end{thm}
\begin{proof}
    Since every weakly open set is open in the original topology, each $f$ in $(X,\text{wk})^*$ belongs to $X^*$. Conversely, if $f$ is in $X^*$, then by construction of the weak topology, $f$ is in $(X,\text{wk})^*$.
\end{proof}

\begin{thm}
    If $X$ is a LCS, $(X^*,\text{wk}^*)^* = X$.
\end{thm}
\begin{proof}
    If $x \in X$, $x^*\mapsto \langle x,x^*\rangle$ is a $\text{wk}^*$ continuous functional on $X^*$ by definition of the topology. Hence $X \subseteq (X^*,\text{wk}^*)^*$. Conversely, if $f \in (X^*,\text{wk}^*)^*$, then there are vectors $x_1,...,x_n \in X$ such that $$|f(x^*)| \leq \sum_{k=1}^n|\langle x_k,x^*\rangle|$$
    for all $x^* \in X^*$. This implies that $\bigcap\{\ker x_k:1\leq k \leq n\} \subseteq \ker f$. Then there are scalars $\alpha_1,...,\alpha_n$ such that $f = \sum_{k=1}^n\alpha_kx_k$; hence $f \in X$.
\end{proof}

So $X$ is the dual of a LCS, and hence has a weak-star topology $\sigma((X,\text{wk}^*),X^*)$. This is equivalent to the weak topology on $X$. 

Moving forward, all unmodified topological statements about $X$ refer to its original topology. 

\begin{thm}
    If $X$ is a LCS and $A$ is a convex subset of $X$, then $\text{cl}A = \text{wk-cl}A$.
\end{thm}
\begin{proof}
    If $\tau$ is the original topology of $X$, then $\text{wk} \subseteq \tau$, hence $\text{cl}A \subseteq \text{wk-cl}A$. Conversely, if $x \in X\backslash\text{cl}A$, then there is an $x^* \in X^*$, an $\alpha \in \R$, and an $\epsilon > 0$ such that $$\text{Re}\langle a,x^*\rangle \leq \alpha < \alpha + \epsilon \leq \text{Re}\langle x,x^*\rangle$$
    for all $a \in \text{cl}A$. Hence $\text{cl}A \subseteq B \equiv \{y \in X:\text{Re}\langle y,x^*\rangle \leq \alpha\}$. But $B$ is wk-closed since $x^*$ is wk-continuous. Thus $\text{wk-cl}A \subseteq B$. Since $x \notin B$, $x \notin\text{wk-cl}A$.
\end{proof}


\begin{cor}
    A convex subset of $X$ is closed if and only if it is weakly closed.
\end{cor}


\begin{defn}
    If $A \subseteq X$, the \textbf{polar} of $A$, denoted by $A^{\circ}$, is the subset of $X^*$ defined by $$A^{\circ} := \{x^* \in X^*:|\langle a,x^*\rangle| \leq 1,\forall a \in A\}$$
    If $B \subseteq X^*$, the \textbf{prepolar} of $B$, denoted ${}^{\circ}B$, is the subset of $X$ defined by $${}^{\circ}B = \{x \in X:|\langle x,b^*\rangle| \leq 1,\forall b^* \in B\}$$
    If $A \subseteq X$ the \textbf{bipolar} of $A$ is the set ${}^{\circ}(A^{\circ})$.
\end{defn}


\begin{prop}
    If $A \subseteq X$, then \begin{itemize}
        \item[(a)] $A^{\circ}$ is convex and balanced. 
        \item[(b)] If $A_1 \subseteq A$, then $A^{\circ} \subseteq A^{\circ}_1$.
        \item[(c)] If $\alpha \in \F$, and $\alpha \neq 0$, $(\alpha A)^{\circ} = \alpha^{-1}A^{\circ}$
        \item[(d)] $A \subseteq {}^{\circ}(A^{\circ})$
        \item[(e)] $A^{\circ}=({}^{\circ}(A^{\circ}))^{\circ}$
    \end{itemize}
\end{prop}
\begin{proof}
    Let $x^*,y^* \in A^{\circ}$ and $t \in [0,1]$. Then we have that for $a \in A$, $$|\langle a,tx^*+(1-t)y^*\rangle|\leq t|\langle a,x^*\rangle|+(1-t)|\langle a,y^*\rangle| \leq 1$$
    and for $s, |s| \leq 1$, $|\langle a,sx^*\rangle| \leq |\langle a,x^*\rangle| \leq 1$, so $tx^*+(1-t)y^*,sx^* \in A^{\circ}$.

    (b), (c), and (d) are immediate by construction. From (d) with $A = A^{\circ}$ we have that $A^{\circ} \subseteq ({}^{\circ}(A^{\circ}))^{\circ}$. But also from (d) $A \subseteq {}^{\circ}(A^{\circ})$, so $A^{\circ} \supseteq ({}^{\circ}(A^{\circ}))^{\circ}$, and hence we have equality.
\end{proof}

If $A$ is a linear manifold in $X$ and $x^* \in A^{\circ}$, then $ta \in A$ for all $t > 0$ and $a \in A$. So $1 \geq |\langle ta,x^*\rangle| = t|\langle a,x^*\rangle|$. Letting $t\rightarrow \infty$ shows that $A^{\circ} = A^{\perp}$ where $$A^{\perp} := \{x^* \in X^*:\langle a,x^*\rangle = 0,\forall a\in A\}$$
Similarly, if $B$ is a linear manifold in $X^*$, ${}^{\circ}B = {}^{\perp}B$.


\begin{nthm}{Bipolar Theorem}
    If $X$ is a LCS and $A \subseteq X$, then ${}^{\circ}A^{\circ}$ is the closed convex balanced hull of $A$.
\end{nthm}
\begin{proof}
    Let $A_1$ be the intersection of all closed convex balanced subsets of $X$ that contain $A$. It must be shown that $A_1 = {}^{\circ}A^{\circ}$. Since ${}^{\circ}A^{\circ}$ is closed, convex, and balanced and contains $A$, it contains $A_1$.

    Now assume that $x_0 \in X\backslash A_1$. $A_1$ is a closed convex balanced set, so there is an $x^* \in X^*$, an $\alpha \in \R$, and an $\epsilon > 0$ such that $$\text{Re}\langle a_1,x^*\rangle < \alpha < \alpha+\epsilon < \text{Re}\langle x_0,x^*\rangle$$
    for all $a_1 \in A_1$. Since $0 \in A_1$, $0 = \langle 0,x^*\rangle < \alpha$. By replacing $x^*$ with $\alpha^{-1}x^*$ it follows that there is an $\epsilon > 0$ such that $$\text{Re}\langle a_1,x^*\rangle < 1 < 1+\epsilon < \text{Re}\langle x_0,x^*\rangle$$
    for all $a_1 \in A_1$. If $a_1 \in A_1$ and $\langle a_1,x^*\rangle = |\langle a_1,x^*\rangle| e^{-i\theta}$, then $e^{-i\theta}a_1 \in A_1$ and so $$|\langle a_1,x^*\rangle| = \text{Re}\langle e^{-i\theta}a_1,x^*\rangle < 1 < \text{Re}\langle x_0,x^*\rangle$$
    for all $a_1 \in A_1$. Hence $x^* \in A_1^{\circ}$ and $x_0 \notin {}^{\circ}A^{\circ}$. That is, $X\backslash A_1\subseteq {}^{\circ}A^{\circ}$.
\end{proof}

\begin{cor}
    If $X$ is a LCS and $B \subseteq X^*$, then $({}^{\circ}B)^{\circ}$ is the wk$^*$ closed convex balanced hull of $B$.
\end{cor}


\begin{thm}
    If $X$ is a Banach space, $Y$ is a normed space, and $\mathscr{A} \subseteq \mathscr{B}(X,Y)$ such that for every $x \in X$, $\{Ax:A\in \mathscr{A}\}$ is weakly bounded in $Y$, then $\mathscr{A}$ is norm bounded in $\mathscr{B}(X,Y)$.
\end{thm}



\section{The Dual of a Subspace and a Quotient Space}
\label{sec:DualQuot}

\begin{prop}
    If $p$ is a seminorm on $X$, $M$ is a linear manifold in $X$, and $\overline{p}:X/M\rightarrow [0,\infty)$ is defined by $$\overline{p}(x+M)=\inf\{p(x+y):y \in M\}$$
    then $\overline{p}$ is a seminorm on $X/M$. If $X$ is a locally convex space and $\mathscr{P}$ is the family of all continuous seminorms on $X$, then the family $\overline{\mathscr{P}} = \{\overline{p}: p \in \mathscr{P}\}$ defines the quotient topology on $X/M$.
\end{prop}
\begin{proof}
    By construction $\overline{p}$ is well-defined, non-negative, and satisfies the triangle inequality and absolute homogeneity. Thus $\overline{p}$ is a seminorm on $X/M$. Suppose $X$ is a LCS and $\mathscr{P}$ is the family of all continuous seminorms on $X$. If $U$ is open in $X/M$ and $\pi$ is the projection, $\pi^{-1}(U)$ is open in $X$. Let $x \in \pi^{-1}(U)$. Then there exists $p_1,...,p_n$ and $\epsilon_1,...,\epsilon_n > 0$ such that $\bigcap_{i=1}^nB_{p_i,\epsilon_i}(x) \subseteq \pi^{-1}(U)$. Now let $y+M \in \bigcap_{i=1}^nB_{\overline{p_i},\epsilon_i}(x+M)$. For $\epsilon > 0$, there exists $m_1,...,m_n \in M$ such that $p_i(x-y+m_i) < \epsilon_i + \epsilon/n$. (\textbf{NOT DONE})
\end{proof}

If $X$ is a LCS and $M \leq X$, then $X/M$ is a LCS. Let $f \in (X/M)^*$. If $Q:X\rightarrow X/M$ is the natural map, then $f\circ Q \in X^*$. Moreover, $f\circ Q \in M^{\perp}$. Hence $f\mapsto f\circ Q$ is a map of $(X/M)^*\rightarrow M^{\perp}\subseteq X^*$.

\begin{thm}
    If $X$ is a LCS, $M \leq X$, and $Q:X\rightarrow X/M$ is the natural map, then $f\mapsto f\circ Q$ defines a linear bijection between $(X/M)^{\circ}$ and $M^{\perp}$. If $(X/M)^*$ has its weak-star topology $\sigma((X/M)^*,X/M)$ and $M^{\perp}$ has the relative weak-star topology $\sigma(X^*,X)\vert_{M^{\perp}}$, then this bijection is a homeomorphism. If $X$ is a normed space, then this bijection is an isometry.
\end{thm}
\begin{proof}
    Let $\rho:(X/M)^*\rightarrow M^{\perp}$ be defined by $\rho(f) = f\circ Q$. It was shown prior to the statement of the theorem that $\rho$ is well defined and maps $(X/M)^*$ into $M^{\perp}$. It is easy to see that $\rho$ is linear and if $0 = \rho(f) = f\circ Q$, then $f = 0$ since $Q$ is surjective. So $\rho$ is injective. Now let $x^* \in M^{\perp}$ and define $f:X/M\rightarrow \F$ by $f(x+M)=\langle x,x^*\rangle$. Because $M \subseteq \ker x^*$, $f$ is well defined and linear. Also, $Q^{-1}\{x+M:|f(x+M)|< 1 \} = \{x \in X:|\langle x,x^*\rangle| < 1\}$ and this is open in $X$ since $x^*$ is continuous. Thus $\{x+M:|f(x+M)| < 1\}$ is open in $X/M$ and so $f$ is continuous. Clearly $\rho(f) = x^*$, so $\rho$ is a bijection.

    If $X$ is a normed space, it was shown that $\rho$ is an isometry. It remains to show that $\rho$ is a weak-star homeomorphism. Let $\text{wk}^* = \sigma(X^*,X)$ and let $\sigma^* = \sigma((X/M)^*,X/M)$. If $\{f_i\}$ is a net in $(X/M)^*$ and $f_0\rightarrow 0$ in $\sigma^*$, then for each $x \in X$, $\langle x,\rho(f_i)\rangle = f_i(Q(x))\rightarrow 0$. Hence $\rho(f_i)\rightarrow 0$ in the $\text{wk}^*$ topology. Conversely, if $\rho(f_i)\rightarrow 0$ in $\text{wk}^*$, then for each $x \in X$, $f_i(x+M) = \langle x,\rho(f_i)\rangle \rightarrow 0$; hence $f_i\rightarrow 0$ in $\sigma^*$.
\end{proof}

Once again let $M \leq X$. If $x^* \in X^*$, then the restrction of $X^*$ ot $M$ belongs to $M^*$. Also, the Hahn-Banach Theorem implies that the map $x^*\rightarrow x^*\vert_M$ is surjective. If $\rho(x^*) = x^*\vert_M$, then $\rho$ is linear and surjective onto $M^*$. It fails, however, to be injective. But $\rho$ induces a linear bijection $\widetilde{\rho}:X^*/M^{\perp}\rightarrow M^*$.

\begin{thm}
    If $X$ is a LCS, $M \leq X$, and $\rho:X^*\rightarrow M^*$ is the restriction map, then $\rho$ induces a linear bijection $\widetilde{\rho}:X^*/M^{\perp}\rightarrow M^*$. If $X^*/M^{\perp}$ has the quotient topology induced by $\sigma(X^*,X)$ and $M^*$ has its weak-star topology $\sigma(M^*,M)$, then $\widetilde{\rho}$ is a homeomorphism. If $X$ is a normed space, then $\widetilde{\rho}$ is an isometry.
\end{thm}
\begin{proof}
    Let $\text{wk}^* = \sigma(M^*,M)$ and let $\eta^*$ be the quotient topology on $X^*/M^{\perp}$ defined by $\sigma(X^*,X)$. Let $Q:X^*\rightarrow Q^*/M^{\perp}$ be the natural map. Then $\widetilde{\rho} \circ Q = \rho$. If $y \in M$, then the commutativity implies that \begin{align*}
        Q^{-1}\{\widetilde{\rho}^{-1}\{y^*\in M^*:|\langle y,y^*\rangle| < 1\}) &= Q^{-1}\{x^*+M^{\perp}:|\langle y,x^*\rangle| < 1\} \\
        &= \{x^* \in X^*:|\langle y,x^*\rangle| < 1\}
    \end{align*}
    which is weak-star open in $X^*$. Hence $\widetilde{\rho}:(X^*/M^{\perp},\eta^*)\rightarrow (M^*,\text{wk}^*)$ is continuous.

    How is the topology on $X^*/M^{\perp}$ defined? If $x \in X$, $p_x(x^*) = |\langle x,x^*\rangle|$ is a typical seminorm on $X^*$. The topology on $X^*/M^{\perp}$ is defined by the seminorms $\{\overline{p}_x:x \in X\}$, where $$\overline{p}_x(x^*+M^{\perp}) = \inf\{|\langle x,x^*+z^*\rangle|:z^* \in M^{\perp}\}$$

    If $x \in M$, we aim to show $\overline{p}_x = 0$. In fact, let $\mathcal{L} = \{\alpha x:\alpha \in \F\}$. If $x \notin M$, then $\mathcal{L}\cap M = (0)$. Since $\text{dim}\mathcal{L} < \infty$, $M$ is topologicall complementd in $\mathcal{L}+M$. Let $x^* \in X^*$ and define $f:\mathcal{L}+M\rightarrow \F$ by $f(\alpha x+y) = \langle y,x^*\rangle$ for $y \in M$ and $\alpha \in \F$. Because $M$ is topologically complemented in $\mathcal{L}+M$, if $\alpha_ix+y_i\rightarrow 0$, then $y_i\rightarrow 0$. Hence $f(\alpha_ix+y_i)=\langle y_i,x^*\rangle \rightarrow 0$. Thus $f$ is continuous. By the Hahn-Banach Theorem, there is an $x_1^* \in X^*$ that extends $f$. Note that $x^* - x_1^* \in M^{\perp}$. Thus $\overline{p}_x(x^*+M^{\perp}) = \overline{p}_x(x_1^*+M^{\perp}) \leq p_x(x_1^*) = |\langle x,x_1^*\rangle | = 0$. This proves the claim.


    Now suppose that $\{x_i^*+M^{\perp}\}$ is a net in $X^*/M^{\perp}$ such that $\overline{\rho}(x_i^*+M^{\perp}) = x_i^*\vert_M\rightarrow 0$ in $\text{wk}^*$ in $M^*$. If $x \in X$ and $x \notin M$, then $\overline{p}_x(x_i^*+M^{\perp}) = 0$. If $x \in M$, then $\overline{p}_x(x_i^*+M^{\perp}) \leq |\langle x,x_i^*\rangle| \rightarrow 0$. Thus $x_i^*+M^{\perp}\rightarrow 0$ in $\eta^*$ and $\widetilde{\rho}$ is a weak-star homeomorphism.
\end{proof}


\section{Alaoglu's Theorem}
\label{sec:Alaoglu}

If $X$ is any normed space, let $\overline{B}_1^X(0)$ denote the closed unit ball.

\begin{nthm}{Alaoglu's Theorem}
    If $X$ is a normed space, then $\overline{B}_1^{X^*}(0)$ is weak-star compact.
\end{nthm}
\begin{proof}
    For each $x \in \overline{B}_1^X(0)$, let $D_x := \{\alpha \in \F:|\alpha| \leq 1\}$ and put $D = \prod_{x \in \overline{B}_1^X(0)}D_x$. By Tychonoff's Theorem, $D$ is compact with the product topology. Define $\tau:\overline{B}_1^{X^*}(0)\rightarrow D$ by $$\tau(x^*)(x) = \langle x,x^*\rangle$$
    I claim $\tau$ is a homeomorphism from $(\overline{B}_1^{X^*}(0),\text{wk}^*)$ onto $\tau(\overline{B}_1^{X^*}(0))$ with the relative topology from $D$, and that $\tau(\overline{B}_1^{X^*}(0))$ is closed  in $D$. Thus it will follow that $\tau(\overline{B}_1^{X^*}(0))$, and hence $\overline{B}_1^{X^*}(0)$, is compact.

    Note the image of two elements under $\tau$ being equal implies their values on any element of $\overline{B}_1^X(0)$ is equal. Using linearity we find the original maps are equal.

    Now let $\{x_i^*\}$ be a net in $\overline{B}_1^{X^*}(0)$ such that $x_i^*\rightarrow x^*$. Then for each $x$ in $\overline{B}_1^X(0)$, $\tau(x_i^*)(x) = \langle x,x_i^*\rangle\rightarrow \langle x,x^*\rangle = \tau(x^*)(x)$. That is, each coordinate of $\{\tau(x_i^*)\}$ converges to $\tau(x^*)$. Hence $\tau(x_i^*)\rightarrow \tau(x^*)$ and $\tau$ is continuous. 

    Let $x_i^*$ be a net in $\overline{B}_1^{X^*}(0)$, let $f \in D$, and suppose $\tau(x_i^*)\rightarrow f$ in $D$. So $f(x) = \lim\langle x,x_i^*\rangle$ exists for every $x \in \overline{B}_1^X(0)$. If $x \in X$, let $\alpha > 0$ such that $\norm{\alpha x}\leq 1$. Then define $f(x) = \alpha^{-1}f(\alpha x)$. If also $\beta > 0$ such that $\norm{\beta x} \leq 1$, then $\alpha^{-1}f(\alpha x) = \alpha^{-1}\lim\langle \alpha x,x_i^*\rangle = \beta^{-1}\lim\langle \beta x,x_i^*\rangle = \beta^{-1}f(\beta x)$. So $f(x)$ is well defined.  It can be shown that $f:X\rightarrow \F$ is a linear functional. Also, if $\norm{x} \leq 1$, $f(x) \in D_x$, so $|f(x)| \leq 1$. Thus $x^* \in \overline{B}_1^{X^*}(0)$ and $\tau(x^*) = f$. Thus $\tau(\overline{B}_1^{X^*}(0))$ is closed in $D$. This implies the image is compact. Further, it can be shown $\tau^{-1}$ is continuous, which completes the proof.
\end{proof}




% Problems or Exercises should be sorted chapterwise
\section*{Problems}
\addcontentsline{toc}{section}{Problems}
%
% Use the following environment.
% Don't forget to label each problem;
% the label is needed for the solutions' environment
\begin{prob}
\label{prob1}
A given problem or Excercise is described here. The
problem is described here. The problem is described here.
\end{prob}


% \begin{prob}
% \label{prob2}
% \textbf{Problem Heading}\\
% (a) The first part of the problem is described here.\\
% (b) The second part of the problem is described here.
% \end{prob}

%%%%%%%%%%%%%%%%%%%%%%%% referenc.tex %%%%%%%%%%%%%%%%%%%%%%%%%%%%%%
% sample references
% %
% Use this file as a template for your own input.
%
%%%%%%%%%%%%%%%%%%%%%%%% Springer-Verlag %%%%%%%%%%%%%%%%%%%%%%%%%%
%
% BibTeX users please use
% \bibliographystyle{}
% \bibliography{}
%


% \begin{thebibliography}{99.}%
% and use \bibitem to create references.
%
% Use the following syntax and markup for your references if 
% the subject of your book is from the field 
% "Mathematics, Physics, Statistics, Computer Science"
%
% Contribution 
% \bibitem{science-contrib} Broy, M.: Software engineering --- from auxiliary to key technologies. In: Broy, M., Dener, E. (eds.) Software Pioneers, pp. 10-13. Springer, Heidelberg (2002)
% %
% Online Document

% \end{thebibliography}

