%%%%%%%%%%%%%%%%%%%%% chapter.tex %%%%%%%%%%%%%%%%%%%%%%%%%%%%%%%%%
%
% sample chapter
%
% Use this file as a template for your own input.
%
%%%%%%%%%%%%%%%%%%%%%%$% Springer-Verlag %%%%%%%%%%%%%%%%%%%%%%%%%%
%\motto{Use the template \emph{chapter.tex} to style the various elements of your chapter content.}
\chapter{The Vestal Habit}
\label{VestHab} % Always give a unique label
% use \chaptermark{}
% to alter or adjust the chapter heading in the running head


%%% Questions to think about
%The \textbf{Thesis} or general sense of the article is ...

%The \textbf{method} the author uses to argue their point is ...

%In their \textbf{analysis} the author uses tools such as ...
% How do they look at the evidence? Do they place it in some theoretical framework? (i.e. gender studies, music studies, etc.)

%Additionally they conclude ...
% How does this compare to others throughout time? What is the societal context?

%What connections does the author portray with regard to \textbf{space}, \textbf{relationships}, \textbf{occupation}, and \textbf{religion}.


\abstract{}

\section{Questions and Remarks}
\label{sec:QR10}

\begin{qst}
    What is the \textbf{Vestal Habit}?
\end{qst}


\begin{qst}
    Who were the \textbf{Vestal virgins}?
\end{qst}


\begin{qst}
    Who is \textbf{Vesta}?
\end{qst}




\section{First Reading}
\label{sec:FirRead10}


\textbf{Mary Beard} interpreted the Vestal virgins through the proposition that by combining features relating to the status of unmarried daughters (\textbf{virgines}) with those of married women (\textbf{matronae}), the priestesses became in themselves vessels for the symbolic mediation between culturally opposed categories that \textbf{Claude L\'{e}vi-Strauss}, \textbf{Mary Douglas}, and others have identified as a central function of myth and ritual. 

Scholars who identified the origins of the Vestals' priesthood in the religious duties of the wives of Rome's early kings have pointed to those articles of clothing that are also characteristic of matrons (their \textbf{vittae}, or headbands, and \textbf{stola}, or floor-length gown) and brides (the ``six-lock" hairstyle, or \textbf{seni crines}, and belt tied in a square knot, or \textbf{nodus Herculaneus}).

\begin{rmk}
    \textbf{Beard} read these sartorial markers as evidence for a symbolic affinity with the category of matrons, which produced the above-mentioned ambiguity when combined with the obvious fact that the Vestals' physical virginity precluded them from the status of mothers and wives.
\end{rmk}

\textbf{Nina Mekacher} emphasizes the hybrid nature of the Vestals' ensemble, which she claims marked the priestesses out as \textbf{sui generis}.

\begin{rmk}
    \textbf{Vittae}, the cloth ribbons with which the Vestals bound their hair, were perhaps the most recognizable feature of their attire. These headbands are clearly depicted in almost all of the visual representations of the priestesses that have come down to us.
\end{rmk}

The Vestal vittae is most probably what Dionysius of Halicarnassus is referring to when he recounts the story of \textbf{Aemilia}, a priestess in the third century BCE, miraculously revived \textbf{Vesta}'s fire by removing a linen strap from her clothing and placing it on the hearth.

It is also true that \textbf{vittae} were closely associated with the status of the Roman matron.

\begin{rmk}
    \textbf{Valerius Maximus} suggests that the privilege of wearing these headbands was granted to the \textbf{ordo matronarum} by the senate in recognition of the role played by \textbf{Coriolanus'} wife and mother in averting the exiled commander's wrath from Rome.
\end{rmk}

\textbf{Vittae} in themselves do not necessarily denote any particular stage of the female life course. The \textbf{vittae} of the Vestal virgins might be seen as creating a unique connection with the costume of the Roman bride. Although the nature of the \textbf{seni crines} has become a source of much scholarly confusion and debate, the author says the best explanation of the term may be to read it as referring to the division of the hair into six strands, three on either side of the head, which were then braided into pigtails.

\begin{rmk}
    The dress of the Vestals corresponded to that of brides in the way that their gowns were girded as well.
\end{rmk}

Describing the woolen belt (\textbf{cingillum}) worn by brides, \textbf{Festus} explains that it was secured with the \textbf{Herculanean knot}, the same as seen in visual depicts ofthe priestesses. This lends particular weight to an interpretation of the priestesses as icons of ambivalence:
\begin{quotation}
    like the girl on the day of her wedding, they are seen as on the brink between virginal and marital status, but perpetually on the brink, perpetually fixed at the moment of transition from one category to another.
\end{quotation}

There were key elements in the dress of brides that the Vestals did not share. Most notably, the Vestals did not veil themselves with the bright yellow \textbf{flammeum}, as brides did. When Vestals appeared on ritual occasions, their heads were covered with the \textbf{suffibulum}, a cowl-like, shoulder-length veil, which was white with a purple border and fastened in front with a brooch.

The Vestals also wore the long flowing \textbf{stola} associated with matrons instead of the wide-cut \textbf{tunica recta} (or \textbf{regilla}) of the bride. \textbf{Festus} informs us that the bridal garment in question did not have an exclusive association with a signle category of individuals: it was also worn by boys when they received the \textbf{toga virilis}. The \textbf{tunica recta} marked the transition into adulthood that was common to both, even as the normative expectations of Roman gender roles required that this rite of passage operated quite differently for males and females.


The most intriguing example of this principle of \textbf{boundary-crossing} is the \textbf{toga}, which not only marked the status of the adult (male) Roman citizen, with all its pride and consequence, but was also the garment assigned to female prostitutes and adulteresses.

\textbf{Cicero} aptly exploits the inherent ambivalence of the toga in his account of \textbf{Antony}'s dissolute youth: 
\begin{quotation}
    You took up the toga of a man, which you promptly rendered womanly. At first, you were a common whore; the price of your shame was fixed, and it was not small. But soon Curio came along and took you out of the prostitute's trade, as if he had given you a \textbf{stola} and settled you in a stable and steady marriage.
\end{quotation}

In this passage the difference between \textbf{toga virilis} and \textbf{toga muliebris} lies not in the garment itself, but in the nature of the person who wears it.

\begin{qst}
    Why was the toga, when worn by a woman, the mark of a whore in Roman culture?
\end{qst}

Let us place the \textbf{stola}, the common dress of Vestals and matrons, on one side and the toga, as worn by adult men and whores, on the other. Viewed in this way the social categories associated with these garments produce a clear axis of sexual restraint or availability. The sexuality of the Vestal virgins was obviously the most restricted. 
\begin{rmk}
    To violate the chastity of a Vestal was to commit \textbf{incestum}, a crime punished with death.
\end{rmk}

Similarly, adulterous sexual activity involving a matron was a form of \textbf{stuprum}, which had been dealt with harshly. By covering their bodies with the long, flowing stola when they appeared in public, both matrons and Vestals signaled the prohibitions that governed their sexuality.

\begin{nte}
    Under the terms of an old law recorded by \textbf{Valerius Maximus}, the message conveyed by the stola was ``hands off".
\end{nte}


Any matron who failed in her duty to wear the stola should be punished as if she had committed stuprum according to \textbf{Caecina Severus} and \textbf{Tertullian}. Conversely, an adulteress was forbidden to wear the stola.

\begin{nte}
    The toga left one shoulder bare and did not fully cover the legs. Therefore the toga marked an absence of external constraints on an adult wearer's sexuality.
\end{nte}

For the Romans it was only ``natural" that adult men should be in charge of their own sexuality. The assumption of the \textbf{toga virilis} signaled one's transformation from the status of \textbf{puer}, who was potentially vulnerable to penetration, into a mature and fully-formed sexual subject. Roman women were by definition denied this kind of subjectivity. According to the normative gender categories available to them, the only meaningful control that could be exercised over their sexuality consisted in placing it off limits. 

\begin{nte}
    The toga marked women's bodies as lacking the chastity that was necessary for them to secure a respectable position in society.
\end{nte}


Gradations of fabric, cut, and color might differentiate the clothing of particular groups ( and thus mark out the cross-dresser), but there is nothing inherently transgressive about wearing a garment that is appropriate to one's station. 

The maiden \textbf{Cloelia} has a famous equestrian statue along the Sacria Via which is suggested to depict her ``wrapped in a toga". The honors bestowed on this heroine all but assimilated her to the status of a man, whose characteristic \textbf{virtus} she more than equaled. Cloelia was a \textbf{virgo} whose heroism saved not only herself, but also other vulnerable children (\textbf{impubes}) from the threat of defloration by their Etruscan captors. 

\begin{rmk}
    We msut recognize that Cloelia's toga was not that of an adult (man or prostitue), but rather it should be understood as the purple-bordered \textbf{toga praetexta} worn by freeborn children of both genders. It has nothing to do with the paradox of her virginal virtus.
\end{rmk}

The \textbf{toga praetexta} was also worn by curule magistrates and certain priests, providing another instance of overlapping categories. \textbf{Lynn Sebesta} suggests that the purple border of such garments signified the chastity associated with tis wearer.

This proposed connection explains why children regularly served as ritual ministrants and acolytes, as well as why the Vestal virgins wore a purple-bordered \textbf{suffibulum} when participating in sacrificial rites.

As a mark of status available to matrons, brides, and virgins alike, vittae were used to contrast the honorable status of these female roles with that of prostitutes, whom Servius explains were not allowed to wear hairbands, and other females of dubious integrity, such as slaves. \textbf{Ovid} describes thin \textbf{vittae} together with the long \textbf{instita} or lower border of the \textbf{stola} as a ``sign of chastity".

\begin{rmk}
    Just as the wearing of the \textbf{stola} signaled the difference between the chaste and unchaste woman's body, the binding of one's hair with vittae accomplished a similar distinction for females of all ages.
\end{rmk}

\textbf{Festus} suggests: 

\begin{quotation}
    Brides are adorned with the \textbf{seni crines} because it was the[ir] most ancient form of adornment. Certain [writers have argued?] that they do so because the Vestal virgins are adorned with it, whose chastity [brides emulate?] for their husbands in marriage[?].
\end{quotation}

Jist as the vittae that bound the hair were a sign of chastity, a belt (whether that of a Vestal or the woolen \textbf{cingulum} of a bride) tided with such a sturdy knot as the \textbf{nodus Herculaneus} would signify a sexuality that was especially well guarded.

\subsection{Conclusion}

The author has argued that overlapping boundaries were a normal feature of the Roman system of dress as a whole. The similarities between the Vestals' costume and that of other kinds of women therefore did not necessarily render their position in Roman society ambiguous.

The \textbf{stola} or \textbf{vittae} should be understood as marking some aspect of identity that women of each status held in common. For the clothing of the Vestal virgins the author had argued that this point of correspondence was their sexual purity.

\textbf{Cicero} ascribes an exemplary function to the Vestal cult: ``there are six virgins to attend upon [Vesta] so that...women may perceive that it is the nature of women to submit wholly to chastity".

Confined not just by the social conventions that governed their sex but by strict religious regulations as well, the Vestal virgins held out more strongly than most women against the changing trends of fashion, thereby providing an essential anchor for Roman attitudes regarding the connection between a woman's chastity and her manner of dress.

\begin{rmk}
    The same senators who sought to penalize matrons for going out without their \textbf{stolae} in the reign of Tiberius also bestowed a gift of two million \textbf{sestertii} on the newly inducted Vestal virgin Cornelia in 23 CE.
\end{rmk}




\section{Notes on Analysis and Societal Context}
\label{sec:SocCont10}


The author seeks to revisit the problem of the priestesses' ambiguity, paying closer attention to the socially contingent nature of its production. 

\begin{rmk}
    The author confines themselves to an aspect of gender construction that other scholars regarded as essential to establishing the \textbf{matronal} piece of the Vestals' supposed \textbf{interstitiality}: their ritual costume.
\end{rmk}

The author notes that systematic analysis of the significance of particle articles of clothing is difficult. The author hopes to shed new light on the way that Romans used clothing to express concepts of gender and sexuality more generally using their analysis of the Vestal virgins.

\begin{nte}
    The author argues that each element of the Vestals' wardrobe helped to reinforce the importance of basic principles of sexual decorum, according to which gender roles were typically constructed in Roman society.
\end{nte}





\section{Terms}
\label{sec:terms10}

\begin{enumerate}
	\item
\end{enumerate}


%
% \begin{acknowledgement}
% If you want to include acknowledgments of assistance and the like at the end of an individual chapter please use the \verb|acknowledgement| environment -- it will automatically render Springer's preferred layout.
% \end{acknowledgement}
%
% \section*{Appendix}
% \addcontentsline{toc}{section}{Appendix}
%


% Problems or Exercises should be sorted chapterwise
\section*{Problems}
\addcontentsline{toc}{section}{Problems}
%
% Use the following environment.
% Don't forget to label each problem;
% the label is needed for the solutions' environment
\begin{prob}
\label{prob1}
A given problem or Excercise is described here. The
problem is described here. The problem is described here.
\end{prob}

% \begin{prob}
% \label{prob2}
% \textbf{Problem Heading}\\
% (a) The first part of the problem is described here.\\
% (b) The second part of the problem is described here.
% \end{prob}

%%%%%%%%%%%%%%%%%%%%%%%% referenc.tex %%%%%%%%%%%%%%%%%%%%%%%%%%%%%%
% sample references
% %
% Use this file as a template for your own input.
%
%%%%%%%%%%%%%%%%%%%%%%%% Springer-Verlag %%%%%%%%%%%%%%%%%%%%%%%%%%
%
% BibTeX users please use
% \bibliographystyle{}
% \bibliography{}
%


% \begin{thebibliography}{99.}%
% and use \bibitem to create references.
%
% Use the following syntax and markup for your references if 
% the subject of your book is from the field 
% "Mathematics, Physics, Statistics, Computer Science"
%
% Contribution 
% \bibitem{science-contrib} Broy, M.: Software engineering --- from auxiliary to key technologies. In: Broy, M., Dener, E. (eds.) Software Pioneers, pp. 10-13. Springer, Heidelberg (2002)
% %
% Online Document

% \end{thebibliography}

