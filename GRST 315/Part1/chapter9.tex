%%%%%%%%%%%%%%%%%%%%% chapter.tex %%%%%%%%%%%%%%%%%%%%%%%%%%%%%%%%%
%
% sample chapter
%
% Use this file as a template for your own input.
%
%%%%%%%%%%%%%%%%%%%%%%$% Springer-Verlag %%%%%%%%%%%%%%%%%%%%%%%%%%
%\motto{Use the template \emph{chapter.tex} to style the various elements of your chapter content.}
\chapter{House and Veil in Ancient Greece}
\label{HouseVeil} % Always give a unique label
% use \chaptermark{}
% to alter or adjust the chapter heading in the running head


%%% Questions to think about
%The \textbf{Thesis} or general sense of the article is ...

%The \textbf{method} the author uses to argue their point is ...

%In their \textbf{analysis} the author uses tools such as ...
% How do they look at the evidence? Do they place it in some theoretical framework? (i.e. gender studies, music studies, etc.)

%Additionally they conclude ...
% How does this compare to others throughout time? What is the societal context?

%What connections does the author portray with regard to \textbf{space}, \textbf{relationships}, \textbf{occupation}, and \textbf{religion}.


\abstract{}

\section{Questions and Remarks}
\label{sec:QR9}

\begin{qst}
    What is the Dream Book \textbf{Oneirokritika}?
\end{qst}

\begin{qst}
    What is Tegidion?
\end{qst}


\section{First Reading}
\label{sec:FirRead9}


\subsection{Primamry Considerations: Symbols and Ideologies}

\textbf{Artemidoros} of \textbf{Daldis} in his Dream Book (\textbf{Oneirokritika}) collected and collated a wealth of reports concerning people's dreams. He provides interpretations of the dreams and impacts on the dreamer's waking world. Artemidoros states that all objects which surround a person, cloacks, tunics, houses, walls, ships, etc., must signify one another (if one appears broken in a dream, another may break in the waking world).

\begin{rmk}
    There is a rich symbolism which was attached to and shared by dress and domestic space in the ancient Greek world. This is particularly evident in regard to women's domestic space and to female clothing; an investigation into the ways in which the Greeks observed and even named parts of the house and items of dress will quickly reveal that the association was very much at the front of the Greek mind and was, in fact, an important component of Greek gender ideology.
\end{rmk}

\textbf{Plutarch} uses the symbolic motif of a tortoise to demonstrate the notion of the shared connection between the covering created by clothes and the covering created by a house. He comments that 

\begin{quotation}
    \textbf{Pheidias} represented the Aphrodite of the \textbf{Elians} as stepping on a tortoise to typify for womanhood staying at home and keeping silent.
\end{quotation}

He suggests a woman should stay at home in security and silence, but when necessity forces her to leave the house, she, like the tortoise, should symbolically carry her house with her and should always act as though she were still passively secreted indoors.

\begin{nte}
    A female's garments become an extension of her living-space.
\end{nte}

By \textbf{veil} we consider an unstitched garment, like a mantle or cloack, which has the capacity to be pulled up onto the head and, if required, across the lower face. 


\subsection{House-Veil: Archaeology, Etymology, and Iconography}

Literary and artistic information testifies to the use of shutters, wooden panels and textiles as curtains and screens in Greece.Archaelogical evidence of such objects is next to none due to largely being made of perishable materials.

The excavators of the houses in the city of Olynthos in northern Greece noted a total absence of pivot-holes in the paved rooms. It is feasible to imagine that hangings were used instead of doors. \textbf{Pollux} refers to `curtains at the doors of bedchambers', \textbf{Theophrastos} mentions `rings for embroidered hangings', and curtain-rings are also mentioned by \textbf{Pliny}.

\begin{nte}
    The common word \textbf{epiblema} has the general meaning of `that which is thrown over' or `covering', but is more specifically linked with a tapestry or wall-hanging, while it simultaneously means `outer garment', `mantle' and by extension, `veil'. A variant of the word, \textbf{epibles}, is used for a cross-beam in a roof, seeming to link a woman's head-covering and a roof.
\end{nte}

The widespread veil-word \textbf{kalumma} is often associated with vocabulary referring to the house and its decoration. But \textbf{kalumma} is also applied to roof-beams and to window shutters, again amplifying the association between the veil and the house.

\begin{rmk}
    The common veil-word \textbf{kredemnon} also translates as `city walls', `towers' or `battlements', and in an epic context, the female veil and the defensive walls of a city-state are regarded as systematically analogous.
\end{rmk}

Another veil-word is \textbf{eruma}, which also means `fence', `fortress' or `bulwar' and more generally `protection'. \textbf{Sophokles} uses the word to describe the walls of Troy. The Iliadic phrase \textbf{kredemnon luesthai}, `to loose a veil/covering/wall', is used as a vivid metaphor for the sacking of a city and for the breaching of a woman's chastity.

The most common word for `roof' in ancient Greek was \textbf{tegos}, with diminutive \textbf{tegidion} meaning `little roof', a word also defined by lexicographer \textbf{Hesychios} as `a manner of adorning the heads of women'. Figures wearing the \textbf{tegidion} have been found in Attica, Boeotia, and Macedonia, the length of the coast of Asia Minor and in the Greek cities of the Levant, Egypt, and Libya.

\begin{qst}
    What was the \textbf{tegidion}?
\end{qst}
What the terracota figures actually show is a face-veil composed by cutting eye-holes into a single rectangular cloth, which is sometimes edged with a delicate fringe. 

The word \textbf{tegidion} appears on an inscription dated to the third century BCE. A statuette from \textbf{Myrina} (circa 250 BCE) shows a woman wearing a mantle-veil that has been pulled around her lower face to mask her mouth and chin and is secured on the right side by being tucked into the headband of the \textbf{tegidion} which is worn in conjunction.

The cloth of the tegidion is folded off the face and back onto the head to form a flat surface with overhanging eaves resembling a little gabled roof. 

\begin{nte}
    The standard depiction of a gesture whereby a woman raises a portion of her veil usually shows it being held out at arm's length, so that it forms a large flap of cloth that frames her profiled face.
\end{nte}

Often we see imagery of a closed door and a lifted veil with a woman peering out. THe outstretched hand that lifts the veil intensifies the effect of the open door, and her clothing, like her house, demarcates the privacy of the female body, a space ideally removed from the public gaze. 

\begin{rmk}
    The house and the veil keep a woman modestly and safely enclosed.
\end{rmk}


\subsection{Veiling and Privacy}

As the busybody penetrates through the door of the house he `unveils' its occupants to his unwanted and shaming gaze and defiles the sanctity of privacy that the house usually offers. Contained within the protective walls of their house the mistress and her unmarried daughters normally have no need for the further protection offered by the veil; but when interrupted in their daily routine by a strange man, their lack of veils leaves them even more exposed and vulnerable to the gaze of the intruder.

\begin{rmk}
    For men to enter into a space that is currently in use by females is a discreditable act that brings dishonour on the violated family and particularly shames the women.
\end{rmk}

The veil is a `logical supplement' to the use of enclosed living spaces.

\begin{nte}
    In ancient Greece the women who attract the most notoriety are those who are conspiciously uncovered to the public view: lower-class prostitutes who are at the call of all men and do not enjoy the protection of a husband or guardian come in for particular attack.
\end{nte}


As one comic fragment attests, `their door is open' (about prostitutes).

\begin{rmk}
    As an extension of domestic space and a symbol of separation, the veil, in any form, enables women to move out of their homes in a kind of portable domestic space, and as a result, despite the modern Western perception of its negative aspects, the veil can be considered a liberating garment that frees a woman from the confines of any form of patriarchal \textbf{purdah} and lets her operate in the public sphere.
\end{rmk}


\subsection{The Tegidion and Female Visibility}

Although a social requirement for a woman to cover her face is an extension of the ideological complex that obliges women to cover their heads, it is a drastically greater step to have an item of clothing specifically designed to cover the face; face-veiling is concomitantly significant.  The \textbf{tegidion} was introduced into the Greek world at the close of the fourth century BCE. 

A veil that was specifically designed to cover the female face was a significant move towards the public control of female sexuality as a guarantor of male honour, yet the tegidion first appears at a time when it is generally assumed that women's lives were literally opening up as they began to take increasingly confident strides into public life.

From the late fourth century, a new group of large and elaborate elite houses began to appear, suggesting that the `status of the \textbf{oikos} and the role of the house were undergoing a rapid change in many areas of the Greek world'. This revision of domestic space resulted in the physical separation of the house into two separate areas, one for domestic activity and one public.

\begin{rmk}
    For the first time we can probably use the terms \textbf{andron} and \textbf{gynaikonitis} in the manner of the ancient sources.
\end{rmk}

\textbf{Nevett} suggests that `contrary to what the epigraphic record appears to suggest, women's status did not improve during the Hellenistic to early Roman periods'. Throughout the Hellenistic period we have reports in some Greek cities of civic bodies known as the \textbf{gynaikonomoi} (`controllers of women') who may have ensured that women's public appearances were policed.

If the veil is an extension of domestic space, then the increasing separation of women from the male world reflected in the two-courtyard house finds an astonishing parallel in the use of the face-veil in the same period. At this time Greek artists start to produce full-sized nude female statuary in the form of a variety of Aphrodite figures. But it should be emphasised that nudity did not transfer from the divine realm to the world of mortal women.

\begin{rmk}
    Although the tegidion may have been another device to control the autonomy of Hellenistic women, it could also have allowed women more freedom to participate in public society.
\end{rmk}



\subsection{Conclusions}

The common conception that the female veil acted as a logical extension of private domestic space can be demonstrated in a rich linguistic and visual symbolism where veils are frequently likened to shells, walls, doors, and roofs.


\section{Notes on Analysis and Societal Context}
\label{sec:SocCont9}

The author suggests that the connection between the veil and the house is a pertinent theme in Greek gender constructions and that the symbolic association between these two facets of female daily experience can be located in both literary and material evidence.


\section{Terms}
\label{sec:terms9}

\begin{enumerate}
	\item
\end{enumerate}



%
% \begin{acknowledgement}
% If you want to include acknowledgments of assistance and the like at the end of an individual chapter please use the \verb|acknowledgement| environment -- it will automatically render Springer's preferred layout.
% \end{acknowledgement}
%
% \section*{Appendix}
% \addcontentsline{toc}{section}{Appendix}
%


% Problems or Exercises should be sorted chapterwise
\section*{Problems}
\addcontentsline{toc}{section}{Problems}
%
% Use the following environment.
% Don't forget to label each problem;
% the label is needed for the solutions' environment
\begin{prob}
\label{prob1}
A given problem or Excercise is described here. The
problem is described here. The problem is described here.
\end{prob}

% \begin{prob}
% \label{prob2}
% \textbf{Problem Heading}\\
% (a) The first part of the problem is described here.\\
% (b) The second part of the problem is described here.
% \end{prob}

%%%%%%%%%%%%%%%%%%%%%%%% referenc.tex %%%%%%%%%%%%%%%%%%%%%%%%%%%%%%
% sample references
% %
% Use this file as a template for your own input.
%
%%%%%%%%%%%%%%%%%%%%%%%% Springer-Verlag %%%%%%%%%%%%%%%%%%%%%%%%%%
%
% BibTeX users please use
% \bibliographystyle{}
% \bibliography{}
%


% \begin{thebibliography}{99.}%
% and use \bibitem to create references.
%
% Use the following syntax and markup for your references if 
% the subject of your book is from the field 
% "Mathematics, Physics, Statistics, Computer Science"
%
% Contribution 
% \bibitem{science-contrib} Broy, M.: Software engineering --- from auxiliary to key technologies. In: Broy, M., Dener, E. (eds.) Software Pioneers, pp. 10-13. Springer, Heidelberg (2002)
% %
% Online Document

% \end{thebibliography}

