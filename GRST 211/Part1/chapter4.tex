%%%%%%%%%%%%%%%%%%%%% chapter.tex %%%%%%%%%%%%%%%%%%%%%%%%%%%%%%%%%
%
% sample chapter
%
% Use this file as a template for your own input.
%
%%%%%%%%%%%%%%%%%%%%%%$% Springer-Verlag %%%%%%%%%%%%%%%%%%%%%%%%%%
%\motto{Use the template \emph{chapter.tex} to style the various elements of your chapter content.}
\chapter{Skin, Hair, and Nails}
\label{Skin} % Always give a unique label
% use \chaptermark{}
% to alter or adjust the chapter heading in the running head


%%% Questions to think about
%The \textbf{Thesis} or general sense of the article is ...

%The \textbf{method} the author uses to argue their point is ...

%In their \textbf{analysis} the author uses tools such as ... 

%Additionally they conclude ...

%What connections does the author portray with regard to \textbf{space}, \textbf{relationships}, \textbf{occupation}, and \textbf{religion}.


\abstract{}


\section{Notes}
\label{sec:NOTE4}

\subsection{Prefixes}

\begin{longtable}{c | p{0.4\textwidth} | p{0.4\textwidth}}
    \caption{Prefixes for the skin, hair, and nails.}
    \hline
    Prefix & Meaning(s) & Example(s) \\ \hline
        amphi-, ampho- & `both,' `on both sides' & amphibian, amphopomorphic \\
        an- & `without,' `not,' `non-' & anaerobic, anacid \\
        dia- & `through,' `apart,' `in a line' & diamagnetism, diagnosis \\
        extra-, extro- & `outside of,' `beyond' & extraordinary, extrapolate \\
        in- & `in,' `into,' `over' & into, inside, inner \\
        meta- & `after,' `change,' `transition' & metamorphosis \\
        non- & `not' & nonconformist, nonsmoker \\
        ob- & `toward,' `in front of,' `against' & obstruction, obtrusive \\
        par- & `beside,' `beyond,' `abnormal' & particular, particle, part \\
        per- & `through' & permission, permit \\
        poly- & `many,' `much' & polyamorous, polyhedra \\
        re- & `again and again,' `backward' & rewind, reverse \\
        semi- & `half,' `partly' & semiautomatic, semifinished \\
        syn- & `together,' `with,' `concurrent' & synthesis \\
    \label{tab:Ch4Prefix}
\end{longtable}


\subsection{Suffixes}

\begin{longtable}{c | p{0.4\textwidth} | p{0.4\textwidth}}
    \caption{Suffixes for the skin, hair, and nails.}
    \hline
    Suffix & Meaning(s) & Example(s) \\ \hline
        -aceous & `pertaining to,' `belonging to,' `having' & herbaceous \\
        -atic & `pertaining to' & pragmatic, automatic \\
        -ation & `process' & illustration \\
        -atory & `pertaining to' & lavatory \\
        -cle & `small' & cubicle \\
        -cyte & `cell' & electrocyte \\
        -escent & `beginning to be,' `becoming' & florescent, adolescent \\
        -esis & `condition,' `abnormal condition,' `process' & synthesis, cyesis \\
        -gen & `that which produces' & microgen, allogen \\
        -in & `substance' & creatin \\
        -ment & `action of,' `product of' & moment, achievement, excitement \\
        -oma & `tumor,' `mass' & melagnoma, glaucoma, hematoma \\
        -ose & `full of,' `having the quality of' & fibrose, comotose \\
        -ula & `small' & scapula \\
        -um & `structure,' `substance' & symposium, compendium \\
    \label{tab:Ch4Suffix}
\end{longtable}


\subsection{Bases}

\begin{longtable}{c | p{0.4\textwidth} | p{0.4\textwidth}}
    \caption{Bases for the skin, hair, and nails.}
    \hline
    Bases & Meaning(s) & Example(s) \\ \hline
        CUT-, CUTANE- & `skin' & cuticle (CUT-I-cle) - small skin (i.e. an edge of skin covering the nail bed, or a thin layer of skin), cutaneous (CUTANE-ous) - pertaining to skin, subcutaneous (sub-CUTANE-ous) - pertaining to below the skin, extracutaneous (extra-CUTANE-ous) - pertaining to outside the skin (i.e. not affecting the skin) \\
        TECT-, TEG- & `to cover' & tectorium (TECT-orium) - a place for covering (i.e. a covering structure or layer), integument (in-TEG-U-ment) - product of over covering (i.e. something that coverse or encloses, especially a skin or membrane) \\
        SARC- & `flesh,' `soft tissue' & sarcoma (SARC-oma) - a tumor of the soft tissues, sarcoid (SARC-oid) - resembling flesh, sarcoidosis (SARC-oid-osis) - abnormal condition of tumors of the soft tissues (note sarcoid used to also refer to tumors of the soft tissues) \\
        THEL- & (i) `nipple': (ii) `cellular layer,' `tissue' & epithelium (epi-THEL-I-um) - structureon the surface tissue (i.e. a tissue that coversall surfaces, including theskin) \\
        CYT- & `cell & cytoid (CYT-oid) - resembling a cell \\
        HIST-, HISTI- & `tissue' & histoid (HIST-oid) - resembling tissue, histiocyte (HISTI-O-cyte) - cel within tissue (i.e. an immune cell that destroys foreign substances that remains within the tissue) \\
        KERAT- & (i) `horn,' `horny tissue' (ii) `cornea' (of the eye) & keratin (KERAT-in) - substance (composed of) horny tissue (i.e. a hard, fibrous protein), keratosis (KERAT-osis) - abnormal condition of the horny tissue, keratoid (KERAT-oid) - resembling horny tissue, resembling corneal tissue \\
        MELAN- & `black' & melanin (MELAN-in) - substance that is black \\
        COLL- & `glue' & colloid (COLL-oid) - like glue, collagen (COLL-A-gen) - that which produces glue \\
        ELAST- & `flexible,' `stretchy' & elastin (ELAST-in) - substance that is stretchy, elastoma (ELAST-oma) - a mass composed of elastic (tissue), nonelastic (non-ELAST-ic) - pertaining to not flexible \\
        SUD-, SUDOR- & `sweat' & sudation (SUD-ation) - process of sweating, suderesis (SUDOR-esis) - abnormal condition of sweating \\
        FER- & `to bear,' `to carry,' `to produce' & dorsiferous (DORS-I-FER-ous) - pertaining to carrying on the back, sudoriferous (SUDOR-I-FER-ous) - pertaining to producing sweat \\
        HIDR- & `sweat' & synhidrosis (syn-HIDR-osis) - condition of concurrent sweating (i.e. the association of sweating along with some other symptom), anhidrosis (an-HIDR-osis) - condition of not sweating (i.e. the inability to sweat normally) \\
        SPIR- & (i) `to breathe': (ii) `coil' & perspiration (per-SPIR-ation) - process of through breathing (i.e. excretion through (the skin)), respiratory (re-SPIR-atory) - pertaining to again and again breathing, spiroid (SPIR-oid) - resembling a coil \\
        LIP- & `fat' & lipocyte (LIP-O-cyte) - cell containing fat, lipoma (LIP-oma) - tumor containing fat (cells) \\
        ADIP- & `fat' & adiposis (ADIP-osis) - abnormal condition of fat (deposit), adipose (ADIP-ose) - full of fat \\
        STEAR-, STEAT- & `fat' & stearic (STEAR-ic) - pertaining to fat, steatosis (STEAT-osis) -abnormal condition of fat (deposit) \\
        TRICH- & `hair' & schizotrichia (SCHIZ-O-TRICH-ia) - condition of hair that is split, melanotrichous (MELAN-O-TRICH-ous) - pertaining to hair that is black, amphitrichous (amphi-TRICH-ous) - pertaining to on both sides hair \\
        PIL- & `hair' & pilose (PIL-ose) - full of hair (i.e. hairy), piliferous (PIL-I-FER-ous) - pertaining to bearing or producing hair, depilation (de-PIL-ation) - process of without hair (i.e. process for removal of hair) \\
        MEDULL- & `innermost part,' `medulla' & medullary (MEDULL-ary) - pertaining to the innermost part \\
        CORT-, CORTIC- & `outer layer,' `cortex' & cortical (CORTIC-al) - pertaining to the outer layer \\
        FOLL- & `sac,' `container' & follicle (FOLL-I-cle) - small sac \\
        SEB- & `grease,' `tallow' & sebum (SEB-um) - substance (that is like) grease, sebaceous (SEB-aceous) - pertaining to grease \\
        UNGU- & `nail' & ungual (UNGU-al) - pertaining to a nail, unguiferate (UNU-I-FER-ate) - pertaining to bearing nails, polyunguia (poly-UNGU-ia) - condition of many nails (i.e. having extra nails on fingers or toes) \\
        ONYCH- & `nail' & paronychia (par-ONYCH-ia) - condition beside the nail, polyonychia (poly-ONYCH-ia) - condition of many nails, melanonychia (MELAN-ONYCH-ia) - condition of a nail that is black \\
        LUN- & `moon,' `moon shaped' (almost always crescent shaped, like a crescent moon) & lunate (LUN-ate) - pertaining to moon shaped, semilunar (semi-LUN-ar) - pertaining to half-moon shaped, lunula (LUN-ula) - small moon \\
        CHRO-, CHROM-, CHROMAT- & `color' & metachroic (meta-CHRO-ic) - pertaining to change in color, dyschromia (dys-CHROM-ia) - condition of abnormal color (i.e. discoloration, especially of skin or nails), achromatic (a-CHROMAT-ic) - pertaining to without color \\
        NIGR- & `black' & nigrescent (NIGR-escent) - beginning to be black \\
        ALB-, ALBID- & `white', `whitish' & albescent (ALB-escent) - beginning to be white \\
        LEUC-, LEUK- & `white' & leukotrichia (LEUK-O-TRICH-ia) - condition of hair that is white, leukocyte (LEUK-O-cyte) - cell that is white (i.e. a white (blood) cell) \\
        CAN- & `white,' `gray' & canescent (CAN-escent) & beginning to be white or gray \\
        CAND- & `white,' `glowing white' & candle \\
        POLI- & `gray' & trichopoliosis (TRICH-O-POLI-osis) - condition of graying of the hair \\
        ERYTHR- & `red' & erythrodermic (ERYTHR-O-DERM-ic) - pertaining to skin that is red, erythrocyte (ERYTHR-O-cyte) - cell that is red (i.e. a red blood cell (in an immature stage)) \\
        ROSE- & `rosy-red,' `pink' & roseate (ROSE-ate) - having rosy-red (coloration) \\
        RHOD- & `rosy-red' & rhodescent (RHOD-escent) - beginning to be rosy-red \\
        PURPUR- & `purple' & purpuric (PURPUR-ic) - pertaining to a purple (condition) (i.e. pertaining to purpura, a condition that at one stage is characterized by a purple discoloration of the skin), purpuriferous (PURPUR-I-FER-ous) - pertaining to producing (visual) purple \\
        PORPHYR- & `purple' & porphyrin (PORPHYR-in) - substance that is purple, porphyria (PORPHYR-ia) - condition of purple \\
        FUSC- & `brown,' `dark' & fuscin (FUSC-in) - substance that is brown (i.e. a brown pigment in the retina of the eye), obfuscate (ob-FUSC-ate) - to cause to (bring) toward darkness (i.e. to cause things to be unclear) \\
        CIRRH- & `yellow,' `tawny' & cirrhosis (CIRRH-osis) - condition of yellow (discoloration) \\
        FLAV- & `golden yellow,' `reddish yellow' & flavin (FLAV-in) - substance that is reddish yellow (i.e. a variety of yellow pigment) \\
        LUTE- & `yellow' & lutein (LUTE-in) - substance that is yellow (i.e. a chemical substance isolated from egg yolk), luteal (LUTE-al) - pertaining to the corpus luteum (the `yellow body,' a structure formed in the ovary after ovulation) \\
        XANTH- & `yellow' & xanthosis (XANTH-osis) - condition of yellowing (of the skin) \\
        CHLOR- & `green' & chlorosis (CHLOR-osis) - condition of green (discoloration of the skin) \\
        GLAUC- & `bluish-gray,' `silvery-gray' & glaucescent (GLAUC-escent) - beginning to be bluish-gray \\
        CYAN- & `blue' & cyanosis (CYAN-osis) - condition of blue (discoloration of the skin) \\
        TEPHR- & `ashes,' `color of ashes' & \\
        CINER-, CINE- & `ashes,' `color of ashes' & \\ 
        PHAEO-, PHEO- & `color of the sky at twilight' (i.e. `gray' or `dusky') & \\
        EOSIN- & `color of the sky at dawn' (i.e. a rosy-red hue) & \\
    \label{tab:Ch4Bases}
\end{longtable}


\subsection{More Things}

Recall that the prefix `an-' is just a modified form of the prefix `a-'. The form `an-' is used when the following base begins with a vowel, or the letter `h.' Many Greek prefixes that end with a vowel have alternative forms to use before bases that begin with vowels or the letter `h.' Usually, this involves dropping off the final vowel of the prefix. This process is known as \textbf{elision}.


\begin{longtable}{c | p{0.4\textwidth} | p{0.4\textwidth}}
    \caption{Prefixes with elisions.}
    \hline
    Prefix & Meaning(s) & Example(s) \\ \hline
        a-, an- & `not,' `without,' `non-' & anaerobic \\
        anti-, ant- & `against,' `opposite' & antihero \\
        apo-, ap- & `away from' & apogy, apart \\
        dia-, di- & `through,' `apart,' `in a line' & direct, diagram \\
        ecto-, ect- & `outside,' `outer' & ectoplasm, \\
        endo-, end- & `inside,' `inner' & endomorphism \\
        epi-, ep- & `upon,' `on the surface' & epidermal \\
        hypo-, hyp- & `below,' `deficient,' `less than normal' & hypoallergenic \\
        meta-, met- & `after,' `change,' `transition' & metamorphosis, \\
        para-, par- & `beside,' `beyond,' `abnormal' & paradigm \\ 
    \label{tab:Ch4Prefix2}
\end{longtable}

\section{Questions and Remarks}
\label{sec:QR4}






%
% \begin{acknowledgement}
% If you want to include acknowledgments of assistance and the like at the end of an individual chapter please use the \verb|acknowledgement| environment -- it will automatically render Springer's preferred layout.
% \end{acknowledgement}
%
% \section*{Appendix}
% \addcontentsline{toc}{section}{Appendix}
%


% Problems or Exercises should be sorted chapterwise
\section*{Problems}
\addcontentsline{toc}{section}{Problems}
%
% Use the following environment.
% Don't forget to label each problem;
% the label is needed for the solutions' environment
\begin{prob}
\label{prob1}
A given problem or Excercise is described here. The
problem is described here. The problem is described here.
\end{prob}

% \begin{prob}
% \label{prob2}
% \textbf{Problem Heading}\\
% (a) The first part of the problem is described here.\\
% (b) The second part of the problem is described here.
% \end{prob}

%%%%%%%%%%%%%%%%%%%%%%%% referenc.tex %%%%%%%%%%%%%%%%%%%%%%%%%%%%%%
% sample references
% %
% Use this file as a template for your own input.
%
%%%%%%%%%%%%%%%%%%%%%%%% Springer-Verlag %%%%%%%%%%%%%%%%%%%%%%%%%%
%
% BibTeX users please use
% \bibliographystyle{}
% \bibliography{}
%


% \begin{thebibliography}{99.}%
% and use \bibitem to create references.
%
% Use the following syntax and markup for your references if 
% the subject of your book is from the field 
% "Mathematics, Physics, Statistics, Computer Science"
%
% Contribution 
% \bibitem{science-contrib} Broy, M.: Software engineering --- from auxiliary to key technologies. In: Broy, M., Dener, E. (eds.) Software Pioneers, pp. 10-13. Springer, Heidelberg (2002)
% %
% Online Document

% \end{thebibliography}

