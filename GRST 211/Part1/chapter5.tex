%%%%%%%%%%%%%%%%%%%%% chapter.tex %%%%%%%%%%%%%%%%%%%%%%%%%%%%%%%%%
%
% sample chapter
%
% Use this file as a template for your own input.
%
%%%%%%%%%%%%%%%%%%%%%%$% Springer-Verlag %%%%%%%%%%%%%%%%%%%%%%%%%%
%\motto{Use the template \emph{chapter.tex} to style the various elements of your chapter content.}
\chapter{Eyes}
\label{Eyes} % Always give a unique label
% use \chaptermark{}
% to alter or adjust the chapter heading in the running head


%%% Questions to think about
%The \textbf{Thesis} or general sense of the article is ...

%The \textbf{method} the author uses to argue their point is ...

%In their \textbf{analysis} the author uses tools such as ... 

%Additionally they conclude ...

%What connections does the author portray with regard to \textbf{space}, \textbf{relationships}, \textbf{occupation}, and \textbf{religion}.


\abstract{}


\section{Notes}
\label{sec:NOTE5}


\subsection{Prefixes}

\begin{longtable}{c | p{0.4\textwidth} | p{0.4\textwidth}}
    \caption{Prefixes for the eyes.}
    \hline
    Prefix & Meaning(s) & Example(s) \\ \hline
        ante- & `before,' `in front of' & anterior, antenna \\
        cata-, cat- & `down,' `complete' & cataracts \\
        con- & `together,' `with' & connected, connections \\
        ec-, ex- & `out,' `outside' & exit, excited, eccentric \\
        eiso-, eso- & `inward' & esoteric \\
        ento-, ent- & `inside,' `within' & enter, entomology \\
        hemi- & `half' & hemisphere \\
        infra- & `below,' & infrared \\
        multi- & `many,' `much' & multifaceted \\
        post- & `behind,' `after' & posthumonously \\
        pros- & `toward' & prostate, prosperous \\
        supra- & `above' & supralunar \\
    \label{tab:Ch5Prefix}
\end{longtable}


\begin{longtable}{c | p{0.4\textwidth} | p{0.4\textwidth}}
    \caption{Suffixes for the eyes.}
    \hline
    Suffix & Meaning(s) & Example(s) \\ \hline
        -duct & `duct,' `channel,' `tube' & airduct \\
        -ema & `condition' & eczema \\
        -esce & `to begin,' `to become' & coalesce \\
        -form & `having the form of,' `like' & fungiform \\
        -gram & `record' & milligram \\
        -graph & `instrument used to record' & telegraph, holograph \\
        -iasis & `state of,' `process of,' `abnormal condition' & amebiasis \\
        -iatic & `pertaining to a state,' `pertaining to a process' & sciatic \\
        -ible, -ibil- & `able to be' & visible, plausible \\
        -ical & `pertaining to' & quizical \\
        -itis & `inflammation' & appendicitis \\
        -ize & `to make,' to affect' & apologize, hypnotize \\
        -meter & `instrument used to measure' & multimeter \\
        -scope & `instrument used to examine' & telescope, microscope \\
        -ual & `pertaining to' & usual \\
        -blepsia & `condition of sight' & \\
        -opia & `condition of sight' & \\
        -opsia & `condition of sight' & \\
        -graphy & `process of recording' & \\
        -metry & `process of measuring' & \\
        -scopy & `process of examining with an instrument' & \\
    \label{tab:Ch5Suffix}
\end{longtable}


\subsection{Bases}


\begin{longtable}{c | p{0.4\textwidth} | p{0.4\textwidth}}
    \caption{Bases for the eyes.}
    \hline
    Bases & Meaning(s) & Example(s) \\ \hline
        PHOT- & `light' & photodermatitis (PHOT-O-DERMAT-itis) - inflammation of the skin caused by exposure to light, photoeythema (PHOT-O-ERYTH-ema) - condition of reddening (of the skin due to) light, photometer (PHOT-O-meter) - instrument to measure light \\
        SCOT- & `darkness' & scotosis (SCOT-osis) - abnormal condition of darkness (in the visual field), scotophobia (SCOT-O-phobia) - abnormal fear of the darkness \\
        BLEP-, BLEPS- & `to see,' `sight' & visual (VIS-ual) - pertaining to seeing or sight, visible (VIS-ible) - able to be seen, visibility (VIS-ibil-ity) - state of being able to be seen, xenovision (XEN-o-VIS-ion) - act of seeing strange or foreign (things in the imagination) \\
        OP-, OPS-, OPT- & `eye,' `sight' & scotopia  (SCOT-OP-ia) - process of sight in the dark, hemianopsia (hemi-an-OPS-ia) - condition of half non-sight, entoptic (ent-OPT-ic) - pertaining to within the eye, optomeningeal (OPT-O-MENING-eal) - pertaining to the membrane of the eye (i.e. pertaining to the retina) \\
        PROSOP- & `face' & prosopic (PROSOP-ic) - pertaining to the face \\
        OCUL- & `eye,' `sight' & infraocular (infra-OCUL-ar) - pertaining to below the eye, oculofacial (OCUL-O-FACI-al) - pertaining to the face and eyes, postocular (post-OCUL-ar) - pertaining to behind the eye \\
        OPHTHALM- & `eye,' `sight' & exophthalmic (ex-OPHTHALM-ic) - pertaining to an outward eye, (i.e. pertaining to an abnormal protrusion of the eyeball from the socket), xenophthalmia (XEN-OPHTHALM-ia) - condition of the eye (caused by a) foreign (body), anophthalmia (an-OPHTHALM-ia) - condition of without eyes, opthalmoscope (OPTHALM-O-scope) - instrument to examine the eyes, opthalmograph (OPHTHALM-O-graph) - instrument used to record (movement) of the eyes \\
        ORBIT- & `wheel track,' `circle,' `ring' & orbital (ORBIT-al) - pertaining to a circular (path) \\
        ORBIT- & `orbit' (of the eye - the bony cavity that contains the eyeball and associated parts) - anteorbital (ante-ORBIT-al) - pertaining to in front of the orbit of the eye, orbitotemporal (ORBIT-O-TEMPOR-al) - pertaining to the temporal and orbital bones and regions, supraorbital (supra-ORBIT-al) - pertaining to above the orbit of the eye \\
        BLEPHAR- & `eyelid' & ablephary (a-BLEPHAR-y) - state of without eyelids, symblepharosis (sym-BLEPHAR-osis) - abnormal condition of together eyelids (i.e. the upper and lower eyelids stuck together, or to the eyeball), blepharogram (BLEPHAR-O-gram) - record of eyelid (movement) \\
        PALPEBR- & `eyelid' & postpalpebral (post-PALPEBR-al) - pertaining to behind the eyelid, palpebrate (PALPEBR-ate) - having eyelids, palpebration (PALPEBR-ation) - process with the eyelid (i.e. winking) \\
        GEN-, GENIT- & `to produce,' `to beget' & genesis (GEN-esis) - process of production, adipogenic (ADIP-O-GEN-ic) - pertaining to production of fat, metagenesis (meta-GEN-esis) - process of transitional production (i.e. alternation of generations), genital (GENIT-al) - pertaining to reproduction \\
        CILI- & `eyelid,' `eyelash,' `hair-like structures' & ciliary (CILI-ary) - pertaining to the eyelid or eyelashes, ciliogenesis (CILI-O-GEN-esis) - process of production of eyelashes or hair-like structures, multiciliate (multi-CILI-ate) - having many hair-like structures\\
        JUNCT- & `to join' & conjunctive (con-JUNCT-ive) - pertaining to together joined (i.e. joining, connecting) \\
        CONJUCTIV- & `conjuctiva' (membrane of the eye) & conjunctival (CONJUCTIV-al) - pertaining to the conjunctiva, conjunctivitis (CONJUNCTIV-itis) - inflimmation of the conjunctiva, conjunctivoma (CONJUNCTIV-oma) - tumor of the conjunctiva \\
        SCLER- & `hard,' `firm,' `thick' & sclerodermal (SCLER-O-DERM-al) - pertaining to the skin that is hard or thick, scleronychia (SCLER-ONYCH-ia) - condition of hardening or thickening of the nails \\
        SCLER- & `sclera' (the white of the eye) & scleritis (SCLER-itis) - inflammation of the sclera \\
        CORN-, CORNE- & `horn' & corneous (CORNE-ous) pertaining to horn, cornual (CORN-ual) - pertaining to horn \\
        CORNE- & `cornea' (of the eye) & corneal (CORNE-al) - pertaining to the cornea, corneoscleral (CORNE-O-SCLER-al) - pertaining to the sclera and cornea \\
        RET- & `net,' `network' & retiform (RET-I-form) - like a net \\
        RETIN- & `retina' (of the eye) & retinitis (RETIN-itis) - inflimmation of the retina, retinosis (RETIN-osis) - abnormal condition of the retina, retinoschisal (RETIN-O-SCHIS-al) - pertaining to a split or rupture within the retina \\
        CHOR-, CHORI- & `membrane' & choroid (CHOR-oid) - like a membrane, chorioid (CHORI-oid) - like a membrane \\
        CHOR-, CHORION-, CHOROID- & `chorion,' (fetal membrane) `choroid' (membrane that is part of the internal structure of the eye) & chorionic (CHORION-ic) - pertaining to the chorion, choroidal (CHOROID-al) - pertaining to the choroid, choroiditis (CHOROID-itis) - inflammation of the choroid \\
        TROCH- & `wheel,' `round shaped,' `pulley shaped' (TROCHLE-) & trochoid (TROCH-oid) - like a wheel, trochocephalia (TROCH-O-CEPHAL-ia) - condition of the head that is (abnormally) round shaped \\
        TROCHLE- & `trochlea' (any structure in which a loop acts like a pulley for a tendon to run through, or a grooved structure that acts like a pulley wheel) & trochlear (TROCHLE-ar) - pertaining to a trochlea, trochleiform (TROCHLE-I-form) - like a trochlea \\
        CYST- & `bladder,' `cyst,' `sac' & cystities (CYST-itis) - inflammation of the (urinary) bladder, cystic (CYST-ic) - pertaining to a bladder, cyst, or sac \\
        PHAC-, PHAK- & (i) `lentil bean': (ii) `eye lens' & phacoid (PHAC-oid) - resembling a lentil bean, resembling an eye lens, phakoma (PHAK-oma) - tumor associated with the eye lens, phacocystic (PHAC-O-CYST-ic) - pertaining to the sac (containing) the eye lens \\
        LENT-, LENTICUL- (diminutive - i.e. small) & (i) `lentil bean': (ii) `eye lens' & lentiform (LENT-I-form) - having the form of a lentil seed, having the form of an eye lens, lenticular (LENTICUL-ar) - pertaining to a lentil, pertaining to a lens \\
        IRID-, IRIS- & `rainbow' & iridesce (IRID-esce) - to become like a rainbow (i.e. having rainbow-like colors) \\
        IRID-, IRIS- & `iris' (colored portion of the eye) & iridesis (IRID-esis) - abnormal condition of the iris, iridoschisal (IRID-O-SCHIS-al) - pertaining to a split or rupture within the iris, irisopsia (IRIS-OPS-ia) - condition of seeing rainbow colors (around objects) \\
        UV- & `grape' & uviform (UV-I-form) - like a grape, uvula (UV-ula) - small grape (i.e. uvula, a small grape-like structure) \\
        UVUL- & `uvula' & uvulitis (UVUL-itis) - inflammation of the uvula \\
        UVE- & `uvea' (the iris, ciliary body, and the choroid in the eye) & uveal (UVE-al) - pertaining to the uvea, uveitis (UVE-itis) - inflammation of the uvea, uveomeningitis (UVE-O-MENING-itis) - inflammation of the meninges and the uvea \\
        STEN- & `narrow,' `contracted' & dermostenosis (DERM-O-STEN-osis) - condition of contraction of the skin, sclerostenosis (SCLER-O-STEN-osis) - condition of contraction caused by hardening or thickening (of tissue), stenophobia (STEN-O-phobia) - abnormal fear of narrow (spaces) \\
        TON- & `tone,' `tension,' `stretching' & tonia (TON-ia) - condition of tone (i.e. normal condition of tone or tension in a muscle), atony (a-TON-y) - quality of without tone, hypertonic (hyper-TON-ic) - pertaining to more than normal tone or tension, catatoia (cata-TON-ia) - condition of complete tension \\
        COR-, CORE- & `pupil' (of the eye) & dyscoria (dys-COR-ia) - condition of abnormal pupil, stenocoriasis (STEN-O-COR-iasis) - state of the pupil marked by narrowing \\
        PUPILL- & `pupil' & interpupillary (inter-PUPILL-ary) - pertaining to between the pupils, pupillatonia (PUPILL-a-TON-ia) - condition of without tone in the pupil \\
        LACRIM- & `tears' & lacrimatory (LACRIM-atory) - pertaining to tears, lacrimal (LACRIM-al) - pertaining to tears, lacrimation (LACRIM-ation) - process of (producing) tears \\
        DACRY- & `tears' & dacryogenic (DACRY-O-GEN-ic) - pertaining to producing tears, dacryocystic (DACRY-O-CYST-ic) - pertaining to the sac related to tears (the lacrimal sac), dacryocystitis (DACRY-O-CYST-itis) - inflammation of the sac related to tears \\
        VITR-, VITRE- & `glass' & vitric (VITR-ic) - pertaining to glass, vitreous (VITRE-ous) - like glass \\
        VITR-, VITRE- & `vitreous humor' (gelatinous, glassy-like, fluid between the lens and retina in the eye) & vitritis (VITR-itis) - inflammation of the vitreous humor, vitreoretinal (VITRE-O-RETIN-al) - pertaining to the retina and the vitreous humor \\
        HYAL- & `glass' & hyalophobia (HYAL-O-phobia) - abnormal fear of glass (objects), hyaloid (HYAL-oid) - resembling glass \\
        HYAL- & `vitreous humor' & hyalitis (HYAL-itis) - inflammation of the vitreous humor, hyalosis (HYAL-osis) - abnormal condition of the vitreous humor \\
        AQU-, AQUE- & `water,' `watery fluid' & aquaphobia (AQU-A-phobia) - abnormal fear of water, aqueduct (AQUE-duct) - a duct or channel that carries water, aqueous (AQUE-ous) - like water \\
        MEI-, MI- & `lesser,' `smaller' & meiosis (MEI-osis) - process of making lesser, miosis (MI-osis) - process of making smaller, miotic (MI-O-tic) - pertaining to the process of miosis \\
        MYOP- & `short-sighted,' `near-sighted' & myopia (MYOP-ia) - condition of near-sightedness, myopic (MYOP-ic) - pertaining to near sightedness \\
        TROP- & `to turn,' `turning' & phototropic (PHOT-O-TROP-ic) - pertaining to turning (in response) to the light, esotropia (eso-TROP-ia) - condition of inward turning (i..e a condition of the eyes in which one or both eyes turn inward) \\
        GRAPH- & `to write,' `to record' & dysgraphia (dys-GRAPH-ia) - condition of bad or difficult writing \\
        METR- & (i) `measurement': (ii) `womb,' `uterus' & metrical (METR-ical) - pertaining to measurement, metricize (METR-ic-ize) - to make pertaining to measurement \\
        SCOP- & `to view,' `to examine' & optoscopic (OPTO-SCOP-ic) - pertaining to examination of the eye \\
    \label{tab:Ch5Bases}
\end{longtable}




\section{Questions and Remarks}
\label{sec:QR5}






%
% \begin{acknowledgement}
% If you want to include acknowledgments of assistance and the like at the end of an individual chapter please use the \verb|acknowledgement| environment -- it will automatically render Springer's preferred layout.
% \end{acknowledgement}
%
% \section*{Appendix}
% \addcontentsline{toc}{section}{Appendix}
%


% Problems or Exercises should be sorted chapterwise
\section*{Problems}
\addcontentsline{toc}{section}{Problems}
%
% Use the following environment.
% Don't forget to label each problem;
% the label is needed for the solutions' environment
\begin{prob}
\label{prob1}
A given problem or Excercise is described here. The
problem is described here. The problem is described here.
\end{prob}

% \begin{prob}
% \label{prob2}
% \textbf{Problem Heading}\\
% (a) The first part of the problem is described here.\\
% (b) The second part of the problem is described here.
% \end{prob}

%%%%%%%%%%%%%%%%%%%%%%%% referenc.tex %%%%%%%%%%%%%%%%%%%%%%%%%%%%%%
% sample references
% %
% Use this file as a template for your own input.
%
%%%%%%%%%%%%%%%%%%%%%%%% Springer-Verlag %%%%%%%%%%%%%%%%%%%%%%%%%%
%
% BibTeX users please use
% \bibliographystyle{}
% \bibliography{}
%


% \begin{thebibliography}{99.}%
% and use \bibitem to create references.
%
% Use the following syntax and markup for your references if 
% the subject of your book is from the field 
% "Mathematics, Physics, Statistics, Computer Science"
%
% Contribution 
% \bibitem{science-contrib} Broy, M.: Software engineering --- from auxiliary to key technologies. In: Broy, M., Dener, E. (eds.) Software Pioneers, pp. 10-13. Springer, Heidelberg (2002)
% %
% Online Document

% \end{thebibliography}

